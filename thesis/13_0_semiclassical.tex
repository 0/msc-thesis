\chapter{Semiclassical IVR with PIGS}
\chaptermark{SC-IVR}

\label{chap:semiclassical}

Real-time correlation functions are of particular interest in chemistry, because they relate to experimental observables.
For example, the spectrum (\textit{i.e.} the Fourier transform) of the dipole-dipole correlation function of a molecule is proportional to the electromagnetic spectrum of that molecule~\cite[12,56]{zwanzig2001nonequilibrium},\cite[473]{mcquarrie1976statistical}.
The ability to computationally generate spectra for molecules in various chemical environments (such as clusters) could help with the identification of existing compounds and the creation of novel ones.

The time evolution of a quantum mechanical system is well-known and is described in the Schrödinger picture by the time-dependent Schrödinger equation~\cite[111]{dirac1981principles}
\begin{align}
	\dpd{}{t} \ket{\psi(t)} = -\frac{i}{\hbar} \hat{H} \ket{\psi(t)}.
\end{align}
Starting from an initial state $\ket{\psi}$, this has the formal solution
\begin{align}
	\ket{\psi(t)}
	&= e^{-\frac{i \hat{H} t}{\hbar}} \ket{\psi}.
\end{align}
We will refer to the operator
\begin{align}
	\hat{U}(t)
	&= e^{-\frac{i \hat{H} t}{\hbar}}
\end{align}
as the \emph{real-time propagator}.
We find it convenient to work in the Heisenberg picture, where we propagate the operators through time, rather than the wavefunctions.
Since we wish to preserve expectation values, for any operator $\hat{O}$, it must be that
\begin{align}
	\braket{\psi(t) | \hat{O} | \psi(t)}
	&= \braket{\psi | \hat{U}\adj(t) \hat{O} \hat{U}(t) | \psi}
	= \braket{\psi | \hat{O}(t) | \psi},
\end{align}
which gives us the usual definition for the time dependence of an operator,~\cite[315]{messiah1999quantum}
\begin{align}
	\hat{O}(t)
	&= \hat{U}\adj(t) \hat{O} \hat{U}(t)
	= e^{\frac{i \hat{H} t}{\hbar}} \hat{O} e^{-\frac{i \hat{H} t}{\hbar}}.
\end{align}

The correlation functions in which we are interested have the general form
\begin{align}
	C_{\hat{A} \hat{B}}(t)
	&= \mean{\hat{A}(t) \hat{B}}
	= \frac{\Tr{\hat{\rho} \hat{A}(t) \hat{B}}}{\Tr{\hat{\rho}}},
\end{align}
for a density operator $\hat{\rho}$ and some operators $\hat{A}$ and $\hat{B}$.
In the case that the state is a normalized ground state $\ket{0}$ with energy $E_0$, we may write this as
\begin{align}
	C_{\hat{A} \hat{B}}(t)
	&= \braket{0 | \hat{A}(t) \hat{B} | 0}
	= \braket{0 | e^{\frac{i \hat{H} t}{\hbar}} \hat{A} e^{-\frac{i \hat{H} t}{\hbar}} \hat{B} | 0}
	= e^{\frac{i E_0 t}{\hbar}} \braket{0 | \hat{A} \hat{U}(t) \hat{B} | 0}.
\end{align}
At $t = 0$, this is simply the ground state expectation value of $\hat{A} \hat{B}$.
However, at later times, we must take into account the real-time propagator $\hat{U}(t)$; to do this exactly is as difficult as diagonalizing $\hat{H}$, which is essentially impossible for interesting problems.

Consider a particle of mass $m$ in a potential $\hat{V}$.\footnote{
	In this \namecref{chap:semiclassical}, we look at only the one-dimensional problem, but all the discussion is applicable to any number of degrees of freedom with some modifications.
	Among other things, the scaling factor in the coherent state resolution of the identity is changed to $(2 \pi \hbar)^{-F}$ for $F$ degrees of freedom, $\gamma$ is turned into a matrix, and the HK prefactor is generalized to be in the form of the determinant of a matrix.
}
One approximation to the real-time propagator is given by the \nomencl{HK}{Herman--Kluk} propagator~\cite{miller2002alternate}
\begin{align}
	\hat{U}\hk(t)
	&= \frac{1}{2 \pi \hbar} \iint\! \dif p \dif q \,
			R\coh{p}{q}_t e^{\frac{i}{\hbar} S\coh{p}{q}_t}
			\ketbra{p\coh{p}{q}_t \, q\coh{p}{q}_t}{p \, q},
\end{align}
where $p\coh{p}{q}_t$ and $q\coh{p}{q}_t$ are phase space variables for classical trajectories, $p$ and $q$ are the initial conditions for those trajectories, $\ket{p \, q}$ is a coherent state\footnote{
	An introduction to coherent states may be found in ref.~\cite[242-245]{schulman1996techniques} and a thorough summary of their properties is available in ref.~\cite[99-106]{gardiner2004quantum}.
} of reciprocal width $\gamma$,
\begin{align}
	R\coh{p}{q}_t
	&= \sqrt{\frac{1}{2} \left(
			\dpd{p\coh{p}{q}_t}{p}
			+ \dpd{q\coh{p}{q}_t}{q}
			+ \frac{i}{\hbar \gamma} \dpd{p\coh{p}{q}_t}{q}
			- i \hbar \gamma \dpd{q\coh{p}{q}_t}{p}
		\right)}
\end{align}
is the HK prefactor,
\begin{align}
	S\coh{p}{q}_t
	&= \int_0^t\! \dif \tau \, (p\coh{p}{q}_\tau \dot{q}\coh{p}{q}_\tau - H\coh{p}{q})
	= \frac{1}{m} \int_0^t\! \dif \tau \, (p\coh{p}{q}_\tau)^2 - t H\coh{p}{q}
\end{align}
is the classical action, and
\begin{align}
	H\coh{p}{q}
	&= \frac{p^2}{2 m} + V(q)
\end{align}
is the classical Hamiltonian (total energy) for the given initial conditions.
This approximation is a \emph{semiclassical} one, because it relies on the combination of classical trajectories to produce time evolution that takes into account quantum effects, such as interference~\cite{gelabert2000log,thoss2001generalized}.
Unlike other semiclassical methods, this one does not suffer from the root search problem, because it is an initial value method; it is part of a family of methods known as \nomencl{SC-IVR}{Semiclassical Initial Value Representation}~\cite{gelabert2000log}.
At $t = 0$, it is exact:
\begin{align}
	\hat{U}\hk(t = 0)
	&= \frac{1}{2 \pi \hbar} \iint\! \dif p \dif q \, \ketbra{p \, q}{p \, q}
	= \hat{1}
	= \hat{U}(t = 0).
\end{align}
At later times, we need to obtain the aforementioned quantities from classical trajectories.
For all but the most trivial of problems, this will need to be done by numerical integration of the equations of motion with a time step $\dt\alt$.

For simplicity, we will consider $\hat{A} = \hat{B} = \hat{1}$, in which case the correlation function is trivially unity:
\begin{align}
	C_{\hat{1} \hat{1}}(t)
	&= e^{\frac{i E_0 t}{\hbar}} \braket{0 | \hat{U}(t) | 0}
	= 1.
\end{align}
This may appear to be useless, but it contains a quantity known as the \emph{survival amplitude}~\cite{issack2007semiclassical}
\begin{align}
	S_{\psi}(t)
	&= \braket{\psi | \hat{U}(t) | \psi}
	= \braket{\psi | \psi(t)},
\end{align}
which is the overlap of a wavefunction with itself at a later time.
The survival amplitude retains nearly all the difficulty associated with correlation functions, since it still contains the real-time propagator.

In order to verify our methods, we wish to look at the ground state survival amplitude
\begin{align}
	S_0(t)
	&= \braket{0 | \hat{U}(t) | 0}
	= e^{-\frac{i E_0 t}{\hbar}}
	= \cos{(\omega_0 t)} - i \sin{(\omega_0 t)}
\end{align}
for model systems, where $\omega_0 = E_0 / \hbar$ may also be determined by other means.
Our method of choice for finding the ground state is PIGS in the position representation, so we have some access to $\braket{q | 0}$, but not to $\braket{p \, q | 0}$.
However, the HK propagator is explicitly in the coherent state representation.
We may try to get around the difference in representations by inserting a resolution of the identity in the position representation, as in
\begin{subequations}
\begin{align}
	\braket{p \, q | 0}
	&= \int\! \dif q' \, \braket{p \, q | q'} \braket{q' | 0} \\
	&= \int\! \dif q' \, \expb{-\frac{\gamma}{2} (q' - q)^2 - \frac{i}{\hbar} p (q' - q)} \braket{q' | 0}.
\end{align}
\end{subequations}
When we try to use the HK propagator to obtain the ground state survival amplitude, we get
\begin{subequations}
\begin{align}
	S_0\hk(t)
	&= \braket{0 | \hat{U}\hk(t) | 0} \\
	&= \frac{1}{2 \pi \hbar} \iint\! \dif p \dif q \,
			R\coh{p}{q}_t e^{\frac{i}{\hbar} S\coh{p}{q}_t}
			\braket{0 | p\coh{p}{q}_t \, q\coh{p}{q}_t} \braket{p \, q | 0} \\
	&= \frac{1}{2 \pi \hbar} \iiiint\! \dif p \dif q\LLup \dif q \dif q\RRup \,
			R\coh{p}{q}_t e^{\frac{i}{\hbar} S\coh{p}{q}_t}
			\braket{0 | q\LLup} \braket{q\LLup | p\coh{p}{q}_t \, q\coh{p}{q}_t}
			\braket{p \, q | q\RRup} \braket{q\RRup | 0} \\
	&= \frac{1}{2 \pi \hbar} \iiiint\! \dif p \dif q\LLup \dif q \dif q\RRup \,
			R\coh{p}{q}_t \braket{0 | q\LLup} \braket{q\RRup | 0}
			\expb{-\frac{\gamma}{2} \left( (q\LLup - q\coh{p}{q}_t)^2 + (q\RRup - q)^2 \right)} \notag \\
	&\qquad\qquad\qquad\times
			\expb{\frac{i}{\hbar} \left( S\coh{p}{q}_t + p\coh{p}{q}_t (q\LLup - q\coh{p}{q}_t) - p (q\RRup - q) \right)}.
				\label{eq:survival-hk}
\end{align}
\end{subequations}
Naturally, one cannot hope to perform this integration analytically.
Should we try to use another method, we are faced with the fact that all the integrals are from $-\infty$ to $\infty$.
For the position coordinates, we at least have some hope of narrowing this range down thanks to the exponential decay of the ground state wavefunctions and the Gaussian from the coherent states.
For the momentum coordinate, however, there is no such possibility; instead, we will be forced to explore other ideas.


\section{Numerical integration}

Since we are working in only one spatial dimension, the integrals in \cref{eq:survival-hk} may be performed on a grid.
We begin with the assumption that we have four evenly-spaced grids whose elements are $p_i$, $q_j$, $q_k$, $q_\ell$ (corresponding to $p$, $q\LLup$, $q$, and $q\RRup$) and whose spacings are, respectively, $\DP$, $\DQ\alt$, $\DQ$, $\DQ\alt$.\footnote{
	Since the wavefunction must be evaluated at $q_j$ and $q_\ell$, we require that these grids be identical to keep things simple.
}
These grids are symmetric about zero and extend in each direction to $p\mx$, $q\mx\alt$, $q\mx$, $q\mx\alt$.
We may then approximate the integrals in \cref{eq:survival-hk} by sums:
\begin{align}
	S_0\hk(t)
	&\approx \frac{\DP (\DQ\alt)^2 \DQ}{2 \pi \hbar} \sum_{i,j,k,\ell}
			R\coh{p_i}{q_k}_t \braket{0 | q_j} \braket{q_\ell | 0}
			\expb{-\frac{\gamma}{2} \left( (q_j - q\coh{p_i}{q_k}_t)^2 + (q_\ell - q_k)^2 \right)} \notag \\
	&\qquad\qquad\qquad\qquad\qquad\times
			\expb{\frac{i}{\hbar} \left( S\coh{p_i}{q_k}_t + p\coh{p_i}{q_k}_t (q_j - q\coh{p_i}{q_k}_t) - p_i (q_\ell - q_k) \right)}.
\end{align}
For brevity, we introduce
\begin{subequations}
\begin{align}
	C
	&= \frac{\DP (\DQ\alt)^2 \DQ}{2 \pi \hbar}
	&
	w^j
	&= \braket{q_j | 0} \\
	R\coh{i}{k}_t
	&= R\coh{p_i}{q_k}_t
	&
	S\coh{i}{k}_t
	&= S\coh{p_i}{q_k}_t \\
	p\coh{i}{k}_t
	&= p\coh{p_i}{q_k}_t
	&
	q\coh{i}{k}_t
	&= q\coh{p_i}{q_k}_t,
\end{align}
\end{subequations}
which lets us write
\begin{align}
	S_{0,t}\hk
	&= C \sum_{i,j,k,\ell}
			R\coh{i}{k}_t w^j w^\ell
			\expb{-\frac{\gamma}{2} \left( (q_j - q\coh{i}{k}_t)^2 + (q_\ell - q_k)^2 \right)} \notag \\
	&\qquad\qquad\qquad\qquad\times
			\expb{\frac{i}{\hbar} \left( S\coh{i}{k}_t + p\coh{i}{k}_t (q_j - q\coh{i}{k}_t) - p_i (q_\ell - q_k) \right)}.
				\label{eq:survival-hk-num}
\end{align}
We refer to the 4-tuple $T\coh{i}{k}_t = \left( p\coh{i}{k}_t, q\coh{i}{k}_t, R\coh{i}{k}_t, S\coh{i}{k}_t \right)$ as the \emph{classical trajectory} with initial conditions $p_i$, $q_i$.

As mentioned earlier, we do not have as obvious a guideline for how to set up our momentum grid as we do for the position grids.
Taking a hint from the Fourier-transform-like form of the $p$ contribution in \cref{eq:survival-hk-num}, we choose the $p_i$ grid to be the momentum space grid corresponding to a position grid, but it is not immediately clear which position grid to use.
If we look only at the initial time, we see
\begin{align}
	S_{0,t=0}\hk
	&= C \sum_{i,j,k,\ell}
			w^j w^\ell
			\expb{
				-\frac{\gamma}{2} \left( (q_j - q_k)^2 + (q_\ell - q_k)^2 \right)
				+ \frac{i}{\hbar} p_i (q_j - q_\ell)
			},
\end{align}
which suggests that the position grid to use should be the $\{ q\mx\alt, \DQ\alt \}$ grid.\footnote{
	Technically, the range of possible values for the difference of the grid elements is twice the range for the grid itself, but the spacing remains the same, and that is what we care about here.
}
This results in~\cite{fattal1996phase}
\begin{align}
	p\mx
	= \frac{\pi \hbar}{\DQ\alt}.
		\label{eq:hk-pmax}
\end{align}
We will see later what happens if we neglect this and go farther out in momentum space.

Given the ground state wavefunction on the grid ($w$) and the classical trajectory at the appropriate time ($T\coh{i}{k}_t$), it is a straightforward matter to find $S_{0,t}\hk$.
All the details are in the propagation of the classical trajectory.
The phase space variables $p\coh{i}{k}_t$ and $q\coh{i}{k}_t$ may be found from the initial conditions\footnote{
	While the initial conditions are confined to a grid, there is no such restriction placed on the time-propagated phase space variables.
} using a symplectic integrator such as the well-known velocity Verlet algorithm or the lesser known (but higher order) integrator due to Ruth and Forest~\cite{forest1990fourth}.
The classical action $S\coh{i}{k}_t$ is not difficult to obtain once one has the momenta.

\begin{figure}
	\setlength{\figspacing}{5 mm}
	\centering
	\begin{subfigure}[b]{\textwidth}
		\includegraphics[width=\textwidth]{13/harmonic_oscillator_trajectory_pq}
		\caption{
			Convergence of phase space variables with time step.
		}
		\vspace{\figspacing}
	\end{subfigure}
	\begin{subfigure}[b]{\textwidth}
		\includegraphics[width=\textwidth]{13/harmonic_oscillator_trajectory_S}
		\caption{
			Convergence of classical action with time step.
		}
	\end{subfigure}
	\caption[
		Example classical trajectories for harmonic oscillator
	]{
		Some example phase space diagrams and classical actions resulting from harmonic oscillator trajectories with different time steps.
		\explainplotsas{}
	}
	\label{fig:harmonic-oscillator-trajectory-a}
\end{figure}

\begin{figure}
	\centering
	\includegraphics[width=\textwidth]{13/harmonic_oscillator_trajectory_R}
	\caption[
		Example HK prefactors for harmonic oscillator
	]{
		Some example HK prefactors resulting from harmonic oscillator trajectories with different time steps.
		\explainplotsas{}
	}
	\label{fig:harmonic-oscillator-trajectory-b}
\end{figure}

\begin{figure}
	\centering
	\includegraphics[width=\textwidth]{13/harmonic_oscillator_trajectory_sqrt}
	\caption[
		HK prefactors with incorrect square root branch
	]{
		The same HK prefactors as in \cref{fig:harmonic-oscillator-trajectory-b}, but the ``exact'' answers shown by the dashed curves neglect to take into account the branch of the square root.
		The result is a cusp in the real component and a discontinuity in the imaginary component at the point where the branches intersect, as well as the wrong overall sign after this point.
	}
	\label{fig:harmonic-oscillator-trajectory-sqrt}
\end{figure}

The HK prefactor, on the other hand, is a much more interesting quantity to calculate.
Because it is defined as the square root of a complex number, we must exercise caution when deciding which branch of the square root to choose.
However, that is not a tough problem to deal with so long as one is aware of it.
The more pressing matter is the evaluation of the partial derivatives.
If we write them in the form of a \emph{monodromy matrix}
\begin{align}
	\mat{M}_t
	&= \begin{pmatrix}
			\dpd{p_t}{p} & \dpd{p_t}{q} \\[3 mm]
			\dpd{q_t}{p} & \dpd{q_t}{q}
		\end{pmatrix}
	= \begin{pmatrix}
			m\coh{p}{p}_t & m\coh{p}{q}_t \\
			m\coh{q}{p}_t & m\coh{q}{q}_t
		\end{pmatrix}
\end{align}
(where we have dropped the trajectory indices $i$ and $k$ for the time being), we may express the time evolution of this matrix as~\cite{garashchuk2000simplified,gelabert2000log}
\begin{align}
	\dod{\mat{M}_t}{t}
	&= \begin{pmatrix}
			0 & -\nabla^2 V \\
			\frac{1}{m} & 0
		\end{pmatrix}
		\mat{M}_t
\end{align}
with initial conditions
\begin{align}
	\mat{M}_{t=0}
	&= \mat{1}.
\end{align}
From this, we obtain two pairs of coupled ordinary differential equations (for $x$ either $p$ or $q$):
\begin{subequations}
\begin{align}
	\dot{m}\coh{p}{x}_t
	&= -(\nabla^2 V)_t m\coh{q}{x}_t \\
	\dot{m}\coh{q}{x}_t
	&= \frac{1}{m} m\coh{p}{x}_t.
\end{align}
\end{subequations}
These are very similar to Hamilton's equations of motion, but the ``force'' is unfortunately time-dependent.
In order to propagate the HK prefactor through time, we may use an integrator such as the fourth-order Runge--Kutta integrator~\cite[710-713]{press1992numerical}.

\begin{DefExercise}{Harmonic oscillator classical trajectory}{ex:harmonic-oscillator-classical-trajectory}
	Find the analytical expressions for the classical action and HK prefactor for a harmonic oscillator with mass $m$, angular frequency $\omega$, and initial conditions $p$, $q$.
\end{DefExercise}

To summarize the above discussion and to provide verification of our implementation, we provide plots of trajectories (with different time steps $\dt\alt$) in phase space, along with the corresponding classical action and HK prefactor, for a harmonic oscillator (shown in \cref{fig:harmonic-oscillator-trajectory-a,fig:harmonic-oscillator-trajectory-b}).
We use $m = m_\mathrm{e}$ (mass of an electron), $\omega = \SI{1}{\kelvin}$, $\gamma = \SI{0.0181}{\per\square\nano\meter}$, $p = \SI{-5e-3}{\gram\nano\meter\per\pico\second\per\mole}$, and $q = \SI{50}{nm}$.
For reference, \cref{fig:harmonic-oscillator-trajectory-sqrt} shows the effect of neglecting to choose the correct branch for the square root in the HK prefactor.


\subsection{Harmonic oscillator}

\label{sec:semiclassical-numerical-ho}

We are now prepared to find the survival amplitude for a one-dimensional system, so we start with the harmonic oscillator defined by \vref{eq:harmonic-oscillator-hamiltonian}.
As elsewhere in the present work, we use $m = m_\mathrm{e}$ (mass of an electron) and $\omega = \SI{1}{\kelvin}$.
This still leaves us with several parameters to choose: $\dt\alt$, $\gamma$, $q\mx$, $\DQ$, $q\mx\alt$, $\DQ\alt$.
Unless otherwise specified, the parameter values from \cref{tab:model-sa0-harmonic-oscillator} are used for this model system.

One may wish to choose $\dt\alt$ to be sufficiently small that the classical trajectories are stable, but no smaller.
From \cref{fig:harmonic-oscillator-trajectory-a,fig:harmonic-oscillator-trajectory-b}, this appears to be about $\SI{5}{\pico\second}$.
However, since we would like to have good temporal resolution, we typically choose a shorter time step.
For this system, we may cheat and choose the optimal $\gamma$, one which matches that of the system itself: $\gamma = m \omega / \hbar$; however, we want to see the effect this parameter has on the results, so we also perturb it from this value.
Finally, we also need to obtain the ground state wavefunction on the $\{ q\mx\alt, \DQ\alt \}$ grid; because we can, we use the exact wavefunction from \vref{eq:ho-position-wf}.

\begin{table}
	\begin{center}
	\begin{tabular}{ c S[table-format=1.6] c c c c }
		\toprule
		{$\dt\alt / \si{\pico\second}$} & {$\gamma / \si{\per\square\nano\meter}$} & {$q\mx / \si{\nano\meter}$} & {$\DQ / \si{\nano\meter}$} & {$q\mx\alt / \si{\nano\meter}$} & {$\DQ\alt / \si{\nano\meter}$} \\
		\midrule
		1 & 0.001131 & 300 & 20 & 80 & 2 \\
		\bottomrule
	\end{tabular}
	\end{center}
	\caption[
		Selected parameters for harmonic oscillator (numerical)
	]{
		Selected parameters for the harmonic oscillator model system using the numerical method.
	}
	\label{tab:model-sa0-harmonic-oscillator}
\end{table}

As the first step, we examine the real part of $S_{0,t=0}\hk$, which should be exactly $1$ as long as our integrals have converged.
The convergence results for the position grids are shown in \cref{fig:harmonic-oscillator-survival-zero-q-a,fig:harmonic-oscillator-survival-zero-q-b}.
The results for $q\mx\alt$ are surprising: the wavefunction still has non-negligible amplitude (over \SI{2.5}{\percent} of the maximum amplitude) when it is truncated.

\begin{figure}
	\setlength{\figspacing}{5 mm}
	\centering
	\begin{subfigure}[b]{\textwidth}
		\includegraphics[width=\textwidth]{13/harmonic_oscillator_sa0_qmax}
		\caption{
			Convergence of $t = 0$ survival amplitude with $q\mx$.
		}
		\vspace{\figspacing}
	\end{subfigure}
	\begin{subfigure}[b]{\textwidth}
		\includegraphics[width=\textwidth]{13/harmonic_oscillator_sa0_dq}
		\caption{
			Convergence of $t = 0$ survival amplitude with $\DQ$.
		}
	\end{subfigure}
	\caption[
		Convergence of harmonic oscillator survival amplitude with position grids
	]{
		Convergence of $t = 0$ survival amplitude with the $\{ q\mx, \DQ \}$ grid for a harmonic oscillator.
		\explainplotsazero{}
	}
	\label{fig:harmonic-oscillator-survival-zero-q-a}
\end{figure}

\begin{figure}
	\setlength{\figspacing}{5 mm}
	\centering
	\begin{subfigure}[b]{\textwidth}
		\includegraphics[width=\textwidth]{13/harmonic_oscillator_sa0_qmax_alt}
		\caption{
			Convergence of $t = 0$ survival amplitude with $q\mx\alt$.
		}
		\vspace{\figspacing}
	\end{subfigure}
	\begin{subfigure}[b]{\textwidth}
		\includegraphics[width=\textwidth]{13/harmonic_oscillator_sa0_dq_alt}
		\caption{
			Convergence of $t = 0$ survival amplitude with $\DQ\alt$.
		}
	\end{subfigure}
	\caption[
		Convergence of harmonic oscillator survival amplitude with position grids \cont
	]{
		Convergence of $t = 0$ survival amplitude with the $\{ q\mx\alt, \DQ\alt \}$ grid for a harmonic oscillator.
		\explainplotsazero{}
	}
	\label{fig:harmonic-oscillator-survival-zero-q-b}
\end{figure}

As promised, we will now look at the results of increasing the $p$ grid past the proper $p\mx$.
\Cref{fig:harmonic-oscillator-survival-zero-pmax} shows a staircase as a function of $p\mx'$ (the actual extent used for the $p$ grid) with jumps at even integer multiples of $p\mx$.
To find the cause, in \cref{fig:harmonic-oscillator-survival-zero-p-aliasing} we plot the integrand which remains after the $q_j$ and $q_\ell$ integrals have been performed.
It seems that if we exceed the equivalent to the Nyquist frequency in momentum space, we observe aliasing!

\begin{figure}
	\centering
	\includegraphics[width=\textwidth]{13/harmonic_oscillator_sa0_pmax}
	\caption[
		Divergence of harmonic oscillator survival amplitude with momentum grid
	]{
		Divergence of $t = 0$ survival amplitude with $p\mx'$ for a harmonic oscillator.
		Dashed line indicates the exact answer; dotted line marks $p\mx' = p\mx$.
	}
	\label{fig:harmonic-oscillator-survival-zero-pmax}
\end{figure}

\begin{figure}
	\centering
	\begin{subfigure}{0.48\textwidth}
		\includegraphics[width=\textwidth]{13/harmonic_oscillator_sa0_p_aliasing_a}
		\caption{
			Correct extent for momentum grid.
			No aliases to be seen.
		}
	\end{subfigure}
	\hfill
	\begin{subfigure}{0.48\textwidth}
		\includegraphics[width=\textwidth]{13/harmonic_oscillator_sa0_p_aliasing_b}
		\caption{
			Incorrect extent for momentum grid.
			Two aliases are visible.
		}
	\end{subfigure}
	\caption[
		Aliasing of integrand in momentum space
	]{
		Aliasing of the partially integrated HK integrand in momentum space for a harmonic oscillator.
		Intended solely as a sketch, so no values for $q_k$ or color bars are provided.
	}
	\label{fig:harmonic-oscillator-survival-zero-p-aliasing}
\end{figure}

As seen in \cref{fig:harmonic-oscillator-survival-good}, we are able to generate ground state survival amplitudes for the harmonic oscillator system.
With the parameters we have chosen by the $t = 0$ analysis above, the survival amplitudes show no deviation from the expected result for several cycles for all the $\gamma$ we have used.
For reference, we may also try $q\mx = \SI{100}{\nano\meter}$, and $q\mx\alt = \SI{50}{\nano\meter}$; the results in \cref{fig:harmonic-oscillator-survival-bad} have the correct overall frequency, but the shape is incorrect for some $\gamma$.

\begin{figure}
	\centering
	\includegraphics[width=\textwidth]{13/harmonic_oscillator_sas_good}
	\caption[
		Harmonic oscillator survival amplitude with converged parameters
	]{
		Harmonic oscillator ground state survival amplitude with converged parameters.
		\explainplotsas{}
	}
	\label{fig:harmonic-oscillator-survival-good}
\end{figure}

\begin{figure}
	\centering
	\includegraphics[width=\textwidth]{13/harmonic_oscillator_sas_bad}
	\caption[
		Harmonic oscillator survival amplitude with unconverged parameters
	]{
		Harmonic oscillator ground state survival amplitude with unconverged parameters.
		\explainplotsas{}
	}
	\label{fig:harmonic-oscillator-survival-bad}
\end{figure}


\subsection{Double well}

\label{sec:semiclassical-numerical-dw}

We now move on to a more challenging, anharmonic system: a particle of mass $m$ in a symmetric double well potential with the Hamiltonian
\begin{align}
	\hat{H}
	&= \frac{\hat{p}^2}{2 m} - 2 \frac{d}{w^2} \hat{q}^2 + \frac{d}{w^4} \hat{q}^4,
\end{align}
where $d$ is the depth of the minima and $w$ is the distance from the $y$-axis to the minima.
We use $m = m_\mathrm{e}$ (mass of an electron), $d = \SI{2}{\kelvin}$, and $w = \SI{50}{\nano\meter}$.
In order to obtain the ground state wavefunction, we use a numerical PIGS matrix multiplication method with sufficiently converged $\beta$ and $\tau$.

\begin{table}[h]
	\begin{center}
	\begin{tabular}{ c c c c c c }
		\toprule
		{$\dt\alt / \si{\pico\second}$} & {$\gamma / \si{\per\square\nano\meter}$} & {$q\mx / \si{\nano\meter}$} & {$\DQ / \si{\nano\meter}$} & {$q\mx\alt / \si{\nano\meter}$} & {$\DQ\alt / \si{\nano\meter}$} \\
		\midrule
		0.5 & 0.001 & 300 & 20 & 100 & 2 \\
		\bottomrule
	\end{tabular}
	\end{center}
	\caption[
		Selected parameters for double well (numerical)
	]{
		Selected parameters for the double well model system using the numerical method.
	}
	\label{tab:model-sa0-double-well}
\end{table}

\begin{table}[h]
	\begin{center}
	\begin{tabular}{ c c c c c c }
		\toprule
		{$\dt\alt / \si{\pico\second}$} & {$\gamma / \si{\per\square\nano\meter}$} & {$q\mx / \si{\nano\meter}$} & {$\DQ / \si{\nano\meter}$} & {$q\mx\alt / \si{\nano\meter}$} & {$\DQ\alt / \si{\nano\meter}$} \\
		\midrule
		0.1 & 0.001 & 2000 & 2 & 2000 & 8 \\
		\bottomrule
	\end{tabular}
	\end{center}
	\caption[
		Improved parameters for double well (numerical)
	]{
		Improved parameters for the double well model system using the numerical method.
	}
	\label{tab:model-sas-double-well}
\end{table}

The results of the convergence studies in \cref{fig:double-well-survival-zero-q-a,fig:double-well-survival-zero-q-b} are shown in \cref{tab:model-sa0-double-well}.
If we try to use these values to generate ground state survival amplitudes, we see in \cref{fig:double-well-survival-bad} that we do a terrible job regardless of $\gamma$.
In fact, the curves end where they do because the values diverge and the calculations are aborted.

\begin{figure}
	\setlength{\figspacing}{5 mm}
	\centering
	\begin{subfigure}[b]{\textwidth}
		\includegraphics[width=\textwidth]{13/double_well_sa0_qmax}
		\caption{
			Convergence of $t = 0$ survival amplitude with $q\mx$.
		}
		\vspace{\figspacing}
	\end{subfigure}
	\begin{subfigure}[b]{\textwidth}
		\includegraphics[width=\textwidth]{13/double_well_sa0_dq}
		\caption{
			Convergence of $t = 0$ survival amplitude with $\DQ$.
		}
	\end{subfigure}
	\caption[
		Convergence of double well survival amplitude with position grids
	]{
		Convergence of $t = 0$ survival amplitude with the $\{ q\mx, \DQ \}$ grid for a double well.
		\explainplotsazero{}
	}
	\label{fig:double-well-survival-zero-q-a}
\end{figure}

\begin{figure}
	\setlength{\figspacing}{5 mm}
	\centering
	\begin{subfigure}[b]{\textwidth}
		\includegraphics[width=\textwidth]{13/double_well_sa0_qmax_alt}
		\caption{
			Convergence of $t = 0$ survival amplitude with $q\mx\alt$.
		}
		\vspace{\figspacing}
	\end{subfigure}
	\begin{subfigure}[b]{\textwidth}
		\includegraphics[width=\textwidth]{13/double_well_sa0_dq_alt}
		\caption{
			Convergence of $t = 0$ survival amplitude with $\DQ\alt$.
		}
	\end{subfigure}
	\caption[
		Convergence of double well survival amplitude with position grids \cont
	]{
		Convergence of $t = 0$ survival amplitude with the $\{ q\mx\alt, \DQ\alt \}$ grid for a double well.
		\explainplotsazero{}
	}
	\label{fig:double-well-survival-zero-q-b}
\end{figure}

\begin{figure}
	\centering
	\includegraphics[width=\textwidth]{13/double_well_sas_bad}
	\caption[
		Double well survival amplitude with poor parameters
	]{
		Double well ground state survival amplitude with poor parameters.
		\explainplotsas{}
	}
	\label{fig:double-well-survival-bad}
\end{figure}

By changing some of the parameters, we are able to improve the shape of the curves and prolong the length of the calculations, as shown in \cref{fig:double-well-survival-better} using the parameters from \cref{tab:model-sas-double-well}.
However, the results are still far from perfect.

\begin{figure}
	\centering
	\includegraphics[width=\textwidth]{13/double_well_sas_better}
	\caption[
		Double well survival amplitude with improved parameters
	]{
		Double well ground state survival amplitude with improved parameters.
		\explainplotsas{}
	}
	\label{fig:double-well-survival-better}
\end{figure}

\section{Stochastic integration}

\collab{Neil Raymond}

\begin{figure}[h]
	\centering
	\includegraphics[width=\textwidth]{13/path_explanation}
	\caption[
		Graphical notation for survival amplitude
	]{
		Details of the graphical notation used for the survival amplitude.
		Dashed box indicates the region of interest.
	}
	\label{fig:survival-path-explanation}
\end{figure}

Most interesting problems will be too large for us to find their survival amplitudes by direct integration, so we are forced to use a stochastic method.
The main issue is that there is no obvious sampling distribution for the momenta.
One cannot, after all, uniformly sample from $\intcc{-\infty, \infty}$!
In order to remedy this problem, we introduce a Gaussian weight into the integrand.

Our goal is still to obtain the value of the integrals in \vref{eq:survival-hk}, but this time without constructing grids.
Once we add the Gaussian weight, we instead obtain the expression
\begin{align}
	S_0\hkg(t)
	&= \frac{1}{2 \pi \hbar} \iiiint\! \dif p \dif q\LLup \dif q \dif q\RRup \,
			R\coh{p}{q}_t \braket{0 | q\LLup} \braket{q\RRup | 0}
			\expb{-\frac{\gamma}{2} \left( (q\LLup - q\coh{p}{q}_t)^2 + (q\RRup - q)^2 \right)} \notag \\
	&\qquad\qquad\qquad\times
			\expb{
				-\frac{p^2}{2 \sigma_p^2}
				+ \frac{i}{\hbar} \left( S\coh{p}{q}_t + p\coh{p}{q}_t (q\LLup - q\coh{p}{q}_t) - p (q\RRup - q) \right)
			}.
\end{align}
This introduces another parameter, $\sigma_p$, but that is not as bad as it might seem at first: the trade-off is that we no longer have all the grid parameters to tune.
Additionally, we will choose $\gamma$ so that we are able to sample all three position coordinates from the same PIGS simulation.

Recall that we may evaluate integrals of the form
\begin{align}
	\frac{
			\int\! \dif \vec{q} \, \pi(\vec{q}) \mathcal{S}(\vec{q})
		}{
			\int\! \dif \vec{q} \, \pi(\vec{q})
		}
\end{align}
by sampling from $\pi(\vec{q})$ and evaluating $\mathcal{S}(\vec{q})$ at the sampled coordinates.
Consider the distribution
\begin{align}
	\pi_q(q\LL, q\MM, q\RR)
	&= \braket{0 | q\LL} \expb{-\frac{m}{2 \hbar^2 \tau} \left( (q\LL - q\MM)^2 + (q\MM - q\RR)^2 \right)} \braket{q\RR | 0},
		\label{eq:hk-dist-q}
\end{align}
which we have written in a suggestive form (with $L = M - 1$ and $R = M + 1$).
What it suggests is the path drawn in \cref{fig:survival-path-explanation}, which is a complete path in the sense that its beads are all connected, but it lacks the interactions from the two links of interest.\footnote{
	One subtlety here is that we must expand each of the wavefunctions $\braket{q | 0}$ into fragments of length $(\beta - \tau) / 2$ instead of the usual $\beta / 2$, because the piece in the middle is of ``length'' $2 \tau$ and we still require the full path to be $\beta$ long.
}
Consider also the distribution
\begin{align}
	\pi_p(p)
	&= \expb{-\frac{p^2}{2 \sigma_p^2}},
\end{align}
which we combine with the previous one to make the product distribution
\begin{subequations}
\begin{align}
	\pi(p, q\LL, q\MM, q\RR)
	&= \pi_p(p) \pi_q(q\LL, q\MM, q\RR) \\
	&= \expb{-\frac{p^2}{2 \sigma_p^2}}
		\braket{0 | q\LL} \expb{-\frac{m}{2 \hbar^2 \tau} \left( (q\LL - q\MM)^2 + (q\MM - q\RR)^2 \right)} \braket{q\RR | 0}.
\end{align}
\end{subequations}
Because they are obtained from a LePIGS simulation, the wavefunctions $\braket{q | 0}$ will not be normalized.
However, due to the deformation, we are not interested in the normalization, so we may disregard this and write just
\begin{align}
	S_0\hkg(t)
	&\propto \iiiint\! \dif p \dif q\LL \dif q\MM \dif q\RR \,
			\pi(p, q\LL, q\MM, q\RR) R\coh{p}{q\MM}_t \notag \\
	&\qquad\qquad\times
			\expb{-\frac{m}{2 \hbar^2 \tau} \left( (q\LL - q\coh{p}{q\MM}_t)^2 - (q\LL - q\MM)^2 \right)} \notag \\
	&\qquad\qquad\times
			\expb{\frac{i}{\hbar} \left( S\coh{p}{q\MM}_t + p\coh{p}{q\MM}_t (q\LL - q\coh{p}{q\MM}_t) - p (q\RR - q\MM) \right)}.
\end{align}
We have explicitly chosen $\gamma = m / \hbar^2 \tau$, and as a result our estimator is
\begin{align}
	\mathcal{S}_\mathrm{P}(p, q\LL, q\MM, q\RR, t)
	&= R\coh{p}{q\MM}_t \expb{-\frac{m}{2 \hbar^2 \tau} \left( (q\LL - q\coh{p}{q\MM}_t)^2 - (q\LL - q\MM)^2 \right)} \notag \\
	&\qquad\qquad\times
		\expb{\frac{i}{\hbar} \left( S\coh{p}{q\MM}_t + p\coh{p}{q\MM}_t (q\LL - q\coh{p}{q\MM}_t) - p (q\RR - q\MM) \right)}.
\end{align}
Since this is the obvious estimator to use, we will refer to it as the \emph{primitive estimator of the survival amplitude}, with the position distribution
\begin{align}
	\pi_\mathrm{P}(q\LL, q\MM, q\RR)
	&= \symbdistwide{13/dist_primitive}.
\end{align}

We notice that even though we have preemptively removed the extraneous potentials from the distribution, we still have a spring contribution that we remove in the estimator.
If we excise it, we are left with the position distribution
\begin{align}
	\pi_\mathrm{M}(q\LL, q\MM, q\RR)
	&= \symbdistwide{13/dist_minimal}
\end{align}
and the corresponding \emph{minimal estimator of the survival amplitude}
\begin{align}
	\mathcal{S}_\mathrm{M}(p, q\LL, q\MM, q\RR, t)
	&= R\coh{p}{q\MM}_t \expb{-\frac{m}{2 \hbar^2 \tau} \left( (q\LL - q\coh{p}{q\MM}_t)^2 \right)} \notag \\
	&\qquad\times
		\expb{\frac{i}{\hbar} \left( S\coh{p}{q\MM}_t + p\coh{p}{q\MM}_t (q\LL - q\coh{p}{q\MM}_t) - p (q\RR - q\MM) \right)}.
\end{align}
We hope that the reader is not biased toward or against any estimator names due to the results of \cref{chap:renyi}.


\subsection{Harmonic oscillator}

Before we try this stochastic approach with the harmonic oscillator system from \cref{sec:semiclassical-numerical-ho}, we see (using the numerical method used previously) what happens when we deform the integral with the Gaussian weight and then explicitly renormalize the survival amplitude.
Unless otherwise specified, the parameter values from \cref{tab:model-sas-harmonic-oscillator-stochastic} are used for this model system.
The results shown in \cref{fig:harmonic-oscillator-survival-smoothing-norm} are very good over a wide range of $\sigma_p$.\footnote{
	In fact, for the narrowest momentum distribution, we end up using only a single momentum: zero.
	Yet the harmonic oscillator is fine with this.
}
Thus, we may expect that performing the same deformed integral using the stochastic method would result in a smooth survival amplitude.

\begin{figure}
	\centering
	\includegraphics[width=\textwidth]{13/harmonic_oscillator_sas_smoothing_norm}
	\caption[
		Harmonic oscillator survival amplitude with added Gaussian weight
	]{
		Harmonic oscillator ground state survival amplitude with added Gaussian weight and renormalization.
		Dashed curves indicate the exact answer.
	}
	\label{fig:harmonic-oscillator-survival-smoothing-norm}
\end{figure}

On the contrary, we find the disappointing result in \cref{fig:harmonic-oscillator-survival-primitive}.
The overall frequency appears to be correct, but the shapes of both the real and imaginary components are distorted, and the error bars are incredibly large at some points.
We may suspect that this is due to a poor choice of parameters in the LePIGS simulation.

\begin{table}
	\begin{center}
	\begin{tabular}{ c c c c c c c }
		\toprule
		{$\beta / \si{\per\kelvin}$} & {$\tau / \si{\per\kelvin}$} & {$P$} & {$\dt / \si{\pico\second}$} & $\gamma\bead{0} / \si{\per\pico\second}$ & {$\dt\alt / \si{\pico\second}$} & {$\sigma_p / \si{\gram\nano\meter\per\pico\second\per\mole}$} \\
		\midrule
		8 & 0.125 & 65 & 1 & 0.1 & 1 & 1 \\
		\bottomrule
	\end{tabular}
	\end{center}
	\caption[
		Selected parameters for harmonic oscillator (stochastic)
	]{
		Selected parameters for the harmonic oscillator model system using the stochastic method.
	}
	\label{tab:model-sas-harmonic-oscillator-stochastic}
\end{table}

\begin{figure}
	\centering
	\includegraphics[width=\textwidth]{13/harmonic_oscillator_sas_stochastic_primitive}
	\caption[
		Harmonic oscillator survival amplitude using primitive estimator
	]{
		Harmonic oscillator ground state survival amplitude with stochastic sampling using the primitive estimator.
		Average of 16 survival amplitudes from $\num{1e6}$ samples each.
		Dashed curves indicate the exact answer.
	}
	\label{fig:harmonic-oscillator-survival-primitive}
\end{figure}

However, for the harmonic oscillator system, we may sample from $\pi_q(q_L, q_M, q_R)$ in \cref{eq:hk-dist-q} directly, since it is Gaussian:
\begin{align}
	\pi_q(q\LL, q\MM, q\RR)
	&\propto \expb{-\frac{m \omega}{2 \hbar} (q\LL^2 + q\RR^2) - \frac{m}{2 \hbar^2 \tau} \left( (q\LL - q\MM)^2 + (q\MM - q\RR)^2 \right)}.
\end{align}
This allows us to bypass PIGS altogether and see what the survival amplitude should look like given a finite number of samples.
As we see in \cref{fig:harmonic-oscillator-survival-primitive-exact}, the situation was hopeless to start with.

\begin{figure}
	\centering
	\includegraphics[width=\textwidth]{13/harmonic_oscillator_sas_stochastic_primitive_exact}
	\caption[
		Harmonic oscillator survival amplitude using primitive estimator (exact sampling)
	]{
		Harmonic oscillator ground state survival amplitude with exact stochastic sampling using the primitive estimator.
		Average of 16 survival amplitudes from $\num{1e6}$ samples each.
		Dashed curves indicate the exact answer.
	}
	\label{fig:harmonic-oscillator-survival-primitive-exact}
\end{figure}

The minimal estimator fares much better, as shown in \cref{fig:harmonic-oscillator-survival-minimal}.
All the error bars are fairly small and the mean values are mostly along smooth curves.
Again, we get much better results if we break paths in the simulation and sample from a different sector.
Perhaps surprising in this particular case is the fact that $\pi_\mathrm{M}(q_L, q_M, q_R)$ is composed of two independent pieces!

\begin{figure}
	\centering
	\includegraphics[width=\textwidth]{13/harmonic_oscillator_sas_stochastic_minimal}
	\caption[
		Harmonic oscillator survival amplitude using minimal estimator
	]{
		Harmonic oscillator ground state survival amplitude with stochastic sampling using the minimal estimator.
		Average of 16 survival amplitudes from $\num{1e6}$ samples each.
		Dashed curves indicate the exact answer.
	}
	\label{fig:harmonic-oscillator-survival-minimal}
\end{figure}

We have not yet attempted to apply the stochastic approach to the double well system from \cref{sec:semiclassical-numerical-dw}.
Although we do not know of any \textit{a priori} reasons it should fail, our efforts for that system have been focused on first improving the numerical integration results.

