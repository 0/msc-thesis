\section{Estimators}

\label{sec:estimators}

Although there are infinitely many Rényi entropies $S_\alpha$, in the present work we are only interested in the \emph{second Rényi entropy}
\begin{align}
	S_2
	&= -\log{\left( \Tr{\hat{\rho}_A^2} \right)},
		\label{eq:S2}
\end{align}
and the associated quantity $\Tr{\hat{\rho}_A^2}$ to which we shall refer simply as \emph{the trace}.
This particular measure of entropy has been used for studying entanglement of lattice systems~\cite{hastings2010measuring,stephan2012renyi} and more recently in the continuum~\cite{herdman2014path,herdman2014particle}.
To demonstrate why it is appealing in practice, we will make use of the \emph{replica trick}, which involves multiple copies of the system~\cite{hastings2010measuring}.
We start with the exact ground state density operator
\begin{align}
	\hat{\rho}
	&= \ketbraself{0}
\end{align}
and perform a partial trace over the degrees of freedom in partition $B$,\footnote{
	Technically, one partitions a set into subsets, but we will be sloppy with our terminology and refer to the subsets as ``partitions'' themselves.
} leaving us with the reduced density operator for partition $A$
\begin{align}
	\hat{\rho}_A
	&= \Tr_B{\hat{\rho}}.
\end{align}
If we square this and take the trace, we obtain the trace from \cref{eq:S2}.
We may perform these operations explicitly in the position representation:
\begin{subequations} \label{eq:trace}
\begin{align}
	\rho(\vec{q}_A, \vec{q}_B ; \vec{q}_A', \vec{q}_B')
	&= \braket{\vec{q}_A \vec{q}_B | 0} \braket{0 | \vec{q}_A' \vec{q}_B'} \\
	\rho_A(\vec{q}_A; \vec{q}_A')
	&= \int\! \dif \vec{q}_B \, \braket{\vec{q}_A \vec{q}_B | 0} \braket{0 | \vec{q}_A' \vec{q}_B} \\
	\rho_A^2(\vec{q}_A; \vec{q}_A'')
	&= \iiint\! \dif \vec{q}_A' \dif \vec{q}_B \dif \vec{q}_B' \,
			\braket{\vec{q}_A \vec{q}_B | 0} \braket{0 | \vec{q}_A' \vec{q}_B}
			\braket{\vec{q}_A' \vec{q}_B' | 0} \braket{0 | \vec{q}_A'' \vec{q}_B'} \\
	\Tr{\hat{\rho}_A^2}
	&= \iiiint\! \dif \vec{q}_A \dif \vec{q}_A' \dif \vec{q}_B \dif \vec{q}_B' \,
			\braket{\vec{q}_A \vec{q}_B | 0} \braket{0 | \vec{q}_A' \vec{q}_B}
			\braket{\vec{q}_A' \vec{q}_B' | 0} \braket{0 | \vec{q}_A \vec{q}_B'} \\
	&= \iiiint\! \dif \vec{q}_A\sys{\lambda} \dif \vec{q}_B\sys{\lambda} \dif \vec{q}_A\sys{\mu} \dif \vec{q}_B\sys{\mu} \,
			\braket{0 | \vec{q}_A\sys{\red{\mu}} \vec{q}_B\sys{\lambda}} \braket{\vec{q}_A\sys{\lambda} \vec{q}_B\sys{\lambda} | 0}
			\braket{0 | \vec{q}_A\sys{\red{\lambda}} \vec{q}_B\sys{\mu}} \braket{\vec{q}_A\sys{\mu} \vec{q}_B\sys{\mu} | 0}.
				\label{eq:trace-symbols}
\end{align}
\end{subequations}
The final step is nothing more than a reordering and relabelling of the various bits.\footnote{
	We introduce extra noise into our expressions with the $\vec{q}\sys{x}$ labels, but they allow us to talk about the replicas using reasonable names instead of ``unprimed'' and ``primed.''
}
If we replace the exact ground state $\ket{0}$ by the approximate PIGS ground state $\ketbeta{0}$ and expand it into beads and links (see \cref{chap:introduction} for details), \cref{eq:trace-symbols} resembles integrals over two paths, but with some of the indices permuted.

\begin{figure}
	\centering
	\includegraphics[width=\textwidth]{12/path_explanation}
	\caption[
		Graphical notation for Rényi entropy
	]{
		Details of the graphical notation used for the Rényi entropy.
		Dashed box indicates the region of interest.
		Note that we have grouped all paths for a single partition ($A$ or $B$) into a single effective path for the diagram.
		As a consequence, the potential interactions \emph{within} a partition are not displayed (which is fine, because they are not relevant).
		Additionally, for system $\lambda$ the path for partition $B$ appears at the top.
	}
	\label{fig:renyi-path-explanation}
\end{figure}

Before we really dig into the expression for the trace, let us first consider a simpler one:
\begin{align}
	Z^2
	&= \iiiint\! \dif \vec{q}_A\sys{\lambda} \dif \vec{q}_B\sys{\lambda} \dif \vec{q}_A\sys{\mu} \dif \vec{q}_B\sys{\mu} \,
			\betabraket{0 | \vec{q}_A\sys{\lambda} \vec{q}_B\sys{\lambda}} \braketbeta{\vec{q}_A\sys{\lambda} \vec{q}_B\sys{\lambda} | 0}
			\betabraket{0 | \vec{q}_A\sys{\mu} \vec{q}_B\sys{\mu}} \braketbeta{\vec{q}_A\sys{\mu} \vec{q}_B\sys{\mu} | 0}.
\end{align}
Here we have taken paths for two independent replicas of the system ($\lambda$ and $\mu$), multiplied them together, and integrated over all the positions.
We can write this using the graphical notation as
\begin{align}
	Z^2
	&= \int\! \dif \vec{q} \, \symbdist{12/path_unpermuted}.
\end{align}
The particulars of the graphical notation in this \namecref{chap:renyi} are outlined in \cref{fig:renyi-path-explanation}.
What is important to notice here is that, as the diagram alludes, the integrals are obviously separable.
On the other hand, we may express the trace from \cref{eq:trace-symbols} as
\begin{align}
	\Tr{\rho_A^2}
	&\propto \int\! \dif \vec{q} \, \symbdist{12/path_permuted},
\end{align}
and it is clear that we cannot view this as a product of integrals.
We have been careful to claim that these quantities are not equal, but only proportional.
This is because when we substitute $\ketbeta{0}$ for $\ket{0}$ we must take into account the normalization, which amounts to $1/Z^2$, leading to the ratio
\begin{align}
	\Tr{\rho_A^2}
	&= \hugefrac{
			\int\! \dif \vec{q} \, \symbdist{12/path_permuted}
		}{
			\int\! \dif \vec{q} \, \symbdist{12/path_unpermuted}
		}.
\end{align}
As the astute reader has noticed, this is equivalent in form to \vref{eq:lepigs-integral-function}, with the distribution composed of two replicas of the system, and the estimator function
\begin{align}
	\mathcal{N}_\mathrm{P}
	&= \hugefrac{
			\symb{12/link_permuted}
		}{
			\symb{12/link_unpermuted}
		}
	= \hugefrac{
			\symb{12/link_permuted_simplified}
		}{
			\symb{12/link_unpermuted_simplified}
		}.
			\label{eq:renyi-estimator-primitive}
\end{align}
We will refer to $\mathcal{N}_\mathrm{P}$ as the \emph{primitive estimator of the trace}.
Now that the brunt of the work has been done, we may write the result down using conventional notation:
\begin{align}
	\mathcal{N}_\mathrm{P}
	&=
	\expb{\red{-}\sum_{i \in A} \frac{m_i}{2 \hbar^2 \tau} \left(
			\abs{\vec{q}_i\beadsys{M-1}{\lambda} - \vec{q}_i\beadsys{M}{\red{\mu}}}^2
			+ \abs{\vec{q}_i\beadsys{M-1}{\mu} - \vec{q}_i\beadsys{M}{\red{\lambda}}}^2
		\right)} \notag \\
	&\times
	\expb{\red{+} \sum_{i \in A} \frac{m_i}{2 \hbar^2 \tau} \left(
			\abs{\vec{q}_i\beadsys{M-1}{\lambda} - \vec{q}_i\beadsys{M}{\red{\lambda}}}^2
			+ \abs{\vec{q}_i\beadsys{M-1}{\mu} - \vec{q}_i\beadsys{M}{\red{\mu}}}^2
		\right)} \notag \\
	&\times
	\expb{\red{-}\frac{\tau}{2} \left(
			V_{AB}(\vec{q}_A\beadsys{M}{\lambda}, \vec{q}_B\beadsys{M}{\red{\mu}})
			+ V_{AB}(\vec{q}_A\beadsys{M}{\mu}, \vec{q}_B\beadsys{M}{\red{\lambda}})
		\right)} \notag \\
	&\times
	\expb{\red{+} \frac{\tau}{2} \left(
			V_{AB}(\vec{q}_A\beadsys{M}{\lambda}, \vec{q}_B\beadsys{M}{\red{\lambda}})
			+ V_{AB}(\vec{q}_A\beadsys{M}{\mu}, \vec{q}_B\beadsys{M}{\red{\mu}})
		\right)},
\end{align}
where $V_{AB}(\vec{q}_A, \vec{q}_B)$ is the sum of all potentials that act on particles in \emph{both} partitions.\footnote{
	When the system has only pairwise interactions, this is straightforward: if an interaction affects one particle in $A$ and one in $B$, it should be included.
	In the general case, where interactions may include more than two particles, the requirement may become more clear if inverted: $V_{AB}(\vec{q}_A, \vec{q}_B)$ is the sum of all potentials that are \emph{not} restricted to particles in only one partition.
}
Thus, if we can sample path configurations for the two-replica system, all we need to do is average the estimator over those configurations to obtain an estimate of the trace, from which we may obtain $S_2$.

We can generalize this approach by making the observation that
\begin{align}
	\mean{\mathcal{N}}_{\pi}
	&= \frac{\int\! \dif \vec{q} \, \pi(\vec{q}) \mathcal{N}(\vec{q})}{\int\! \dif \vec{q} \, \pi(\vec{q})}
	= \frac{\int\! \dif \vec{q} \, \pi'(\vec{q}) \mathcal{N}'(\vec{q})}{\int\! \dif \vec{q} \, \pi'(\vec{q}) \mathcal{D}'(\vec{q})}
	= \frac{\int\! \dif \vec{q} \, \pi'(\vec{q}) \mathcal{N}'(\vec{q})}{\int\! \dif \vec{q} \, \pi'(\vec{q})} \frac{\int\! \dif \vec{q} \, \pi'(\vec{q})}{\int\! \dif \vec{q} \, \pi'(\vec{q}) \mathcal{D}'(\vec{q})}
	= \frac{\mean{\mathcal{N}'}_{\pi'}}{\mean{\mathcal{D}'}_{\pi'}}
\end{align}
for a suitably chosen distribution $\pi'(\vec{q})$ and estimators $\mathcal{N}'(\vec{q})$ and $\mathcal{D}'(\vec{q})$.
Obviously, we may choose anything whatsoever for $\pi'(\vec{q})$ and offload the details onto the estimators by dividing out the unnecessary bits and multiplying in the required ones, but most arbitrary choices of distribution would not be very sensible.
One natural approach is to choose a distribution which requires nothing to be divided out, and only the minimal amount to be multiplied in.
In our case, this results in the distribution
\begin{align}
	\symbdist{12/dist_minimal}
\end{align}
and the estimators
\begin{align}
	\mathcal{N}_\mathrm{M}
	&= \symb{12/link_permuted_simplified}
	&
	\mathcal{D}_\mathrm{M}
	&= \symb{12/link_unpermuted_simplified}.
			\label{eq:renyi-estimator-minimal}
\end{align}
We will refer to the ratio estimator $\mean{\mathcal{N}_\mathrm{M}} / \mean{\mathcal{D}_\mathrm{M}}$ as the \emph{minimal estimator of the trace}.
Clearly,
\begin{align}
	\mathcal{N}_\mathrm{P}
	&= \frac{\mathcal{N}_\mathrm{M}}{\mathcal{D}_\mathrm{M}},
\end{align}
but
\begin{align}
	\mean{\mathcal{N}_\mathrm{P}}
	&= \bigmean{\frac{\mathcal{N}_\mathrm{M}}{\mathcal{D}_\mathrm{M}}}
	\ne \frac{\mean{\mathcal{N}_\mathrm{M}}}{\mean{\mathcal{D}_\mathrm{M}}},
\end{align}
so these provide distinct estimates of the trace.
