\chapter{Solutions to exercises}

\label{chap:solutions}

\epigraph{
All you need is not to be mistaken.
}{
WBaduk beginner problem BP00107
}

\begin{DefAnswer}{ex:pigs-limit}
	Since
	\begin{subequations}
	\begin{align}
		e^{-\beta \hat{H}} \ket{\psiT}
		&= \sum_{n=0}^\infty e^{-\beta E_n} \ket{n} \braket{n | \psiT} \\
		&= e^{-\beta E_0} \left[ \ket{0} \braket{0 | \psiT} + \sum_{n=1}^\infty e^{-\beta (E_n - E_0)} \ket{n} \braket{n | \psiT} \right],
	\end{align}
	\end{subequations}
	all the terms in the sum after the $n = 0$ term will be exponentially suppressed with increasing $\beta$.
	As long as the factor $\braket{0 | \psiT}$ does not vanish, the ground state will be projected out of the trial function:
	\begin{align}
		\lim_{\beta \to \infty} e^{-\beta \hat{H}} \ket{\psiT}
		&\propto \ket{0}.
	\end{align}
\end{DefAnswer}

\begin{DefAnswer}{ex:pigs-isomorphism}
	We assume a Cartesian, $N$-particle Hamiltonian of the form
	\begin{align}
		\hat{H}
		&= \hat{K} + \hat{V}
		= \left[ \sum_{n=0}^{N-1} \hat{K}_n \right] + \hat{V}
		= \left[ \sum_{n=0}^{N-1} \frac{\abs{\hat{\vec{p}}_n}^2}{2 m_n} \right] + \hat{V},
	\end{align}
	where the potential operator is sufficiently well-behaved that we may write
	\begin{align}
		\hat{H}
		&= \left[ \sum_{n=0}^{N-1} \frac{\abs{\hat{\vec{p}}_n}^2}{2 m_n} \right]
			+ V(\hat{\vec{q}}_0, \hat{\vec{q}}_1, \ldots, \hat{\vec{q}}_{N-1}).
	\end{align}
	We begin with the approximate pseudo partition function
	\begin{align}
		Z_\beta
		&= \Tr{\hat{\rho}_\beta}
		= \braket{\psiT | e^{-\beta \hat{H}} | \psiT}.
	\end{align}
	Recalling that the resolution of the identity in any continuous basis is
	\begin{align}
		\hat{1}
		&= \int\! \dif \vec{x} \, \ketbraself{\vec{x}},
	\end{align}
	we can introduce $\tau$, $P$, and $\links$ such that $\beta = (P-1) \tau = \links \tau$\footnote{
		As per the existing convention, $P$ is the number of \emph{beads}.
		At finite temperature, paths would be closed loops thanks to the additional trace operation, so the number of beads would be equal to the number of \emph{links}.
		In the ground state, paths turn out to be open chains capped by a trial function, so there must be one fewer link than there are beads.
		As a consequence, the value $P - 1$ crops up frequently, but saying $P - 1$ everywhere is tedious, so instead we use $\links$.
	} and (without making any approximations) expand the Boltzmann operator as follows:
	\begin{subequations}
	\begin{align}
		Z_\beta
		&= \braket{\psiT | e^{-\beta \hat{H}} | \psiT} \\
		&= \braket{\psiT | \left[ \prod_{j=0}^{\links - 1} e^{-\tau \hat{H}} \right] | \psiT} \\
		&= \bra{\psiT}
			\left[
				\prod_{j=0}^{\links - 1}
					\left( \int\! \dif \vec{q}\bead{j} \ketbraself{\vec{q}\bead{j}} \right)
					e^{-\tau \hat{H}}
			\right]
			\left( \int\! \dif \vec{q}\bead{P-1} \ketbraself{\vec{q}\bead{P-1}} \right)
			\ket{\psiT} \\
		&= \idotsint\! \left[ \prod_{j=0}^{P-1} \dif \vec{q}\bead{j} \right]
			\braket{\psiT | \vec{q}\bead{0}}
			\left[ \prod_{j=0}^{\links - 1}
				\braket{\vec{q}\bead{j} | e^{-\tau \hat{H}} | \vec{q}\bead{j+1}}
			\right]
			\braket{\vec{q}\bead{P-1} | \psiT} \\
		&= \int\! \dif \vec{q} \,
			\braket{\psiT | \vec{q}\bead{0}}
			\braket{\vec{q}\bead{P-1} | \psiT}
			\prod_{j=0}^{\links - 1}
				\braket{\vec{q}\bead{j} | e^{-\tau \hat{H}} | \vec{q}\bead{j+1}}.
	\end{align}
	\end{subequations}
	The integrand resembles a path of $P$ positions (referred to as ``beads'') connected by $\links$ ``links'' of a nature that is still to be determined.

	We require (without loss of generality) that $\braket{\vec{q} | \psiT} = \psiT(\vec{q})$ be a nodeless, positive, real function.
	This is a perfectly reasonable demand to make, since we expect our ground state wavefunction to also be a nodeless, positive, real function.
	The fact that it's real implies that
	\begin{align}
		\braket{\vec{q} | \psiT} = \braket{\psiT | \vec{q}}.
	\end{align}
	That it's also nodeless and positive means that
	\begin{align}
		\psiT(\vec{q}) = e^{\ln{\psiT(\vec{q})}},
	\end{align}
	so following ref.~\cite{schmidt2014inclusion} we can define a fictitious potential
	\begin{align}
		\VT(\vec{q}) = -\frac{1}{\tau} \ln{\psiT(\vec{q})}
	\end{align}
	arising due to the presence of the trial function.
	Thus, we obtain the expression
	\begin{align}
		Z_\beta
		&= \int\! \dif \vec{q} \,
			e^{-\tau \left[ \VT(\vec{q}\bead{0}) + \VT(\vec{q}\bead{P-1}) \right]}
			\prod_{j=0}^{\links - 1}
				\braket{\vec{q}\bead{j} | e^{-\tau \hat{H}} | \vec{q}\bead{j+1}}.
	\end{align}

	We would like to act with the high-temperature Boltzmann operator $e^{-\tau \hat{H}}$ on the position ket, but to do so would require us to diagonalize the Hamiltonian.
	If we could do that, this entire work would be moot!
	We could try to make use of the fact that $\hat{H} = \hat{K} + \hat{V}$, but since $\hat{K}$ and $\hat{V}$ in general don't commute, we have the inconvenient inequality
	\begin{align}
		e^{-\tau \hat{H}}
		= e^{-\tau \hat{K} - \tau \hat{V}}
		&\ne e^{-\tau \hat{K}} e^{-\tau \hat{V}}.
	\end{align}
	However, if we are willing to live with an additional systematic error, we can apply the symmetric Trotter factorization~\cite{schmidt1995high}
	\begin{align}
		e^{-\tau \hat{H}}
		&= e^{-\frac{\tau}{2} \hat{V}} e^{-\tau \hat{K}} e^{-\frac{\tau}{2} \hat{V}} + \bigo{\tau^3}
	\end{align}
	to each factor in the product, to get
	\begin{subequations}
	\begin{align}
		\braket{\vec{q}\bead{j} | e^{-\tau \hat{H}} | \vec{q}\bead{j+1}}
		&\approx \braket{
			\vec{q}\bead{j} |
			e^{-\frac{\tau}{2} \hat{V}} e^{-\tau \hat{K}} e^{-\frac{\tau}{2} \hat{V}} |
			\vec{q}\bead{j+1}
		} \\
		&= e^{-\frac{\tau}{2} \left[ V(\vec{q}\bead{j}) + V(\vec{q}\bead{j+1}) \right]}
			\braket{
				\vec{q}\bead{j} |
				\expb{-\frac{\tau}{2} \sum_{n=0}^{N-1} \frac{\abs{\hat{\vec{p}}_n}^2}{m_n}} |
				\vec{q}\bead{j+1}
			}.
	\end{align}
	\end{subequations}
	Since the potential operator is diagonal in the position representation, we were able to simply act with the potential part of the exponential.
	The kinetic part is not as obvious, but we can still treat it exactly.
	Using \cref{eq:pq-inner,eq:gaussian-integral-amu} on \cpageref{eq:pq-inner,eq:gaussian-integral-amu} we find that the matrix elements are
	\begin{subequations}
	\begin{align}
		& \braket{
			\vec{q}\bead{j} |
			\expb{-\frac{\tau}{2} \sum_{n=0}^{N-1} \frac{\abs{\hat{\vec{p}}_n}^2}{m_n}} |
			\vec{q}\bead{j+1}
		} \notag \\
		&= \int\! \dif \vec{p}
				\braket{
					\vec{q}\bead{j} |
					\expb{-\frac{\tau}{2} \sum_{n=0}^{N-1} \frac{\abs{\hat{\vec{p}}_n}^2}{m_n}} |
					\vec{p}
				}
				\braket{\vec{p} | \vec{q}\bead{j+1}} \\
		&= \int\! \dif \vec{p}
				\braket{\vec{q}\bead{j} | \vec{p}} \braket{\vec{p} | \vec{q}\bead{j+1}}
				\expb{-\frac{\tau}{2} \sum_{n=0}^{N-1} \frac{\abs{\vec{p}_n}^2}{m_n}} \\
		&= \left[ \frac{1}{2 \pi \hbar} \right]^{N D}
			\int\! \dif \vec{p} \, \expb{
					-\frac{\tau}{2} \left[ \sum_{n=0}^{N-1} \frac{\abs{\vec{p}_n}^2}{m_n} \right]
					+ \frac{i}{\hbar} \left( \vec{q}\bead{j} - \vec{q}\bead{j+1} \right) \cdot \vec{p}
				} \\
		&= \left[ \frac{1}{2 \pi \hbar} \right]^{N D}
			\prod_{n=0}^{N-1} \int\! \dif \vec{p}_n \, \expb{
					-\frac{\tau}{2} \frac{\abs{\vec{p}_n}^2}{m_n}
					+ \frac{i}{\hbar} \left( \vec{q}_n\bead{j} - \vec{q}_n\bead{j+1} \right) \cdot \vec{p}_n
				} \\
		&= \left[ \frac{1}{2 \pi \hbar} \right]^{N D}
			\prod_{n=0}^{N-1} \prod_{d=0}^{D-1} \int\! \dif p_{n,d} \, \expb{
					-\frac{\tau}{2} \frac{p_{n,d}^2}{m_n}
					+ \frac{i}{\hbar} \left( q_{n,d}\bead{j} - q_{n,d}\bead{j+1} \right) p_{n,d}
				} \\
		&= \left[ \frac{1}{2 \pi \hbar} \right]^{N D}
			\prod_{n=0}^{N-1} \prod_{d=0}^{D-1}
				\left[ \frac{2 \pi m_n}{\tau} \right]^\frac{1}{2}
				\expb{
					-\frac{m_n}{2 \hbar^2 \tau} \left( q_{n,d}\bead{j} - q_{n,d}\bead{j+1} \right)^2
				} \\
		&= \prod_{n=0}^{N-1}
				\left[ \frac{m_n}{2 \pi \hbar^2 \tau} \right]^\frac{D}{2}
				\prod_{d=0}^{D-1}
					\expb{
						-\frac{m_n}{2 \hbar^2 \tau} \left( q_{n,d}\bead{j} - q_{n,d}\bead{j+1} \right)^2
					} \\
		&= \prod_{n=0}^{N-1}
				\left[ \frac{m_n}{2 \pi \hbar^2 \tau} \right]^\frac{D}{2}
				\expb{
					-\frac{m_n}{2 \hbar^2 \tau} \abs{\vec{q}_n\bead{j} - \vec{q}_n\bead{j+1}}^2
				}.
	\end{align}
	\end{subequations}

	Before proceeding, we define\footnote{
	Our $\omega_\links$ is identical in spirit to the $\omega_n$ of ref.~\cite{ceriotti2010efficient} and $\omega_{P-1}$ of ref.~\cite{constable2013langevin}.
	}
	\begin{align}
		\omega_\links
		&= \frac{\links}{\hbar \beta}
		= \frac{1}{\hbar \tau}
			\label{eq:omega-links}
	\end{align}
	and
	\begin{align}
		V_\links(\vec{q})
		&= \frac{1}{\links} \left[
				\frac{V(\vec{q}\bead{0})}{2}
				+ \VT(\vec{q}\bead{0})
				+ \left[ \sum_{j=1}^{P-2} V(\vec{q}\bead{j}) \right]
				+ \frac{V(\vec{q}\bead{P-1})}{2}
				+ \VT(\vec{q}\bead{P-1})
			\right].
	\end{align}
	Now we can substitute everything back into the expression for the pseudo partition function (and use ``$=$'' rather than ``$\approx$'', boldly disregarding the systematic error due to the Trotter factorization):
	\begin{subequations}
	\begin{align}
		Z_\beta
		&= \int\! \dif \vec{q} \,
			e^{-\tau \left[ \VT(\vec{q}\bead{0}) + \VT(\vec{q}\bead{P-1}) \right]} \notag \\
		&\qquad\qquad\times
			\prod_{j=0}^{\links - 1}
				e^{-\frac{\tau}{2} \left[ V(\vec{q}\bead{j}) + V(\vec{q}\bead{j+1}) \right]}
				\prod_{n=0}^{N-1}
					\left[ \frac{m_n}{2 \pi \hbar^2 \tau} \right]^\frac{D}{2}
					\expb{
						-\frac{m_n}{2 \hbar^2 \tau} \abs{\vec{q}_n\bead{j} - \vec{q}_n\bead{j+1}}^2
					} \\
		&= \left[ \prod_{n=0}^{N-1} \left[ \frac{m_n}{2 \pi \hbar^2 \tau} \right]^\frac{\links D}{2} \right]
			\int\! \dif \vec{q} \,
				\expb{
					-\tau \left[ \VT(\vec{q}\bead{0}) + \VT(\vec{q}\bead{P-1}) \right]
					- \frac{\tau}{2} \sum_{j=0}^{\links - 1} \left[ V(\vec{q}\bead{j}) + V(\vec{q}\bead{j+1}) \right]
				} \notag \\
		&\qquad\qquad\qquad\qquad\qquad\qquad\times
				\expb{
					-\sum_{n=0}^{N-1} \sum_{j=0}^{\links - 1}
						\frac{m_n}{2 \hbar^2 \tau} \abs{\vec{q}_n\bead{j} - \vec{q}_n\bead{j+1}}^2
				} \\
		&= \left[ \prod_{n=0}^{N-1} \left[ \frac{m_n}{2 \pi \hbar^2 \tau} \right]^\frac{\links D}{2} \right]
			\int\! \dif \vec{q} \,
				\expb{
					-\beta V_\links(\vec{q})
					- \beta \sum_{n=0}^{N-1} \sum_{j=0}^{\links - 1}
						\frac{1}{2} \frac{m_n}{\links} \omega_\links^2 \abs{\vec{q}_n\bead{j} - \vec{q}_n\bead{j+1}}^2
				}.
	\end{align}
\end{subequations}
	Since we are only interested in average properties, the constant factor is irrelevant (it will disappear when we normalize), so we only need to claim that
	\begin{align}
		Z_\beta
		&\propto
			\int\! \dif \vec{q} \,
				\expb{
					-\beta V_\links(\vec{q})
					- \beta \sum_{n=0}^{N-1} \sum_{j=0}^{\links - 1}
						\frac{1}{2} \frac{m_n}{\links} \omega_\links^2 \abs{\vec{q}_n\bead{j} - \vec{q}_n\bead{j+1}}^2
				}.
	\end{align}
	The only approximation we have made so far to $Z_\beta$ is the Trotter factorization, and we now have what looks like a classical potential composed of: the real system potential; the potential from the trial function; the ``kinetic'' potential in the form of harmonic springs with force constants $m_n \omega_\links^2 / \links$.
	Note that the springs connect the beads of a single particle in sequence, forming a structure similar to an open-chain polymer.

	Our goal is to relate the quantum system to a classical one, so we need to make $Z_\beta$ look like a classical partition function:
	\begin{align}
		Z
		&\propto \iint\! \dif \vec{p} \dif \vec{q} \, e^{-\beta H(\vec{p}, \vec{q})}.
	\end{align}
	To this end, we introduce an integral over fictitious momenta $\fict{\vec{p}}$:
	\begin{align}
		\int\! \dif \fict{\vec{p}} \, \expb{-\beta \sum_{n=0}^{N-1} \frac{\abs{\fict{\vec{p}}_n}^2}{2 \fict{m}_n}}.
	\end{align}
	The value of this integral depends on the fictitious masses $\fict{m}_n$, but since we are only interested in proportionality, these masses are arbitrary.
	If we define
	\begin{subequations} \label{eq:classical-equations}
	\begin{align}
		\Vspring(\vec{q})
		&= \sum_{n=0}^{N-1} \sum_{j=0}^{\links - 1}
			\frac{1}{2} \frac{m_n}{\links} \omega_\links^2
			\abs{\vec{q}_n\bead{j} - \vec{q}_n\bead{j+1}}^2
				\label{eq:classical-potential-spring} \\
		\Veff(\vec{q})
		&= V_\links(\vec{q}) + \Vspring(\vec{q})
				\label{eq:classical-potential} \\
		\Hcl(\fict{\vec{p}}, \vec{q})
		&= \left[ \sum_{n=0}^{N-1} \frac{\abs{\fict{\vec{p}}_n}^2}{2 \fict{m}_n} \right] + \Veff(\vec{q}),
				\label{eq:classical-hamiltonian}
	\end{align}
	\end{subequations}
	this leads us directly to
	\begin{align}
		Z_\beta
		&\propto \iint\! \dif \fict{\vec{p}} \dif \vec{q} \, e^{-\beta \Hcl(\fict{\vec{p}}, \vec{q})}.
			\label{eq:classical-Z}
	\end{align}
	Thus, we have obtained a classical system of open-chain polymers at reciprocal temperature $\beta$ which is isomorphic to the original quantum system in the sense that its partition function is an approximation to the quantum pseudo partition function.

	This is useful, because it allows us to indirectly determine average properties for the quantum system.
	For example, consider some property
	\begin{align}
		\hat{A} \ket{\vec{q}}
		&= A(\vec{q}) \ket{\vec{q}}
	\end{align}
	that is diagonal in the position representation.
	The quantum expectation of this property with respect to the state $\hat{\rho}_\beta$ is
	\begin{align}
		\mean{\hat{A}}_{\hat{\rho}_\beta}
		&= \frac{\Tr{\hat{\rho}_\beta \hat{A}}}{\Tr{\hat{\rho}_\beta}}.
	\end{align}
	As demonstrated above, the denominator is $Z_\beta$, and if we repeat the above analysis with the operator $\hat{A}$ present in the expansion, we will see that the numerator is proportional to
	\begin{align}
		\iint\! \dif \fict{\vec{p}} \dif \vec{q} \, e^{-\beta \Hcl(\fict{\vec{p}}, \vec{q})} A(\vec{q}\bead{M}),
	\end{align}
	where $M = \links / 2$ is the index of the middle bead.
	Hence, for operators that are diagonal in the position representation, the quantum average is the same as the classical phase space average for the appropriate system:
	\begin{align}
		\mean{\hat{A}}_{\hat{\rho}_\beta}
		&= \mean{A}_{Z_\beta}.
	\end{align}
	As shown elsewhere in this work, this idea can be extended to off-diagonal operators as well, making this method a rather powerful one.
\end{DefAnswer}

\begin{DefAnswer}{ex:pigs-normal-mode}
	Note that we can write the spring potential from \cref{eq:classical-potential-spring} in quadratic form as
	\begin{subequations}
	\begin{align}
		\Vspring(\vec{q})
		&= \sum_{n=0}^{N-1} \sum_{j=0}^{\links - 1}
			\frac{1}{2} \fict{m}_n \omega_\links^2 \abs{\vec{q}_n\bead{j} - \vec{q}_n\bead{j+1}}^2 \\
		&= \sum_{n=0}^{N-1} \sum_{j=0}^{\links - 1} \sum_{d=0}^{D-1}
			\frac{1}{2} \fict{m}_n \omega_\links^2 \left( q_{n,d}\bead{j} - q_{n,d}\bead{j+1} \right)^2 \\
		&= \omega_\links^2 \sum_{n=0}^{N-1} \fict{m}_n \sum_{d=0}^{D-1} \left(
				\sum_{j=0}^{\links - 1}
					\frac{1}{2} \left( (q_{n,d}\bead{j})^2 - 2 q_{n,d}\bead{j} q_{n,d}\bead{j+1} + (q_{n,d}\bead{j+1})^2 \right)
			\right) \\
		&= \omega_\links^2 \sum_{n=0}^{N-1} \fict{m}_n \sum_{d=0}^{D-1} \left(
				\sum_{j=0}^{\links - 1} \frac{1}{2} \left( (q_{n,d}\bead{j})^2 + (q_{n,d}\bead{j+1})^2 \right)
			\right) - \left(
				\sum_{j=0}^{\links - 1} q_{n,d}\bead{j} q_{n,d}\bead{j+1}
			\right) \\
		&= \omega_\links^2 \sum_{n=0}^{N-1} \fict{m}_n \sum_{d=0}^{D-1} \frac{1}{2} \vec{q}_{n,d}\trans \mat{A} \vec{q}_{n,d},
			\label{eq:pigs-spring-cartesian}
	\end{align}
	\end{subequations}
	where the elements of $\mat{A}$ are
	\begin{align}
		a_{i,j}
		&= \begin{cases}
				1 & i = j \in \{ 0, P-1 \} \\
				2 & i = j \not\in \{ 0, P-1 \} \\
				-1 & \abs{i - j} = 1 \\
				0 & \text{otherwise}
			\end{cases},
				\label{eq:nm-matrix}
	\end{align}
	and $\mat{A}$ itself is a $(P \times P)$ tridiagonal matrix that looks like
	\begin{align}
		\mat{A}
		&= \begin{pmatrix}
				\phantom{-}1 & -1 & \phantom{-}0 & \cdots & \phantom{-}0 & \phantom{-}0 \\
				-1 & \phantom{-}2 & -1 & \cdots & \phantom{-}0 & \phantom{-}0 \\
				\phantom{-}0 & -1 & \phantom{-}2 & \cdots & \phantom{-}0 & \phantom{-}0 \\
				\phantom{-}\vdots & \phantom{-}\vdots & \phantom{-}\vdots & \ddots & \phantom{-}\vdots & \phantom{-}\vdots \\
				\phantom{-}0 & \phantom{-}0 & \phantom{-}0 & \cdots & \phantom{-}2 & -1 \\
				\phantom{-}0 & \phantom{-}0 & \phantom{-}0 & \cdots & -1 & \phantom{-}1 \\
			\end{pmatrix}.
				\label{eq:nm-matrix-full}
	\end{align}
	If we diagonalize $\mat{A}$ as
	\begin{align}
		\mat{A}
		= \mat{S}\trans \mat{\Lambda} \mat{S}
		&\iff
		\mat{\Lambda}
		= \mat{S} \mat{A} \mat{S}\trans,
			\label{eq:nm-diagonalization}
	\end{align}
	where $\mat{\Lambda}$ is diagonal and $\mat{S}$ is orthogonal, then we'll get a new set of coordinates $\nm{\vec{q}}$ corresponding to normal modes with frequencies that are the diagonal elements of $\mat{\Lambda}$:
	\begin{subequations}
	\begin{align}
		\vec{q}_{n,d}\trans \mat{A} \vec{q}_{n,d}
		&= \vec{q}_{n,d}\trans (\mat{S}\trans \mat{\Lambda} \mat{S}) \vec{q}_{n,d} \\
		&= (\vec{q}_{n,d}\trans \mat{S}\trans) \mat{\Lambda} (\mat{S} \vec{q}_{n,d}) \\
		&= (\mat{S} \vec{q}_{n,d})\trans \mat{\Lambda} (\mat{S} \vec{q}_{n,d}) \\
		&= \nm{\vec{q}}_{n,d}\trans \mat{\Lambda} \nm{\vec{q}}_{n,d}.
	\end{align}
	\end{subequations}

	The existing references for LePIGS (refs.~\cite[63-66]{constable2012path} and \cite{constable2013langevin}) state how to perform the transformation, but do not describe its derivation.
	Since the derivation is fairly straightforward, it is presented here.
	The characteristic equation of $\mat{A}$ is
	\begin{align}
		\det{\left( \mat{A} - \lambda \mat{1} \right)}
		&= 0.
	\end{align}
	Since the matrix in question is tridiagonal, we can use the method in ref.~\cite{el2004inverse} to find the determinant.
	We introduce the recurrence relation
	\begin{subequations} \label{eq:pigs-recurrence}
	\begin{align}
		f_0
		&= 1 \\
		f_1
		&= 1 - \lambda \\
		f_n
		&= (2 - \lambda) f_{n-1} - f_{n-2},
	\end{align}
	\end{subequations}
	where
	\begin{align}
		\det{\left( \mat{A} - \lambda \mat{1} \right)}
		&= (1 - \lambda) f_{P-1} - f_{P-2}.
	\end{align}
	By \cref{eq:recurrence-relation}, this recurrence relation has the general solution
	\begin{align}
		f_n
		&= \begin{cases}
				1 & \text{if } \lambda = 0 \\
				(1 + 2 n) (-1)^n & \text{if } \lambda = 4 \\
				\frac{1}{2^{n+1}}
					\left[ \left( 1 - \frac{\lambda}{\sqrt{\lambda^2 - 4 \lambda}} \right)
								\left( 2 - \lambda + \sqrt{\lambda^2 - 4 \lambda} \right)^n
						\right.
					& \\
				\qquad\qquad
						\left.
							+ \left( 1 + \frac{\lambda}{\sqrt{\lambda^2 - 4 \lambda}} \right)
								\left( 2 - \lambda - \sqrt{\lambda^2 - 4 \lambda} \right)^n
						\right]
					& \text{otherwise}
			\end{cases}.
	\end{align}
	To find the eigenvalues $\lambda$ of $\mat{A}$, we need to solve the characteristic equation given above, which can be phrased as
	\begin{align}
		(1 - \lambda) f_{P-1}
		&= f_{P-2}.
	\end{align}
	Clearly, $\lambda = 0$ is always an eigenvalue of $\mat{A}$, while $\lambda = 4$ never is.
	For the others, we need to do some work.
	We'll need the fact that
	\begin{align}
		\left( 2 - \lambda \pm \sqrt{\lambda^2 - 4 \lambda} \right)^2
		&= 2 \left[ \lambda^2 - \left( 4 \pm \sqrt{\lambda^2 - 4 \lambda} \right) \lambda + 2 \left( 1 \pm \sqrt{\lambda^2 - 4 \lambda} \right) \right].
	\end{align}
	Some amount of tedious (but trivial) algebra later, we arrive at
	\begin{align}
		\left( 2 - \lambda + \sqrt{\lambda^2 - 4 \lambda} \right)^P
		&= \left( 2 - \lambda - \sqrt{\lambda^2 - 4 \lambda} \right)^P.
			\label{eq:pigs-eigenvalues-powers}
	\end{align}
	If the square root is real ($\lambda < 0$ or $\lambda > 4$), we get nothing interesting.
	On the other hand, if $0 < \lambda < 4$, then
	\begin{align}
		\sqrt{\lambda^2 - 4 \lambda}
		&= i \sqrt{4 \lambda - \lambda^2},
	\end{align}
	so we can write \cref{eq:pigs-eigenvalues-powers} in unit polar form (the modulus being $2^P$ on both sides) as
	\begin{align}
		e^{i P \theta}
		&= e^{-i P \theta},
			\label{eq:pigs-eigenvalues-polar}
	\end{align}
	where $\theta$ is the complex argument of $2 - \lambda + i \sqrt{4 \lambda - \lambda^2}$.
	By a simple geometrical argument in the complex plane, we must require that $P \theta$ be an integer multiple $\pi$, so $\theta$ is an integer multiple of $\pi / P$.
	The solutions $e^{i \theta}$ of \cref{eq:pigs-eigenvalues-polar} are therefore the $(2 P)$th roots of unity (for $k = 0, 1, \ldots, 2 P - 1$)
	\begin{align}
		e^{i \theta}
		&= \expb{\frac{\pi i k}{P}}.
	\end{align}
	In order to satisfy (from the Cartesian form)
	\begin{subequations}
	\begin{align}
		\cos{\left[ \frac{\pi k}{P} \right]}
		&= 1 - \frac{\lambda}{2} \\
		\sin{\left[ \frac{\pi k}{P} \right]}
		&= \frac{1}{2} \sqrt{4 \lambda - \lambda^2},
	\end{align}
	\end{subequations}
	we must restrict $\lambda$ to
	\begin{align}
		2 - 2 \cos{\left[ \frac{\pi k}{P} \right]}
		&= 4 \sin^2{\left[ \frac{\pi k}{2 P} \right]}.
	\end{align}
	Since
	\begin{align}
		\sin^2{\left[ \frac{\pi (2 P - k)}{2 P} \right]}
		&= \sin^2{\left[ \pi - \frac{\pi k}{2 P} \right]}
		= \sin^2{\left[ \frac{\pi k}{2 P} \right]},
	\end{align}
	we only get distinct eigenvalues for $k$ up to $P$.
	Additionally, for $k = P$, we find the non-eigenvalue $\lambda = 4$, so we must exclude this possibility.
	Thus, the eigenvalues of $\mat{A}$ are (for $k = 0, 1, \ldots, P - 1$)
	\begin{align}
		\lambda_k
		&= 4 \sin^2{\left[ \frac{\pi k}{2 P} \right]}.
	\end{align}
	This covers only the first quarter of the oscillation of a sine, so the eigenvalues are all between 0 and 4 (and monotonically increasing with $k$).

	Now that we've found the eigenvalues, we need to find the transformation matrix $\mat{S}$.
	From the way we defined $\mat{S}$ and $\mat{\Lambda}$ in \cref{eq:nm-diagonalization}, we have
	\begin{align}
		\mat{A} \vec{s}_k
		&= \lambda_k \vec{s}_k,
	\end{align}
	so $\vec{s}_k$ (the rows of $\mat{S}$) are the eigenvectors of $\mat{A}$.
	If we choose an eigenvalue $\lambda$ and the corresponding eigenvector $\vec{v}$, we can write this as
	\begin{align}
		(\mat{A} - \lambda \mat{1}) \vec{v}
		&= \vec{0},
	\end{align}
	which in explicit form (for $i = 0, 1, \ldots, P - 1$) is
	\begin{align}
		\sigma_i
		&= \sum_{j=0}^{P-1} (a_{i,j} - \lambda \delta_{i,j}) v_j
		= 0.
	\end{align}
	Thanks to the tridiagonal structure of $\mat{A}$, none of these sums have more than three terms:
	\begin{align}
		\sigma_i
		&= \begin{cases}
				(1 - \lambda) v_0 - v_1 & \text{if } i = 0 \\
				(2 - \lambda) v_i - v_{i-1} - v_{i+1} & \text{if } 1 \le i \le P - 2 \\
				(1 - \lambda) v_{P-1} - v_{P-2} & \text{if } i = P - 1
			\end{cases}.
	\end{align}
	Since we can renormalize the eigenvectors later, we arbitrarily choose $v_0 = 1$, to find
	\begin{align}
		v_i
		&= \begin{cases}
				1 & \text{if } i = 0 \\
				1 - \lambda & \text{if } i = 1 \\
				(2 - \lambda) v_{i-1} - v_{i-2} & \text{if } 2 \le i \le P - 1 \\
			\end{cases}
	\end{align}
	with the extra constraint that $(1 - \lambda) v_{P-1} = v_{P-2}$.
	This is exactly the same as the recurrence relation in \cref{eq:pigs-recurrence}, so we can use the terms of that relation directly as the elements of the eigenvectors!
	We know that
	\begin{subequations}
	\begin{align}
		1 \mp \frac{\lambda_k}{\sqrt{\lambda_k^2 - 4 \lambda_k}}
		&= 1 \mp \frac{\sin{\left[ \frac{\pi k}{2 P} \right]}}{\sqrt{-\cos^2{\left[ \frac{\pi k}{2 P} \right]}}}
		= 1 \pm i \tan{\left[ \frac{\pi k}{2 P} \right]}
		= \frac{e^{\pm \frac{\pi i k}{2 P}}}{\cos{\left[ \frac{\pi k}{2 P} \right]}} \\
		\frac{1}{2^n} \left( 2 - \lambda_k \pm \sqrt{\lambda_k^2 - 4 \lambda_k} \right)^n
		&= e^{\frac{\pm \pi i k n}{P}},
	\end{align}
	\end{subequations}
	so
	\begin{align}
		s_{k,n}
		&\propto \begin{cases}
				1 & \text{if } k = 0 \\
				\frac{1}{2} \left[
					\frac{e^{\frac{\pi i k}{2 P}}}{\cos{\left[ \frac{\pi k}{2 P} \right]}} e^{\frac{\pi i k n}{P}}
						+ \frac{e^{-\frac{\pi i k}{2 P}}}{\cos{\left[ \frac{\pi k}{2 P} \right]}} e^{\frac{-\pi i k n}{P}}
				\right] & \text{otherwise}
			\end{cases}.
	\end{align}
	We are justified in dropping the cosines in the denominators, since they are constant \emph{within a row} (\ie{} eigenvector) and we will renormalize those explicitly.
	If we introduce the normalization constants $C_k$, we may simplify the above to just
	\begin{align}
		s_{k,n}
		&= C_k \cos{\left[ \frac{\pi}{P} k \left( n + \frac{1}{2} \right) \right]}.
			\label{eq:pigs-nm-transform}
	\end{align}
	For the normalization, we demand that
	\begin{align}
		\sum_{n=0}^{P-1} s_{k,n}^2
		&= C_k^2 \sum_{n=0}^{P-1} \cos^2{\left[ \frac{\pi}{P} k \left( n + \frac{1}{2} \right) \right]}
		= 1
	\end{align}
	By \vref{eq:cosine-sum}, this becomes
	\begin{align}
		C_k
		&= \begin{cases}
				\sqrt{\frac{1}{P}} & \text{if } k = 0 \\
				\sqrt{\frac{2}{P}} & \text{otherwise}
			\end{cases}.
	\end{align}

	Thus, we have rederived the open path normal mode transformation\footnote{
		The reader may be interested in comparing this to the closed path normal mode transformation derived in \vref{sec:pile-odd}.
	} and found it to be the same Discrete Cosine Transform as in ref.~\cite{constable2012path}.\footnote{
		Some technical details relating to the DCT are discussed in \vref{chap:dct}.
	}
	The spring potential from \cref{eq:pigs-spring-cartesian} may be written in the new coordinates as
	\begin{subequations}
	\begin{align}
		\Vspring(\nm{\vec{q}})
		&= \omega_\links^2 \sum_{n=0}^{N-1} \fict{m}_n \sum_{d=0}^{D-1} \frac{1}{2} \nm{\vec{q}}_{n,d}\trans \mat{\Lambda} \nm{\vec{q}}_{n,d} \\
		&= \sum_{n=0}^{N-1} \sum_{k=0}^{P-1} \sum_{d=0}^{D-1} \frac{1}{2} \fict{m}_n \omega_\links^2 \lambda_k (\nm{q}_{n,d}\bead{k})^2 \\
		&= \sum_{n,k,d} \frac{1}{2} \fict{m}_n \omega_k^2 (\nm{q}_{n,d}\bead{k})^2,
	\end{align}
	\end{subequations}
	where we have introduced the mode-specific frequencies
	\begin{align}
		\omega_k
		&= 2 \omega_\links \sin{\left[ \frac{\pi k}{2 P} \right]}
		= \frac{2}{\hbar \tau} \sin{\left[ \frac{\pi k}{2 P} \right]},
	\end{align}
	whose distribution is shown in \cref{fig:pigs-nm-omegas}.
	In these coordinates, we have obtained $N P D$ independent harmonic oscillators with masses $\fict{m}_n$ and frequencies $\omega_k$.

	\begin{figure}[H]
		\centering
		\includegraphics[width=\textwidth]{24/pigs_nm_omegas}
		\caption[
			PIGS normal mode frequency distribution
		]{
			Distribution of $\omega_k$ for PIGS free particle normal modes for $P = 129$.
			The scale is arbitrary, set by a particular choice of $\tau$, but the general shape of the distribution is the same for any $P$ and $\tau$.
			Most of the modes are clustered at the higher frequencies.
		}
		\label{fig:pigs-nm-omegas}
	\end{figure}

	It remains necessary to find the canonical momenta for these new coordinates.
	Since there are no quantum momenta with which to confuse them, we will use the label $p$ (rather than $\fict{p}$) for the fictitious momenta.
	In terms of the Lagrangian $\mathcal{L}$, the canonical momenta are~\cite[35]{evans2008statistical}
	\begin{align}
		\nm{p}_{n,d}\bead{k}
		&= \dpd{\mathcal{L}}{{\dot{\nm{q}}_{n,d}\bead{k}}}
		= \sum_{j=0}^{P-1} \frac{\partial \mathcal{L}}{\partial \dot{q}_{n,d}\bead{j}} \frac{\partial \dot{q}_{n,d}\bead{j}}{\partial \dot{\nm{q}}_{n,d}\bead{k}}
		= \sum_{j=0}^{P-1} p_{n,d}\bead{j} \frac{\partial \dot{q}_{n,d}\bead{j}}{\partial \dot{\nm{q}}_{n,d}\bead{k}}.
	\end{align}
	Since
	\begin{align}
		q_{n,d}\bead{j}(\nm{\vec{q}})
		&= (\mat{S}\trans \nm{\vec{q}})\bead{j}
	\end{align}
	and $\mat{S}$ is independent of time, we have that
	\begin{align}
		\dot{q}_{n,d}\bead{j}(\dot{\nm{\vec{q}}})
		&= (\mat{S}\trans \dot{\nm{\vec{q}}})\bead{j}
		= \sum_{k=0}^{P-1} s_{k,j} \dot{\nm{q}}_{n,d}\bead{k}
	\end{align}
	and
	\begin{align}
		\frac{\partial \dot{q}_{n,d}\bead{j}}{\partial \dot{\nm{q}}_{n,d}\bead{k}}
		&= s_{k,j}.
	\end{align}
	Therefore,
	\begin{align}
		\nm{p}_{n,d}\bead{k}
		&= \sum_{j=0}^{P-1} s_{k,j} p_{n,d}\bead{j}
		= (\mat{S} \vec{p}_{n,d})\bead{k}.
	\end{align}
	That is, to obtain the canonical momenta, we apply the same transformation as for the coordinates.
\end{DefAnswer}

\begin{DefAnswer}{ex:lepigs-algorithm}
	We wish to simulate the classical Hamiltonian given in \vref{eq:classical-hamiltonian} at reciprocal temperature $\beta$ using a Langevin thermostat.
	We choose the fictitious masses to be
	\begin{align}
		\fict{m}_n
		&= \frac{m_n}{\links}.
	\end{align}
	Since we are not particularly interested in particles, beads, or dimensions for the time being, we will index things with a generic subscript $f \in \{ 0, 1, \ldots, F - 1 \}$, signifying the $f$th degree of freedom, greatly reducing visual clutter.
	To find the simulation algorithm, we will closely follow the approach of refs.~\cite{bussi2007accurate,ceriotti2010efficient}.

	For the framework of the algorithm, we want to use the Trotter-like expansion of the real-time propagator presented in ref.~\cite{tuckerman1992reversible}, so we will need to figure out what this propagator is.\footnote{
		Some background on the subject is provided in ref.~\cite[44-50]{evans2008statistical} and ref.~\cite[31-35]{zwanzig2001nonequilibrium}.
	}
	Because it is more convenient for all parts of the algorithm other than application of the potential $V_\links$, we will do everything in the normal mode coordinates given by the transformation $\mat{S}$ in \vref{eq:pigs-nm-transform}.
	Since we are using Langevin dynamics, our equations of motion are~\cite{kubo1966fluctuation}
	\begin{subequations} \label{eq:langevin-eom}
	\begin{align}
		\dot{\nm{p}}_f(t)
		&= \nm{F}_f(\nm{\vec{q}}(t)) - \gamma_f \nm{p}_f(t) + R_f(t)
			\label{eq:langevin-eom-p} \\
		\dot{\nm{q}}_f(t)
		&= \frac{\nm{p}_f(t)}{m_f},
	\end{align}
	\end{subequations}
	where $\vec{\gamma}$ are \emph{frictions} and $\vec{R}(t)$ are \emph{random forces} (independent for each degree of freedom).
	In our notation, $\nm{F}_f$ is the force applied to a normal mode degree of freedom by the total effective potential:
	\begin{align}
		\nm{\vec{F}}(\nm{\vec{q}})
		&= -\frac{\partial \Veff}{\partial \nm{\vec{q}}}.
	\end{align}

	It may seem strange that we wanted to introduce temperature, but there is no mention of temperature in these equations.
	However, we have not yet explained what the random force needs to be!
	By the fluctuation-dissipation theorem, each element of the random force vector must obey~\cite[5-6]{zwanzig2001nonequilibrium}
	\begin{subequations}
	\begin{align}
		\mean{R_f(t)}
		&= 0 \\
		\mean{R_f(t) R_f(t')}
		&= \frac{2 m_f \gamma_f}{\beta} \ddf{t - t'}.
	\end{align}
	\end{subequations}
	It is crucial to note that $\beta = 1/\kB T$ here is exactly the temperature we are trying to simulate, and it is only indirectly related to concepts like path length and imaginary propagation time; this distinction becomes important when the classical polymers in the simulation may have different lengths.
	The autocorrelation of a standard normal random number (``Gaussian number'') $\xi_f(t)$ is~\cite{ceriotti2010efficient}
	\begin{align}
		\mean{\xi_f(t) \xi_f(t')}
		&= \ddf{t - t'},
	\end{align}
	from which we deduce that
	\begin{align}
		R_f(t)
		&= \sqrt{\frac{2 m_f \gamma_f}{\beta}} \xi_f(t).
	\end{align}

	To find the time evolution operator, we need the Fokker--Planck equation corresponding to the Langevin equations in \cref{eq:langevin-eom}.
	If we were dealing with Hamiltonian dynamics, this would just be the Liouville equation~\cite[32]{zwanzig2001nonequilibrium}
	\begin{align}
		\dpd{}{t} \phi
		&= -\hat{L} \phi
	\end{align}
	for the probability distribution function $\phi$, with the Liouvillian
	\begin{align}
		\hat{L}
		&= -\dpd{}{t}
		= \sum_f \dpd{H}{\nm{p}_f} \dpd{}{\nm{q}_f} - \dpd{H}{\nm{q}_f} \dpd{}{\nm{p}_f}.
	\end{align}
	However, \cref{eq:langevin-eom-p} is a stochastic differential equation, so things are not as pleasant as in the case of Hamiltonian dynamics.
	To deal with it, we use eqs.~(3.110), (3.118), (3.119), (4.96), and (4.97) from ref.~\cite[54-55,83]{risken1984fokker}.
	Because the force field will in general couple the degrees of freedom, we put them all together and consider a system of $2 F$ Langevin equations.
	We introduce the shorthand $\nm{\vec{\Gamma}}$ to mean the combined vector of momenta $\nm{\vec{p}}$ and positions $\nm{\vec{q}}$.
	The $h$ and $g$ functions of ref.~\cite{risken1984fokker} are then
	\begin{subequations}
	\begin{align}
		h_{\nm{p},f}(\nm{\vec{\Gamma}})
		&= \nm{F}_f(\nm{\vec{q}}) - \gamma_f \nm{p}_f &
		h_{\nm{q},f}(\nm{\vec{\Gamma}})
		&= \frac{\nm{p}_f}{m_f} \\
		g_{\nm{p},ff'}(\nm{\vec{\Gamma}})
		&= \delta_{f,f'} \sqrt{\frac{m_f \gamma_f}{\beta}} &
		g_{\nm{q},ff'}(\nm{\vec{\Gamma}})
		&= 0.
	\end{align}
	\end{subequations}
	In our case there is no explicit time dependence.
	The Kramers--Moyal coefficients are
	\begin{subequations}
	\begin{align}
		D_{\nm{p},f}(\nm{\vec{\Gamma}})
		&= \nm{F}_f(\nm{\vec{q}}) - \gamma_f \nm{p}_f &
		D_{\nm{q},f}(\nm{\vec{\Gamma}})
		&= \frac{\nm{p}_f}{m_f} \\
		D_{\nm{p},ff'}(\nm{\vec{\Gamma}})
		&= \delta_{f,f'} \frac{m_f \gamma_f}{\beta} &
		D_{\nm{q},ff'}(\nm{\vec{\Gamma}})
		&= 0,
	\end{align}
	\end{subequations}
	with all the higher coefficients zero.
	From this we can construct the (backward) Fokker--Planck operator\footnote{
	Ref.~\cite[83]{risken1984fokker} refers to this operator as $\boldsymbol{L}_\mathrm{FP}\adj$, while ref.~\cite[42-43]{zwanzig2001nonequilibrium} calls it $\mathcal{D}\adj$.
	}
	\begin{subequations}
	\begin{align}
		\hat{L}
		&= \sum_f
			D_{\nm{p},f}(\nm{\vec{\Gamma}}) \dpd{}{\nm{p}_f}
			+ D_{\nm{q},f}(\nm{\vec{\Gamma}}) \dpd{}{\nm{q}_f}
			+ D_{\nm{p},ff'}(\nm{\vec{\Gamma}}) \md{}{2}{\nm{p}_f}{}{\nm{p}_{f'}}{} \\
		&= \sum_f
			\nm{F}_f(\nm{\vec{q}}) \dpd{}{\nm{p}_f} - \gamma_f \nm{p}_f \dpd{}{\nm{p}_f}
			+ \frac{\nm{p}_f}{m_f} \dpd{}{\nm{q}_f}
			+ \frac{m_f \gamma_f}{\beta} \dpd[2]{}{\nm{p}_f} \\
		&= \sum_f
			\nm{F}_f(\nm{\vec{q}}) \dpd{}{\nm{p}_f}
			+ \frac{\nm{p}_f}{m_f} \dpd{}{\nm{q}_f}
			- \gamma_f \left( \nm{p}_f \dpd{}{\nm{p}_f} - \frac{m_f}{\beta} \dpd[2]{}{\nm{p}_f} \right),
				\label{eq:fokker-planck-operator}
	\end{align}
	\end{subequations}
	which gives us our real-time propagator $e^{\hat{L} t}$~\cite[42-43]{zwanzig2001nonequilibrium}.\footnote{
		Our expression for a single degree of freedom in \cref{eq:fokker-planck-operator} differs slightly from eq.~(4) of ref.~\cite{bussi2007accurate}, because we have chosen a different convention for the operator (using its adjoint instead).
	}

	We may now apply the factorization to this propagator.
	We will break the total time $t$ into $S$ time steps of length $\dt$: $t = S \dt$.
	Instead of the splitting recommended in ref.~\cite{bussi2007accurate}, we will group the ``kinetic'' spring potential (see \cref{eq:classical-potential-spring}) with the free particle motion and use the splitting given in ref.~\cite{ceriotti2010efficient}.
	The effective potential in \cref{eq:classical-potential} is already in the form we desire, so it should be simple to split the force.
	However, the force we are splitting is in the transformed coordinates, so we need to be careful.
	We define
	\begin{subequations}
	\begin{align}
		\vec{F}_K(\vec{q})
		&= -\dpd{\Vspring}{\vec{q}} \\
		\vec{F}_V(\vec{q})
		&= -\dpd{V_\links}{\vec{q}}.
	\end{align}
	\end{subequations}
	At this point we can no longer ignore the structure of the problem, and we return to the full subscripts for the degrees of freedom.
	Naturally,
	\begin{subequations}
	\begin{align}
		F_{n,d}\bead{j}(\vec{q})
		&= -\frac{\partial \Veff}{\partial q_{n,d}\bead{j}}
		= -\frac{\partial \Vspring}{\partial q_{n,d}\bead{j}} - \frac{\partial V_\links}{\partial q_{n,d}\bead{j}} \\
		&= F_{K,n,d}\bead{j}(\vec{q}) + F_{V,n,d}\bead{j}(\vec{q}).
	\end{align}
	\end{subequations}
	Less obviously,
	\begin{subequations}
	\begin{align}
		\nm{F}_{n,d}\bead{k}(\nm{\vec{q}})
		&= -\frac{\partial \Veff}{\partial \nm{q}_{n,d}\bead{k}}
		= -\sum_{j=0}^{P-1} \frac{\partial \Veff}{\partial q_{n,d}\bead{j}} \frac{\partial q_{n,d}\bead{j}}{\partial \nm{q}_{n,d}\bead{k}}
		= -\sum_{j=0}^{P-1} \left[ \frac{\partial \Vspring}{\partial q_{n,d}\bead{j}} + \frac{\partial V_\links}{\partial q_{n,d}\bead{j}} \right] \frac{\partial q_{n,d}\bead{j}}{\partial \nm{q}_{n,d}\bead{k}} \\
		&= -\sum_{j=0}^{P-1} s_{k,j} \left[ \frac{\partial \Vspring}{\partial q_{n,d}\bead{j}} + \frac{\partial V_\links}{\partial q_{n,d}\bead{j}} \right]
				\label{eq:force-nm-transform} \\
		&= \nm{F}_{K,n,d}\bead{k}(\nm{\vec{q}}) + \nm{F}_{V,n,d}\bead{k}(\nm{\vec{q}}).
	\end{align}
	\end{subequations}
	Thus, we may write
	\begin{subequations}
	\begin{align}
		\hat{L}_0
		&= \sum_{n,k,d}
				\frac{\nm{p}_{n,d}\bead{k}}{\fict{m}_n} \frac{\partial}{\partial \nm{q}_{n,d}\bead{k}}
				+ \nm{F}_{K,n,d}\bead{k}(\nm{\vec{q}}) \frac{\partial}{\partial \nm{p}_{n,d}\bead{k}} \\
		\hat{L}_V
		&= \sum_{n,k,d}
				\nm{F}_{V,n,d}\bead{k}(\nm{\vec{q}}) \frac{\partial}{\partial \nm{p}_{n,d}\bead{k}} \\
		\hat{L}_\gamma
		&= \sum_{n,k,d} \gamma_{n,d}\bead{k} \left(
				\frac{\fict{m}_n}{\beta} \frac{\partial^2}{\partial (\nm{p}_{n,d}\bead{k})^2}
				- \nm{p}_{n,d}\bead{k} \frac{\partial}{\partial \nm{p}_{n,d}\bead{k}}
			\right),
	\end{align}
	\end{subequations}
	so that $\hat{L} = \hat{L}_0 + \hat{L}_V + \hat{L}_\gamma$.
	This leads to (for a single time step)~\cite{tuckerman1992reversible,bussi2007accurate}
	\begin{align}
		e^{\hat{L} \dt}
		&= e^{\hat{L}_\gamma \frac{\dt}{2}}
				e^{\hat{L}_V \frac{\dt}{2}}
				e^{\hat{L}_0 \dt}
				e^{\hat{L}_V \frac{\dt}{2}}
				e^{\hat{L}_\gamma \frac{\dt}{2}}
			+ \bigo{\dt^3}.
	\end{align}
	The last piece we need is the result of each propagator on a state $\vec{\Gamma}(t)$.
	Let us proceed from the inside out.

	The propagator $e^{\hat{L}_0 \dt}$ describes the motion of a collection of independent harmonic oscillators with the equations of motion
	\begin{subequations}
	\begin{align}
		\dot{\nm{p}}_{n,d}\bead{k}(t)
		&= -\fict{m}_n \omega_k^2 \nm{q}_{n,d}\bead{k}(t) \\
		\dot{\nm{q}}_{n,d}\bead{k}(t)
		&= \frac{\nm{p}_{n,d}\bead{k}(t)}{\fict{m}_n}.
	\end{align}
	\end{subequations}
	The trajectory that is the solution to these differential equations is~\cite[31]{zwanzig2001nonequilibrium}
	\begin{subequations} \label{eq:harmonic-oscillator-trajectory-exact}
	\begin{align}
		\nm{p}_{n,d}\bead{k}(t\final)
		&= \cos{(\omega_k \dt)} \nm{p}_{n,d}\bead{k}(t\initial) - \fict{m}_n \omega_k \sin{(\omega_k \dt)} \nm{q}_{n,d}\bead{k}(t\initial) \\
		\nm{q}_{n,d}\bead{k}(t\final)
		&= \frac{1}{\fict{m}_n \omega_k} \sin{(\omega_k \dt)} \nm{p}_{n,d}\bead{k}(t\initial) + \cos{(\omega_k \dt)} \nm{q}_{n,d}\bead{k}(t\initial),
	\end{align}
	\end{subequations}
	which may be written more compactly as
	\begin{align}
		\begin{pmatrix}
			\nm{p}_{n,d}\bead{k}(t\final) \\[2 mm]
			\nm{q}_{n,d}\bead{k}(t\final)
		\end{pmatrix}
		&= \begin{pmatrix}
				\cos{\omega_k \dt} & -\fict{m}_n \omega_k \sin{\omega_k \dt} \\[2 mm]
				\frac{1}{\fict{m}_n \omega_k} \sin{\omega_k \dt} & \cos{\omega_k \dt}
			\end{pmatrix}
			\begin{pmatrix}
				\nm{p}_{n,d}\bead{k}(t\initial) \\[2 mm]
				\nm{q}_{n,d}\bead{k}(t\initial)
			\end{pmatrix}.
	\end{align}
	We need to be careful about the centroid mode, since $\omega_0 = 0$ and we'd rather not divide by that.
	Applying l'Hôpital's rule, we get the sinc limit:
	\begin{align}
		\lim_{\omega_k \to 0} \frac{\sin{\omega_k \dt}}{\omega_k \dt}
		&= \lim_{\omega_k \to 0} \cos{\omega_k \dt}
		= 1,
	\end{align}
	so the centroid is updated by
	\begin{align}
		\begin{pmatrix}
			\nm{p}_{n,d}\bead{0}(t\final) \\[2 mm]
			\nm{q}_{n,d}\bead{0}(t\final)
		\end{pmatrix}
		&= \begin{pmatrix}
				1 & 0 \\[2 mm]
				\frac{\dt}{\fict{m}_n} & 1
			\end{pmatrix}
			\begin{pmatrix}
				\nm{p}_{n,d}\bead{0}(t\initial) \\[2 mm]
				\nm{q}_{n,d}\bead{0}(t\initial)
			\end{pmatrix}.
	\end{align}
	If we make use of the function
	\begin{align}
		\sinc{\omega_k \dt}
		&= \begin{cases}
				1 & \text{if } \omega_k \dt = 0 \\
				\frac{\sin{\omega_k \dt}}{\omega_k \dt} & \text{otherwise}
			\end{cases},
	\end{align}
	we can write the general case as
	\begin{align}
		\begin{pmatrix}
			\nm{p}_{n,d}\bead{k}(t\final) \\[2 mm]
			\nm{q}_{n,d}\bead{k}(t\final)
		\end{pmatrix}
		&= \begin{pmatrix}
				\cos{\omega_k \dt} & -\fict{m}_n \omega_k \sin{\omega_k \dt} \\[2 mm]
				\frac{\dt}{\fict{m}_n} \sinc{\omega_k \dt} & \cos{\omega_k \dt}
			\end{pmatrix}
			\begin{pmatrix}
				\nm{p}_{n,d}\bead{k}(t\initial) \\[2 mm]
				\nm{q}_{n,d}\bead{k}(t\initial)
			\end{pmatrix}.
	\end{align}

	The propagator $e^{\hat{L}_V \frac{\dt}{2}}$ is extremely simple, and its effect is most easily determined using \cref{eq:exp-deriv}.
	Without making any additional approximations, the propagator may be written as
	\begin{align}
		\expb{\hat{L}_V \frac{\dt}{2}}
		&= \expb{\frac{\dt}{2} \sum_{n,k,d} \nm{F}_{V,n,d}\bead{k} \frac{\partial}{\partial \nm{p}_{n,d}\bead{k}}}
		= \expb{\frac{\dt}{2} \nm{\vec{F}}_{V} \cdot \frac{\partial}{\partial \nm{\vec{p}}}}.
	\end{align}
	Thus,
	\begin{subequations}
	\begin{align}
		(\nm{\vec{p}}(t\final), \nm{\vec{q}}(t\final))
		&= e^{\hat{L}_V \frac{\dt}{2}} (\nm{\vec{p}}(t\initial), \nm{\vec{q}}(t\initial)) \\
		&= \left( \nm{\vec{p}}(t\initial) + \frac{\dt}{2} \nm{\vec{F}}_V, \nm{\vec{q}}(t\initial) \right).
	\end{align}
	\end{subequations}
	The only potential difficulty with this result is that in practice $\vec{F}_V$ will most likely be expressed in Cartesian coordinates.
	However, this is easy to handle, since
	\begin{subequations}
	\begin{align}
		\nm{\vec{p}}_{n,d} + \frac{\dt}{2} \nm{\vec{F}}_{V,n,d}
		&= \mat{S} \vec{p}_{n,d} + \frac{\dt}{2} \mat{S} \vec{F}_{V,n,d} \\
		&= \mat{S} \left( \mat{S}\trans \nm{\vec{p}}_{n,d} + \frac{\dt}{2} \vec{F}_{V,n,d} \right)
	\end{align}
	\end{subequations}
	means that we can update $\vec{p}$ in Cartesian coordinates using our regular expression for the force.

	Finally, we must attack the thermostat propagator $e^{\hat{L}_\gamma \frac{\dt}{2}}$.
	Quite conveniently, the terms of $\hat{L}_\gamma$ for the different degrees of freedom commute, so we only need to solve for the generic one-dimensional case:
	\begin{align}
		\expb{\frac{\dt}{2} \gamma \left( \frac{m}{\beta} \dpd[2]{}{\nm{p}} - \nm{p} \dpd{}{\nm{p}} \right)}.
	\end{align}
	This is an Ornstein--Uhlenbeck process, for which there is an exact ``updating formula''~\cite[551]{gillespie1992markov}:
	\begin{align}
		\nm{p}(t\final)
		&= e^{-\gamma \frac{\dt}{2}} \nm{p}(t\initial) + \sqrt{\frac{m}{\beta} (1 - e^{-\gamma \dt})} \xi(t\initial),
	\end{align}
	where $\xi(t)$ is again a Gaussian number.
	This result can be found by solving for the transition probability, which is a Gaussian distribution, and then applying the linear transformation theorem for a Gaussian distribution~\cite[27-28]{gillespie1992markov}.
	It is the same result as obtained in ref.~\cite{bussi2007accurate}, and to be consistent we will define the same coefficients
	\begin{subequations}
	\begin{align}
		c_1
		&= e^{-\gamma \frac{\dt}{2}} \\
		c_2
		&= \sqrt{\frac{m}{\beta} (1 - c_1^2)}.
	\end{align}
	\end{subequations}
	It is then straightforward to extend this to all the degrees of freedom:
	\begin{subequations}
	\begin{align}
		c_{1,n,d}\bead{k}
		&= e^{-\gamma_{n,d}\bead{k} \frac{\dt}{2}} \\
		c_{2,n,d}\bead{k}
		&= \sqrt{\frac{\fict{m}_n}{\beta} (1 - (c_{1,n,d}\bead{k})^2)} \\
		\nm{p}_{n,d}\bead{k}(t\final)
		&= c_{1,n,d}\bead{k} \nm{p}_{n,d}\bead{k}(t\initial) + c_{2,n,d}\bead{k} \xi_{n,d}\bead{k}(t\initial).
	\end{align}
	\end{subequations}

	We finally have a procedure for taking one time step in a LePIGS simulation.
	All the individual substeps are exact, and the only error we introduce is due to the factorization with a finite time step $\dt$.
	The complete procedure for a single time step $\dt$ is as follows:
	\begin{enumerate}[itemsep=1 mm]
		\item $\nm{p}_{n,d}\bead{k} \gets c_{1,n,d}\bead{k} \nm{p}_{n,d}\bead{k} + c_{2,n,d}\bead{k} \xi_{n,d}\bead{k}$ \label{item:alg-step-thermo}
		\item $p_{n,d}\bead{j} \gets p_{n,d}\bead{j} + \frac{\dt}{2} F_{V,n,d}\bead{j}(\vec{q})$ \label{item:alg-step-force}
		\item $
			\begin{pmatrix}
				\nm{p}_{n,d}\bead{k} \\[2 mm]
				\nm{q}_{n,d}\bead{k}
			\end{pmatrix}
			\gets \begin{pmatrix}
					\cos{\omega_k \dt} & -\fict{m}_n \omega_k \sin{\omega_k \dt} \\[2 mm]
					\frac{\dt}{\fict{m}_n} \sinc{\omega_k \dt} & \cos{\omega_k \dt}
				\end{pmatrix}
				\begin{pmatrix}
					\nm{p}_{n,d}\bead{k} \\[2 mm]
					\nm{q}_{n,d}\bead{k}
				\end{pmatrix}
			$
		\item Repeat substep~\ref{item:alg-step-force}.
		\item Repeat substep~\ref{item:alg-step-thermo}.
	\end{enumerate}
	The following are implied:
	\begin{itemize}
		\item Each substep is performed for all degrees of freedom before moving on to the next substep.
			The order of updates within a substep is irrelevant.
		\item The coordinates and momenta are converted between Cartesian and normal mode representations between substeps as necessary.
		\item Each ``invocation'' of $\xi_{n,d}\bead{k}$ results in an independent number randomly drawn from a standard normal distribution.
	\end{itemize}

	Since our form of the equations of motion for a single thermostatted ``free particle'' degree of freedom is the same as of those in eq.~(32) of ref.~\cite{ceriotti2010efficient}, we are justified in using the same optimal frictions as those they give in eq.~(36).
	Namely, for $k \ge 1$,
	\begin{align}
		\gamma_{n,d}\bead{k}
		&= 2 \omega_k.
	\end{align}
	The centroid frictions $\gamma_{n,d}\bead{0}$ remain free parameters of the simulation.
\end{DefAnswer}

\begin{DefAnswer}{ex:harmonic-oscillator-classical-trajectory}
	For a harmonic oscillator with mass $m$, angular frequency $\omega$, and initial conditions $p$, $q$, the total energy is
	\begin{align}
		H
		&= \frac{p^2}{2 m} + \frac{1}{2} m \omega^2 q^2.
	\end{align}
	The trajectory is given by \vref{eq:harmonic-oscillator-trajectory-exact}, but is reproduced here without the extra flair on the symbols:
	\begin{subequations}
	\begin{align}
		p_t
		&= \cos{(\omega t)} p - m \omega \sin{(\omega t)} q \\
		q_t
		&= \frac{1}{m \omega} \sin{(\omega t)} p + \cos{(\omega t)} q.
	\end{align}
	\end{subequations}

	We note that
	\begin{subequations}
	\begin{align}
		p_t q_t
		&= \left( \cos{(\omega t)} p - m \omega \sin{(\omega t)} q \right)
			\left( \frac{1}{m \omega} \sin{(\omega t)} p + \cos{(\omega t)} q \right) \\
		&= \frac{1}{m \omega} \cos{(\omega t)} \sin{(\omega t)} p^2
			+ 2 \cos^2{(\omega t)} p q
			- m \omega \cos{(\omega t)} \sin{(\omega t)} q^2
			- p q.
	\end{align}
	\end{subequations}
	Therefore, the action is
	\begin{subequations}
	\begin{align}
		S_t
		&= \frac{1}{m} \int_0^t\! \dif \tau \, (p_\tau)^2 - t H \\
		&= \frac{1}{m} \int_0^t\! \dif \tau \, (\cos{(\omega \tau)} p - m \omega \sin{(\omega \tau)} q)^2
			- \frac{t}{2} \frac{p^2}{m} - \frac{t}{2} m \omega^2 q^2 \\
		&= \frac{1}{m} \left(
				p^2 \int_0^t\! \dif \tau \, \cos^2{(\omega \tau)}
				- 2 m \omega p q \int_0^t\! \dif \tau \, \cos{(\omega \tau)} \sin{(\omega \tau)}
				+ m^2 \omega^2 q^2 \int_0^t\! \dif \tau \,  \sin^2{(\omega \tau)}
			\right) \notag \\
		&\qquad
			- \frac{1}{m} \left( \frac{t}{2} p^2 + \frac{t}{2} m^2 \omega^2 q^2 \right) \\
		&= \frac{1}{m} \left(
				p^2 \left[ \frac{t}{2} + \frac{\sin{(2 \omega t)}}{4 \omega} \right]
				- m p q \sin^2{(\omega t)}
				+ m^2 \omega^2 q^2 \left[ \frac{t}{2} - \frac{\sin{(2 \omega t)}}{4 \omega} \right]
				- \frac{t}{2} p^2
				- \frac{t}{2} m^2 \omega^2 q^2
			\right) \\
		&= \frac{p^2}{2 m} \frac{\sin{(2 \omega t)}}{2 \omega}
			- p q \sin^2{(\omega t)}
			- \frac{1}{2} m \omega^2 q^2 \frac{\sin{(2 \omega t)}}{2 \omega} \\
		&= \frac{1}{2} \left(
				\frac{1}{m \omega} \cos{(\omega t)} \sin{(\omega t)} p^2
				+ 2 \cos^2{(\omega t)} p q
				- m \omega \cos{(\omega t)} \sin{(\omega t)} q^2
				- 2 p q
			\right) \\
		&= \frac{1}{2} (p_t q_t - p q).
	\end{align}
	\end{subequations}

	For the HK prefactor, we need the elements of the monodromy matrix:
	\begin{subequations}
	\begin{align}
		\dpd{p_t}{p}
		&= \cos{(\omega t)}
		&
		\dpd{p_t}{q}
		&= -m \omega \sin{(\omega t)} \\
		\dpd{q_t}{p}
		&= \frac{1}{m \omega} \sin{(\omega t)}
		&
		\dpd{q_t}{q}
		&= \cos{(\omega t)}.
	\end{align}
	\end{subequations}
	Thus, the HK prefactor is
	\begin{subequations}
	\begin{align}
		R_t
		&= \sqrt{\frac{1}{2} \left(
				\dpd{p\coh{p}{q}_t}{p}
				+ \dpd{q\coh{p}{q}_t}{q}
				+ \frac{i}{\hbar \gamma} \dpd{p\coh{p}{q}_t}{q}
				- i \hbar \gamma \dpd{q\coh{p}{q}_t}{p}
			\right)} \\
		&= \sqrt{
				\cos{(\omega t)}
				- \frac{i}{2} \left( \frac{m \omega}{\hbar \gamma} + \frac{\hbar \gamma}{m \omega} \right) \sin{(\omega t)}
			}.
	\end{align}
	\end{subequations}
	Note that in the case $\gamma = m \omega / \hbar$, the coherent states are on resonance with the oscillator and
	\begin{align}
		R_t
		&= \sqrt{\cos{(\omega t)} - i \sin{(\omega t)}}.
	\end{align}
\end{DefAnswer}
