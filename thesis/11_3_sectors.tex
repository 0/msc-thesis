\section{Configuration space sectors}

The integrals in \cref{eq:lepigs-integral-function} are over all space for each degree of freedom, so it may be tempting to call the space of all possible spatial configurations of the polymers the ``configuration space,'' but we will find that this is too restrictive for our goals.
In conventional PIMC and PIMD, the configuration space is restricted to such configurations (typically referred to as \emph{diagonal} or $Z$), which are sufficient for physical quantities that are diagonal in the position representation~\cite{boninsegni2006worm}.
Inspired by the worm algorithm, we will embrace the notion of an extended configuration space, which is separated into disjoint \emph{sectors}~\cite{prokof1998worm,boninsegni2006worm}.
The aforementioned $Z$ configurations reside in the $Z$-sector.

A sector contains all the spatial configurations of a particular set of classical polymers, which is defined by the number of beads in the classical system as well as the way in which they are connected.
For example, the addition of an additional polymer (or even of a single bead) implies the discrete transition to another sector.
This sort of action is necessary for the sampling of a grand-canonical ensemble (for example to obtain chemical potentials)~\cite{herdman2014quantum}, but will not be pursued in the present work.
We will focus instead on the removal of elements from the sampling distribution.
For example, consider the $Z$-sector distribution from \cref{eq:lepigs-integral-function}:
\begin{align}
	\pi(\vec{q})
	&= \psiT(\vec{q}\bead{0})
		\left[ \prod_{j=0}^{P-2} \expb{
			-\frac{m}{2 \hbar^2 \tau} \abs{\vec{q}\bead{j} - \vec{q}\bead{j+1}}^2
			- \frac{\tau}{2} \left( V(\vec{q}\bead{j}) + V(\vec{q}\bead{j+1}) \right)
		} \right]
		\psiT(\vec{q}\bead{P-1}).
			\label{eq:ugly-distribution1}
\end{align}
We may choose to remove the link between beads $M-1$ and $M$ (where $M = \links / 2$ is the index of the middle bead), which would lead to the following distribution in a different sector:
\begin{align}
	\pi(\vec{q})
	&= \psiT(\vec{q}\bead{0})
		\left[ \prod_{j=0}^{M-2} \expb{
			-\frac{m}{2 \hbar^2 \tau} \abs{\vec{q}\bead{j} - \vec{q}\bead{j+1}}^2
			- \frac{\tau}{2} \left( V(\vec{q}\bead{j}) + V(\vec{q}\bead{j+1}) \right)
		} \right] \notag \\
	&\qquad\qquad\times
		\left[ \prod_{j=M}^{P-2} \expb{
			-\frac{m}{2 \hbar^2 \tau} \abs{\vec{q}\bead{j} - \vec{q}\bead{j+1}}^2
			- \frac{\tau}{2} \left( V(\vec{q}\bead{j}) + V(\vec{q}\bead{j+1}) \right)
		} \right]
		\psiT(\vec{q}\bead{P-1}).
			\label{eq:ugly-distribution2}
\end{align}
Note that this removal involves both the ``kinetic'' spring potential between the beads and the corresponding halves of the quantum potential.
We may perform a more surgical removal and only take out the spring, leading to another distribution in another sector:
\begin{align}
	\pi(\vec{q})
	&= \psiT(\vec{q}\bead{0})
		\left[ \prod_{j=0}^{M-2} \expb{
			-\frac{m}{2 \hbar^2 \tau} \abs{\vec{q}\bead{j} - \vec{q}\bead{j+1}}^2
			- \frac{\tau}{2} \left( V(\vec{q}\bead{j}) + V(\vec{q}\bead{j+1}) \right)
		} \right] \notag \\
	&\qquad\qquad\times
		\expb{-\frac{\tau}{2} \left( V(\vec{q}\bead{M-1}) + V(\vec{q}\bead{M}) \right)} \notag \\
	&\qquad\qquad\times
		\left[ \prod_{j=M}^{P-2} \expb{
			-\frac{m}{2 \hbar^2 \tau} \abs{\vec{q}\bead{j} - \vec{q}\bead{j+1}}^2
			- \frac{\tau}{2} \left( V(\vec{q}\bead{j}) + V(\vec{q}\bead{j+1}) \right)
		} \right]
		\psiT(\vec{q}\bead{P-1}).
			\label{eq:ugly-distribution3}
\end{align}

An entire simulation may be done in a single sector ($Z$ or otherwise), in which case the equations of motion may need to be tweaked, but it is still a straightforward simulation of \emph{some} classical canonical ensemble.
The more interesting case is when the sector is changed during the course of a simulation.
This possibility is not considered in the present work, but it is necessary in order to implement the worm algorithm~\cite{boninsegni2006worm}, sampling of the grand-canonical ensemble~\cite{herdman2014quantum}, and efficient sampling for particle entanglement~\cite{herdman2014path}.
