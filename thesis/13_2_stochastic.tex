\section{Stochastic integration}

\collab{Neil Raymond}

\begin{figure}[h]
	\centering
	\includegraphics[width=\textwidth]{13/path_explanation}
	\caption[
		Graphical notation for survival amplitude
	]{
		Details of the graphical notation used for the survival amplitude.
		Dashed box indicates the region of interest.
	}
	\label{fig:survival-path-explanation}
\end{figure}

Most interesting problems will be too large for us to find their survival amplitudes by direct integration, so we are forced to use a stochastic method.
The main issue is that there is no obvious sampling distribution for the momenta.
One cannot, after all, uniformly sample from $\intcc{-\infty, \infty}$!
In order to remedy this problem, we introduce a Gaussian weight into the integrand.

Our goal is still to obtain the value of the integrals in \vref{eq:survival-hk}, but this time without constructing grids.
Once we add the Gaussian weight, we instead obtain the expression
\begin{align}
	S_0\hkg(t)
	&= \frac{1}{2 \pi \hbar} \iiiint\! \dif p \dif q\LLup \dif q \dif q\RRup \,
			R\coh{p}{q}_t \braket{0 | q\LLup} \braket{q\RRup | 0}
			\expb{-\frac{\gamma}{2} \left( (q\LLup - q\coh{p}{q}_t)^2 + (q\RRup - q)^2 \right)} \notag \\
	&\qquad\qquad\qquad\times
			\expb{
				-\frac{p^2}{2 \sigma_p^2}
				+ \frac{i}{\hbar} \left( S\coh{p}{q}_t + p\coh{p}{q}_t (q\LLup - q\coh{p}{q}_t) - p (q\RRup - q) \right)
			}.
\end{align}
This introduces $\sigma_p$, which is yet another parameter, but that is not as bad as it might seem at first: the trade-off is that we no longer have all the grid parameters to tune.
Additionally, we choose $\gamma$ so that we are able to sample all three position coordinates from the same PIGS simulation.

Recall that we may evaluate integrals of the form
\begin{align}
	\frac{
			\int\! \dif \vec{q} \, \pi(\vec{q}) \mathcal{S}(\vec{q})
		}{
			\int\! \dif \vec{q} \, \pi(\vec{q})
		}
\end{align}
by sampling from $\pi(\vec{q})$ and evaluating $\mathcal{S}(\vec{q})$ at the sampled coordinates.
Consider the distribution
\begin{align}
	\pi_q(q\LL, q\MM, q\RR)
	&= \braket{0 | q\LL} \expb{-\frac{m}{2 \hbar^2 \tau} \left( \left( q\LL - q\MM \right)^2 + \left( q\MM - q\RR \right)^2 \right)} \braket{q\RR | 0},
		\label{eq:hk-dist-q}
\end{align}
which we have written in a suggestive form (with $L = M - 1$ and $R = M + 1$).
What it suggests is the path drawn in \cref{fig:survival-path-explanation}, which is a complete path in the sense that its beads are all connected, but it lacks the interactions from the two links of interest.\footnote{
	One subtlety here is that we must expand each of the wavefunctions $\braket{q | 0}$ into fragments of length $(\beta - \tau) / 2$ instead of the usual $\beta / 2$, because the piece in the middle is of ``length'' $2 \tau$ and we still require the full path to be $\beta$ long.
}
Consider also the distribution
\begin{align}
	\pi_p(p)
	&= \expb{-\frac{p^2}{2 \sigma_p^2}},
\end{align}
which we combine with the previous one to make the product distribution
\begin{subequations}
\begin{align}
	& \pi(p, q\LL, q\MM, q\RR) \notag \\
	&= \pi_p(p) \pi_q(q\LL, q\MM, q\RR) \\
	&= \expb{-\frac{p^2}{2 \sigma_p^2}}
		\braket{0 | q\LL} \expb{-\frac{m}{2 \hbar^2 \tau} \left( (q\LL - q\MM)^2 + (q\MM - q\RR)^2 \right)} \braket{q\RR | 0}.
\end{align}
\end{subequations}
Because they are obtained from a LePIGS simulation, the wavefunctions $\braket{q | 0}$ will not be normalized.
However, due to the deformation, we are not interested in the normalization, so we may disregard it and write
\begin{align}
	S_0\hkg(t)
	&\propto \iiiint\! \dif p \dif q\LL \dif q\MM \dif q\RR \,
			\pi(p, q\LL, q\MM, q\RR) R\coh{p}{q\MM}_t \notag \\
	&\qquad\qquad\times
			\expb{-\frac{m}{2 \hbar^2 \tau} \left( \left( q\LL - q\coh{p}{q\MM}_t \right)^2 - \left( q\LL - q\MM \right)^2 \right)} \notag \\
	&\qquad\qquad\times
			\expb{\frac{i}{\hbar} \left( S\coh{p}{q\MM}_t + p\coh{p}{q\MM}_t \left( q\LL - q\coh{p}{q\MM}_t \right) - p \left( q\RR - q\MM \right) \right)}.
\end{align}
We have explicitly chosen $\gamma = m / \hbar^2 \tau$, and as a result our estimator is
\begin{align}
	& \mathcal{S}_\mathrm{P}(p, q\LL, q\MM, q\RR, t) \notag \\
	&= R\coh{p}{q\MM}_t \expb{-\frac{m}{2 \hbar^2 \tau} \left( \left( q\LL - q\coh{p}{q\MM}_t \right)^2 - \left( q\LL - q\MM \right)^2 \right)} \notag \\
	&\qquad\qquad\times
		\expb{\frac{i}{\hbar} \left( S\coh{p}{q\MM}_t + p\coh{p}{q\MM}_t \left( q\LL - q\coh{p}{q\MM}_t \right) - p \left( q\RR - q\MM \right) \right)}.
\end{align}
Since this is the obvious estimator to use, we will refer to it as the \emph{primitive estimator of the survival amplitude}, with the position distribution
\begin{align}
	\pi_\mathrm{P}(q\LL, q\MM, q\RR)
	&= \symbdistwide{13/dist_primitive}.
\end{align}

We notice that even though we have preemptively removed the extraneous potentials from the distribution, we still have a spring contribution that we remove in the estimator.
If we excise it, we are left with the position distribution
\begin{align}
	\pi_\mathrm{M}(q\LL, q\MM, q\RR)
	&= \symbdistwide{13/dist_minimal}
\end{align}
and the corresponding \emph{minimal estimator of the survival amplitude}
\begin{align}
	\mathcal{S}_\mathrm{M}(p, q\LL, q\MM, q\RR, t)
	&= R\coh{p}{q\MM}_t \expb{-\frac{m}{2 \hbar^2 \tau} \left( q\LL - q\coh{p}{q\MM}_t \right)^2} \notag \\
	&\qquad\times
		\expb{\frac{i}{\hbar} \left( S\coh{p}{q\MM}_t + p\coh{p}{q\MM}_t \left( q\LL - q\coh{p}{q\MM}_t \right) - p \left( q\RR - q\MM \right) \right)}.
\end{align}
We hope that the reader is not biased toward or against any estimator names due to the results of \cref{chap:renyi}.


\subsection{Harmonic oscillator}

Before we try this stochastic approach with the harmonic oscillator system from \cref{sec:semiclassical-numerical-ho}, we can see (using the numerical method used previously) what happens when we deform the integral with the Gaussian weight and then explicitly renormalize the survival amplitude.
Unless otherwise specified, the parameter values from \cref{tab:model-sas-harmonic-oscillator-stochastic} are used for this model system.
The results shown in \cref{fig:harmonic-oscillator-survival-smoothing-norm} are very good over a wide range of $\sigma_p$.\footnote{
	In fact, for the narrowest momentum distribution, we end up using only a single momentum: zero.
	Yet the harmonic oscillator is fine with this.
}
Thus, we may expect that performing the same deformed integral using the stochastic method would result in a smooth survival amplitude.

\begin{figure}
	\centering
	\includegraphics[width=\textwidth]{13/harmonic_oscillator_sas_smoothing_norm}
	\caption[
		Harmonic oscillator survival amplitude with added Gaussian weight
	]{
		Harmonic oscillator ground state survival amplitude with added Gaussian weight and renormalization.
		Dashed curves indicate the exact answer.
	}
	\label{fig:harmonic-oscillator-survival-smoothing-norm}
\end{figure}

On the contrary, we find the disappointing result in \cref{fig:harmonic-oscillator-survival-primitive}.
The overall frequency appears to be correct, but the shapes of both the real and imaginary components are distorted, and the error bars are incredibly large at some points.
One may suspect that this is due to a poor choice of parameters in the LePIGS simulation.
However, for the harmonic oscillator system, we can sample from $\pi_q(q\LL, q\MM, q\RR)$ in \cref{eq:hk-dist-q} directly, since it is Gaussian:
\begin{align}
	\pi_q(q\LL, q\MM, q\RR)
	&\propto \expb{
			-\frac{m \omega}{2 \hbar} \left( \left( q\LL \right)^2 + \left( q\RR \right)^2 \right)
		} \notag \\
	&\qquad\times
		\expb{
			-\frac{m}{2 \hbar^2 \tau} \left( \left( q\LL - q\MM \right)^2 + \left( q\MM - q\RR \right)^2 \right)
		}.
\end{align}
This allows us to bypass PIGS altogether and see what the survival amplitude should look like given a finite number of samples.
As we see in \cref{fig:harmonic-oscillator-survival-primitive-exact}, the situation was hopeless to start with.

\begin{table}
	\begin{center}
	\begin{tabular}{ c c c c c c c }
		\toprule
		{$\beta / \si{\per\kelvin}$} & {$\tau / \si{\per\kelvin}$} & {$P$} & {$\dt / \si{\pico\second}$} & $\gamma\bead{0} / \si{\per\pico\second}$ & {$\dt\alt / \si{\pico\second}$} & {$\sigma_p / \si{\gram\nano\meter\per\pico\second\per\mole}$} \\
		\midrule
		8 & 0.125 & 65 & 1 & 0.1 & 1 & 1 \\
		\bottomrule
	\end{tabular}
	\end{center}
	\caption[
		Selected parameters for harmonic oscillator (stochastic)
	]{
		Selected parameters for the harmonic oscillator model system using the stochastic method.
	}
	\label{tab:model-sas-harmonic-oscillator-stochastic}
\end{table}

\begin{figure}
	\centering
	\includegraphics[width=\textwidth]{13/harmonic_oscillator_sas_stochastic_primitive}
	\caption[
		Harmonic oscillator survival amplitude using primitive estimator
	]{
		Harmonic oscillator ground state survival amplitude with stochastic sampling using the primitive estimator.
		Average of 16 survival amplitudes from $\num{1e6}$ samples each.
		Dashed curves indicate the exact answer.
	}
	\label{fig:harmonic-oscillator-survival-primitive}
\end{figure}

\begin{figure}
	\centering
	\includegraphics[width=\textwidth]{13/harmonic_oscillator_sas_stochastic_primitive_exact}
	\caption[
		Harmonic oscillator survival amplitude using primitive estimator (exact sampling)
	]{
		Harmonic oscillator ground state survival amplitude with exact stochastic sampling using the primitive estimator.
		Average of 16 survival amplitudes from $\num{1e6}$ samples each.
		Dashed curves indicate the exact answer.
	}
	\label{fig:harmonic-oscillator-survival-primitive-exact}
\end{figure}

The minimal estimator fares much better, as shown in \cref{fig:harmonic-oscillator-survival-minimal}.
All the error bars are fairly small and the mean values are mostly along smooth curves.
Again, we get much better results if we break paths in the simulation and sample from a different sector.
Perhaps surprising in this particular case is the fact that $\pi_\mathrm{M}(q\LL, q\MM, q\RR)$ is composed of two independent pieces!

\begin{figure}
	\centering
	\includegraphics[width=\textwidth]{13/harmonic_oscillator_sas_stochastic_minimal}
	\caption[
		Harmonic oscillator survival amplitude using minimal estimator
	]{
		Harmonic oscillator ground state survival amplitude with stochastic sampling using the minimal estimator.
		Average of 16 survival amplitudes from $\num{1e6}$ samples each.
		Dashed curves indicate the exact answer.
	}
	\label{fig:harmonic-oscillator-survival-minimal}
\end{figure}

We have not yet attempted to apply the stochastic approach to the double well system from \cref{sec:semiclassical-numerical-dw}.
Although we do not know of any \textit{a priori} reasons it should fail, our efforts for that system have been focused on first improving the numerical integration results.
