\chapter{Code listings}

The following listings show the Maple\texttrademark{} scripts and their outputs for some of the calculations performed in the present work.\footnote{
	Maple 17. Maplesoft, a division of Waterloo Maple Inc., Waterloo, Ontario.
}
Maple is a trademark of Waterloo Maple Inc.


\begin{lst}{lst:radial-spherical-a}{Integration of \vref{eq:radial-spherical-a}.}
interface(showassumed=0):
assume(R >= 0, r > 0, r <= R):

Ltheta1 := r -> -r / (2 * R):
Lplus := (r, x) -> -r * x + sqrt(R^2 + r^2 * (x^2 - 1)):

int1 := int(int(r2^2, r2=0..R), x=-1..Ltheta1(r)):
int2 := int(int(r2^2, r2=0..Lplus(r, x)), x=Ltheta1(r)..1):
simplify(int1 + int2);
\end{lst}

\noindent outputs

\begin{lstplain}
     3        2           3
2/3 R  - 1/2 R  r + 1/24 r
\end{lstplain}


\begin{lst}{lst:radial-spherical-b}{Integration of \vref{eq:radial-spherical-b}.}
interface(showassumed=0):
assume(R >= 0, r > 0, r^2 <= 2 * R^2):

Ltheta1 := r -> -r / (2 * R):
Ltheta2 := r -> -sqrt(1 - (R/r)^2):
Lminus := (r, x) -> -r * x - sqrt(R^2 + r^2 * (x^2 - 1)):
Lplus := (r, x) -> -r * x + sqrt(R^2 + r^2 * (x^2 - 1)):

int1 := int(int(r2^2, r2=Lminus(r, x)..R), x=-1..Ltheta1(r)):
int2 := int(int(r2^2, r2=Lminus(r, x)..Lplus(r, x)), x=Ltheta1(r)..Ltheta2(r)):
simplify(int1 + int2);
\end{lst}

\noindent outputs

\begin{lstplain}
     3          2         3
2/3 R  - 1/2 r R  + 1/24 r
\end{lstplain}


\begin{lst}{lst:radial-spherical-c}{Integration of \vref{eq:radial-spherical-c}.}
interface(showassumed=0):
assume(R >= 0, r > 0, 2 * R^2 - r^2 <= 0):

Ltheta1 := r -> -r / (2 * R):
Lminus := (r, x) -> -r * x - sqrt(R^2 + r^2 * (x^2 - 1)):

int1 := int(int(r2^2, r2=Lminus(r, x)..R), x=-1..Ltheta1(r)):
simplify(int1);
\end{lst}

\noindent outputs

\begin{lstplain}
     3         3          2
2/3 R  + 1/24 r  - 1/2 r R
\end{lstplain}


\begin{lst}{lst:radial-spherical-expected}{Integration of \vref{eq:radial-spherical-expected}.}
with(VectorCalculus):

S := Sphere(<0, 0, 0>, R):

int(int(1, [x1, y1, z1]=S), [x2, y2, z2]=S);
int(int((x1-x2)^2 + (y1-y2)^2 + (z1-z2)^2, [x1, y1, z1]=S), [x2, y2, z2]=S);
\end{lst}

\noindent outputs

\begin{lstplain}
     2  6
16 Pi  R
---------
    9

     2  8
32 Pi  R
---------
   15
\end{lstplain}


\begin{lst}{lst:radial-box-a}{Integration of \vref{eq:radial-box-a}.}
int(1/sqrt(r^2 - x^2 - y^2), y=0..sqrt(r^2 - x^2), x=0..r);
int(1/sqrt(r^2 - x^2 - y^2), y=0..sqrt(r^2 - x^2), x=sqrt(r^2 - M^2)..M);
\end{lst}

\noindent outputs

\begin{lstplain}
Pi r
----
 2

           2    2 1/2
Pi (M - (-M  + r )   )
----------------------
          2
\end{lstplain}


\begin{lst}{lst:radial-box-b}{Integration of \vref{eq:radial-box-b}.}
interface(showassumed=0):
assume(A >= 0, A >= y, B >= 0, M >= 0):

int(1/sqrt(A^2 - y^2), y=B..M);
\end{lst}

\noindent outputs

\begin{lstplain}
-arcsin(B/A) + arcsin(M/A)
\end{lstplain}


\begin{lst}{lst:radial-box-c}{Integration of \vref{eq:radial-box-c}.}
interface(showassumed=0):
assume(r >= 0, M >= 0, r >= sqrt(2) * M):

simplify(int(arcsin(x/sqrt(r^2 - x^2)), x=0..M));
simplify(int(-arcsin(sqrt(r^2 - 2 * x^2)/sqrt(r^2 - x^2)), x=0..M));
\end{lst}

\noindent outputs

\begin{lstplain}
                                                   2    2 1/2
               M          r Pi                (-2 M  + r )    r
M arcsin(-------------) - ---- + 1/2 r arctan(-----------------)
            2    2 1/2     4                          2
         (-M  + r )                                  M

                       2    2 1/2                      2    2 1/2
  r Pi            (-2 M  + r )                    (-2 M  + r )    r
- ---- - M arcsin(---------------) + 1/2 r arctan(-----------------)
   4                  2    2 1/2                          2
                   (-M  + r )                            M
\end{lstplain}


\begin{lst}{lst:radial-box-expected}{Integration of \vref{eq:radial-box-expected}.}
with(VectorCalculus):

P := Parallelepiped(-L/2..L/2, -L/2..L/2, -L/2..L/2):

L^3 * int(1, [x, y, z]=P);
L^3 * int(x^2 + y^2 + z^2, [x, y, z]=P);

L := 42:

L^3 * int(1, [x, y, z]=P);
L^3 * int(x^2 + y^2 + z^2, [x, y, z]=P);
evalf(L^3 * int(exp(-(x^2 + y^2 + z^2)), [x, y, z]=P));
\end{lst}

\noindent outputs

\begin{lstplain}
 6
L

  8
 L
----
 4

5489031744

2420662999104

412546.2847
\end{lstplain}
