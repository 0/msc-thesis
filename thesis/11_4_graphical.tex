\section{Graphical notation}

\label{sec:graphical}

Like every useful notation, Dirac's kets (and their dual bras) form a powerful abstraction over the underlying machinery.
In the case of the kets, not only is there less clutter for common activities, like taking inner products of elements of the Hilbert space,
\begin{align}
	\braket{\phi | \psi}
	&= \int\! \dif x \, \phi\conj(x) \psi(x),
\end{align}
but there are also conceptual advantages.
Kets are not tied to any basis, allowing us to think in more abstract terms.
They can also represent objects which cannot even exist in Hilbert space (such as the delta-function-normalized position basis states $\ket{q}$), and which would normally require the theory of distributions and a rigged Hilbert space to account for rigorously~\cite{de2005role}.

However, even Dirac notation can become tedious and opaque if one accumulates sufficiently many bras and kets, as tends to happen with path integrals.
If the bras, kets, and operators are further expanded, one obtains inscrutable expressions, such as those in \cref{eq:ugly-distribution1,eq:ugly-distribution2,eq:ugly-distribution3}.
It is therefore beneficial to introduce a graphical notation for representing paths, especially when the goal is to manipulate and discuss the structure of the paths at the level of beads and springs.

The elements in which we're interested are the beads themselves, the ``kinetic'' spring links between them, and the interaction terms due to the quantum potential.
The beads, being point particles, are naturally represented as points in a schematic drawing.
The springs connect beads and can be represented as wavy curves between the points.
The basic ``unit'' of interaction is half of a single term, since each link contributes half of the interactions on either side.
Additionally, interactions may either involve multiple particles or a single particle; the former are referred to as \emph{$N$-body} interactions and the latter are referred to as \emph{central} or \emph{trapping} interactions.

\begin{figure}
	\setlength{\figspacing}{10 mm}
	\centering
	\begin{subfigure}[b]{\textwidth}
		\includegraphics[width=\textwidth]{11/paths_with_central}
		\caption{
			Two interacting quantum particles depicted as classical polymers (paths) composed of $P$ beads, showing the Trotter springs (blue, wavy) and potentials (red, dashed) connecting the beads.
			The end beads experience only half of the regular potential, but they also feel a potential due to the trial function, which is not displayed here.
			The letters along the left are the particle labels and the numbers along the top are the bead indices.
			Most beads have been elided, as indicated by the $\cdots$, implying that their presence would add nothing of interest to the diagram.
		}
		\label{fig:paths-with-central}
		\vspace{\figspacing}
	\end{subfigure}
	\begin{subfigure}[b]{\textwidth}
		\includegraphics[width=\textwidth]{11/paths}
		\caption{
			The same situation as in \subref{fig:paths-with-central}, but with the central potentials omitted, resulting in a cleaner diagram.
		}
		\label{fig:paths}
		\vspace{\figspacing}
	\end{subfigure}
	\begin{subfigure}[b]{\textwidth}
		\includegraphics[width=\textwidth]{11/paths_no_link1}
		\caption{
			A similar situation to the one in \subref{fig:paths}, but without the link between beads $M-1$ and $M$.
		}
		\label{fig:paths-no-link1}
		\vspace{\figspacing}
	\end{subfigure}
	\begin{subfigure}[b]{\textwidth}
		\includegraphics[width=\textwidth]{11/paths_no_link2}
		\caption{
			A similar situation to the one in \subref{fig:paths}, but without the kinetic springs between beads $M-1$ and $M$.
		}
		\label{fig:paths-no-link2}
		\vspace{\figspacing}
	\end{subfigure}
	\mbox{}
	\hfill
	\begin{subfigure}[b]{0.4\textwidth}
		\centering
		\includegraphics[width=0.25\textwidth]{11/stubs}
		\caption{
			A more compact version of \subref{fig:paths}, focused on the link between $M-1$ and $M$.
		}
		\label{fig:stubs}
	\end{subfigure}
	\hfill
	\begin{subfigure}[b]{0.4\textwidth}
		\centering
		\includegraphics[width=0.25\textwidth]{11/stubs_no_link1}
		\caption{
			A more compact version of \subref{fig:paths-no-link1}, focused on the link between $M-1$ and $M$.
		}
		\label{fig:stubs-no-link1}
	\end{subfigure}
	\hfill
	\mbox{}
	\vspace{\figspacing}

	\mbox{}
	\hfill
	\begin{subfigure}[b]{0.4\textwidth}
		\centering
		\includegraphics[width=0.25\textwidth]{11/stubs_no_link2}
		\caption{
			A more compact version of \subref{fig:paths-no-link2}, focused on the link between $M-1$ and $M$.
		}
		\label{fig:stubs-no-link2}
	\end{subfigure}
	\hfill
	\begin{subfigure}[b]{0.4\textwidth}
		\centering
		\includegraphics[width=0.25\textwidth]{11/missing_link}
		\caption{
			The missing link from \subref{fig:paths-no-link1} and \subref{fig:stubs-no-link1}, showing only beads $M-1$ and $M$.
		}
		\label{fig:missing-link}
	\end{subfigure}
	\hfill
	\mbox{}
	\caption[
		Examples of path diagrams
	]{
		Examples of path diagrams, showing the distillation of the compact graphical notation.
	}
\end{figure}

The easiest way to understand this notation is to see a concrete example.
Let us consider a system of two particles $A$ and $B$, each of which experiences a central potential of some sort ($\hat{V}_A$ and $\hat{V}_B$).
Additionally, they interact with a coupling potential ($\hat{V}_{AB}$).
We shall encounter a model system like this very shortly.
The distribution from \cref{eq:ugly-distribution1} for our simple system may be represented visually as in \cref{fig:paths-with-central}.
The trial function is not displayed in the diagrams, as its presence (or absence) should be straightforward to describe in words.

Since we are often not interested in the central interactions, they may be omitted from the diagram, as in \cref{fig:paths}.
So far these diagrams have not been particularly eventful, but we can start to see how they may be useful if we represent the distributions from \cref{eq:ugly-distribution2,eq:ugly-distribution3}, which are shown in \cref{fig:paths-no-link1,fig:paths-no-link2}.
Despite the elision of many irrelevant beads (namely those between $1$ and $M-1$ and those between $M+1$ and $P-2$), these diagrams still display information that is not relevant to what we are trying to show: how the paths are affected around beads $M-1$ and $M$.
Of course, if we are careful to specify which beads we wish to focus on (in this case $M-1$ and $M$), we may disregard everything else, as in \cref{fig:stubs,fig:stubs-no-link1,fig:stubs-no-link2}.
In these diagrams, the tails on the left and right signify that we wish to refer to the rest of the paths as well.
Sometimes we want to refer only to a particular segment of the path, in which case these tails are omitted, as in \cref{fig:missing-link}.

The reader may understandably be bored at this point, as there has not been anything so far resembling a ``graphical notation,'' merely diagrams presented in a figure on a separate page.
That is far from convenient and only marginally useful!
However, these diagrams can be inserted in an intuitive way into mathematical statements:
\begin{subequations}
\begin{align}
	\symbdist{11/stubs_no_link1}
	&= \hugefrac{\symbdist{11/stubs}}{\symb{11/missing_link}} \\
	\symbdist{11/stubs}
	&= \symbdist{11/stubs_no_link1} \times \symb{11/missing_link}.
\end{align}
\end{subequations}
These two equations show the removal (and reinsertion) of a full link.
