\chapter{Rényi entropy for particle entanglement}
\chaptermark{Rényi entropy}

\label{chap:renyi}

Entropy is a well-known fundamental concept in both thermodynamics and information theory.
In the former, it appears as the Gibbs entropy,~\cite[48]{mcquarrie1976statistical}
\begin{align}
	S_\mathrm{G}
	&= -\kB \sum_n p_n \log{p_n},
\end{align}
and in the latter as Shannon entropy,~\cite{shannon1948mathematical}
\begin{align}
	S_\mathrm{S}
	&= -\sum_n p_n \log{p_n}.
\end{align}
In both cases, the sum is over all possible states, and $p_n$ is the probability of being in a given state, whether it is a microstate of the system or an output symbol on a channel.
Found in quantum mechanics as a connection between the two is the von Neumann entropy,~\cite[253]{wilde2013quantum}
\begin{align}
	S_\mathrm{vN}
	&= -\Tr{\hat{\rho} \log{\hat{\rho}}}.
\end{align}
In a basis which diagonalizes $\hat{\rho}$, we find
\begin{align}
	S_\mathrm{vN}
	&= -\sum_n \rho_n \log{\rho_n},
\end{align}
which is exactly analogous to the above entropies, since the interpretation of the diagonal elements of the density matrix is the probability of being in the corresponding basis states~\cite[102]{wilde2013quantum}.
The von Neumann entropy provides a measure of the \emph{purity} of a state: how \emph{pure} or \emph{mixed} the state is.
This makes intuitive sense: it assigns a number to the uncertainty about which pure state one actually holds if one only has a mixed state description of it~\cite[254]{wilde2013quantum}.
Any pure state naturally has zero entropy, while maximally-mixed states have the largest possible entropy for systems of that Hilbert space dimension~\cite[255]{wilde2013quantum}.

Some quantum states may be written as tensor products of other states:
\begin{subequations} \label{eq:product-state}
\begin{align}
	\ket{\psi_{AB}}
	&= \ket{\psi_A} \otimes \ket{\psi_B} \\
	\hat{\rho}_{AB}
	&= \hat{\rho}_A \otimes \hat{\rho}_B.
\end{align}
\end{subequations}
Here $A$ and $B$ describe some disjoint sets of degrees of freedom of the system, and it is possible to perform the trace over one set to produce exactly the state for the subsystem containing the remaining degrees of freedom:
\begin{align}
	\Tr_B{\hat{\rho}_{AB}}
	&= \Tr_B{(\hat{\rho}_A \otimes \hat{\rho}_B)}
	= \Tr_B{\hat{\rho}_A} \otimes \Tr_B{\hat{\rho}_B}
	= \hat{\rho}_A.
\end{align}
Such states are known as \emph{product states}.
Importantly, not all states have this property.
Those which are not product states are called \emph{entangled} and they cannot be written in the form of \cref{eq:product-state}~\cite[83]{wilde2013quantum}.\footnote{
	In the case of mixed states, there is the third category of \emph{separable states}~\cite[114]{wilde2013quantum}, but these are not relevant for the present work.
}
Unlike for product states, for entangled states,
\begin{align}
	\hat{\rho}_{AB}
	\ne \Tr_B{(\hat{\rho}_{AB})} \otimes \Tr_A{(\hat{\rho}_{AB})}.
\end{align}

The best-known entangled states are the four so-called ``Bell states'' involving eigenstates $\ket{0}$ and $\ket{1}$ of a two-level system~\cite[88]{wilde2013quantum}.
One of the Bell states is provided here as a simple example:\footnote{
	We omit the ``$\otimes$'' symbol when the tensor product operation is implied by adjacent kets.
}
\begin{align}
	\ket{\Psi^-}
	&= \halfsqrt \left( \ket{0}_A \ket{1}_B - \ket{1}_A \ket{0}_B \right)
	= \halfsqrt \left( \ket{01} - \ket{10} \right).
\end{align}
This state has the density matrix (in the $\{ \ket{00}, \ket{01}, \ket{10}, \ket{11} \}$ basis)
\begin{align}
	\rho_{AB}
	&= \frac{1}{2} \begin{pmatrix}
			0 & 0 & 0 & 0 \\
			0 & 1 & -1 & 0 \\
			0 & -1 & 1 & 0 \\
			0 & 0 & 0 & 0
		\end{pmatrix}.
\end{align}
In this example, taking the trace over $B$ results in the \emph{reduced state}
\begin{align}
	\hat{\rho}_A
	&= \Tr_B{\hat{\rho}_{AB}}
	= \frac{1}{2} \left( \ketbraself{0} + \ketbraself{1} \right),
		\label{eq:bell-reduced}
\end{align}
which is the maximally mixed two-level state.
That is, we learn absolutely nothing about the part of the state in $A$ if we ignore the information we have about the part of the state in $B$.
Additionally, even though we started with a pure state, which can be written as a sum of kets, the result is a mixed state, which cannot.

Since the von Neumann entropy can be used as a measure of the ``mixedness'' of a state, we can use the von Neumann entropy of the reduced state as a measure of the entanglement of the original state~\cite{schumacher1995quantum,vedral1997quantifying}.
For example, since the von Neumann entropy of pure states is zero, and the partial trace of a pure product state results in a pure state, the von Neumann entanglement entropy of any pure product state is zero.
On the other hand, the von Neumann entropy of the state in \cref{eq:bell-reduced} is
\begin{align}
	S_\mathrm{vN}(\hat{\rho}_A)
	&= -\left( \frac{1}{2} \log{\frac{1}{2}} + \frac{1}{2} \log{\frac{1}{2}} \right)
	= \log{2},
\end{align}
which is the maximum possible von Neumann entropy for a two-level state~\cite[255]{wilde2013quantum}.

\begin{figure}
	\centering
	\includegraphics[width=0.4\textwidth]{12/particle_partition}
	\caption[
		Example particle partition
	]{
		Example particle partition used to investigate the particle entanglement between $A$ and $B$.
	}
	\label{fig:particle-partition}
\end{figure}

While the above discussion is focused on a pair of two-level systems, the ideas are general and apply to any system with at least two degrees of freedom.
In the case of an $N$-body system, we may partition the set of $N$ particles into disjoint subsets of $N_A$ and $N_B$ particles, as in \cref{fig:particle-partition}.
We refer to this as a \emph{particle partition} and the quantity in question will be \emph{particle entanglement}; other possibilities, such as partitioning space, will not be considered in the present work.
If our description of the system is in Cartesian coordinates $\vec{q}$, we may split these into $\vec{q}_A$ and $\vec{q}_B$, where the former contains the coordinates for the particles in $A$ and the latter for the particles in $B$.
Our density matrix\footnote{
	It is not strictly a matrix, since we are in the continuous position basis, but we can't be blamed for listening to Dirac in ref.~\cite[69-70]{dirac1981principles}!
} can then be written as $\rho(\vec{q}_A, \vec{q}_B ; \vec{q}_A', \vec{q}_B')$ and we may perform the partial trace as
\begin{align}
	\rho_A(\vec{q}_A ; \vec{q}_A')
	&= \int\! \dif \vec{q}_B \, \rho(\vec{q}_A, \vec{q}_B ; \vec{q}_A', \vec{q}_B).
\end{align}

Thus, we have (at least in principle) a prescription for finding the von Neumann entropy for particle entanglement of any $N$-body system.\footnote{
	Since we are usually interested in systems of \emph{indistinguishable} particles, we must also be careful to preserve permutation symmetry when identical particles are split between $A$ and $B$.
}
We just need to have access to the density matrix of the state of interest, but quantum mechanics rarely yields to such demands.
Thus, we must search for another method of measuring the entanglement of a state.
Conveniently, there is a generalization of the Shannon and von Neumann entropies, named the \emph{Rényi entropy}~\cite{herdman2014path,renyi1961measures}:
\begin{align}
	S_\alpha
	&= \frac{1}{1 - \alpha} \log{\left( \Tr{\hat{\rho}_A^\alpha} \right)}
\end{align}
for $\alpha \ge 0$.
It is a generalization in the sense that $S_\alpha \to S_\mathrm{vN}$ as $\alpha \to 1$ and was originally obtained by Rényi by relaxing the subadditivty requirement of Shannon (that the joint entropy must not exceed the sum of the individual entropies)~\cite{shannon1948mathematical,renyi1961measures}.
As we will see shortly, this measure of entropy can be used with existing methods of quantum statistical mechanics (as described in \cref{chap:introduction}) to get around the need to obtain $\hat{\rho}$ explicitly.


\section{Estimators}

\label{sec:estimators}

Although there are infinitely many Rényi entropies $S_\alpha$, in the present work we are only interested in the \emph{second Rényi entropy}
\begin{align}
	S_2
	&= -\log{\left( \Tr{\hat{\rho}_A^2} \right)},
		\label{eq:S2}
\end{align}
and the associated quantity $\Tr{\hat{\rho}_A^2}$ to which we shall refer simply as \emph{the trace}.
This particular measure of entropy has been used for studying entanglement of lattice systems~\cite{hastings2010measuring,stephan2012renyi} and more recently in the continuum~\cite{herdman2014path,herdman2014particle}.
To demonstrate why it is appealing in practice, we will make use of the \emph{replica trick}, which involves multiple copies of the system~\cite{hastings2010measuring}.
We start with the exact ground state density operator
\begin{align}
	\hat{\rho}
	&= \ketbraself{0}
\end{align}
and perform a partial trace over the degrees of freedom in partition $B$,\footnote{
	Technically, one partitions a set into subsets, but we will be sloppy with our terminology and refer to the subsets as ``partitions'' themselves.
} leaving us with the reduced density operator for partition $A$
\begin{align}
	\hat{\rho}_A
	&= \Tr_B{\hat{\rho}}.
\end{align}
If we square this and take the trace, we obtain the trace from \cref{eq:S2}.
We may perform these operations explicitly in the position representation:
\begin{subequations} \label{eq:trace}
\begin{align}
	\rho(\vec{q}_A, \vec{q}_B ; \vec{q}_A', \vec{q}_B')
	&= \braket{\vec{q}_A \vec{q}_B | 0} \braket{0 | \vec{q}_A' \vec{q}_B'} \\
	\rho_A(\vec{q}_A; \vec{q}_A')
	&= \int\! \dif \vec{q}_B \, \braket{\vec{q}_A \vec{q}_B | 0} \braket{0 | \vec{q}_A' \vec{q}_B} \\
	\rho_A^2(\vec{q}_A; \vec{q}_A'')
	&= \iiint\! \dif \vec{q}_A' \dif \vec{q}_B \dif \vec{q}_B' \,
			\braket{\vec{q}_A \vec{q}_B | 0} \braket{0 | \vec{q}_A' \vec{q}_B}
			\braket{\vec{q}_A' \vec{q}_B' | 0} \braket{0 | \vec{q}_A'' \vec{q}_B'} \\
	\Tr{\hat{\rho}_A^2}
	&= \iiiint\! \dif \vec{q}_A \dif \vec{q}_A' \dif \vec{q}_B \dif \vec{q}_B' \,
			\braket{\vec{q}_A \vec{q}_B | 0} \braket{0 | \vec{q}_A' \vec{q}_B}
			\braket{\vec{q}_A' \vec{q}_B' | 0} \braket{0 | \vec{q}_A \vec{q}_B'} \\
	&= \iiiint\! \dif \vec{q}_A\sys{\lambda} \dif \vec{q}_B\sys{\lambda} \dif \vec{q}_A\sys{\mu} \dif \vec{q}_B\sys{\mu} \,
			\braket{0 | \vec{q}_A\sys{\red{\mu}} \vec{q}_B\sys{\lambda}} \braket{\vec{q}_A\sys{\lambda} \vec{q}_B\sys{\lambda} | 0}
			\braket{0 | \vec{q}_A\sys{\red{\lambda}} \vec{q}_B\sys{\mu}} \braket{\vec{q}_A\sys{\mu} \vec{q}_B\sys{\mu} | 0}.
				\label{eq:trace-symbols}
\end{align}
\end{subequations}
The final step is nothing more than a reordering and relabelling of the various bits.\footnote{
	We introduce extra noise into our expressions with the $\vec{q}\sys{x}$ labels, but they allow us to talk about the replicas using reasonable names instead of ``unprimed'' and ``primed.''
}
If we replace the exact ground state $\ket{0}$ by the approximate PIGS ground state $\ketbeta{0}$ and expand it into beads and links (see \cref{chap:introduction} for details), \cref{eq:trace-symbols} resembles integrals over two paths, but with some of the indices permuted.

\begin{figure}
	\centering
	\includegraphics[width=\textwidth]{12/path_explanation}
	\caption[
		Graphical notation for Rényi entropy
	]{
		Details of the graphical notation used for the Rényi entropy.
		Dashed box indicates the region of interest.
		Note that we have grouped all paths for a single partition ($A$ or $B$) into a single effective path for the diagram.
		As a consequence, the potential interactions \emph{within} a partition are not displayed (which is fine, because they are not relevant).
		Additionally, for system $\lambda$ the path for partition $B$ appears at the top.
	}
	\label{fig:renyi-path-explanation}
\end{figure}

Before we really dig into the expression for the trace, let us first consider a simpler one:
\begin{align}
	Z^2
	&= \iiiint\! \dif \vec{q}_A\sys{\lambda} \dif \vec{q}_B\sys{\lambda} \dif \vec{q}_A\sys{\mu} \dif \vec{q}_B\sys{\mu} \,
			\betabraket{0 | \vec{q}_A\sys{\lambda} \vec{q}_B\sys{\lambda}} \braketbeta{\vec{q}_A\sys{\lambda} \vec{q}_B\sys{\lambda} | 0}
			\betabraket{0 | \vec{q}_A\sys{\mu} \vec{q}_B\sys{\mu}} \braketbeta{\vec{q}_A\sys{\mu} \vec{q}_B\sys{\mu} | 0}.
\end{align}
Here we have taken paths for two independent replicas of the system ($\lambda$ and $\mu$), multiplied them together, and integrated over all the positions.
We can write this using the graphical notation as
\begin{align}
	Z^2
	&= \int\! \dif \vec{q} \, \symbdist{12/path_unpermuted}.
\end{align}
The particulars of the graphical notation in this \namecref{chap:renyi} are outlined in \cref{fig:renyi-path-explanation}.
What is important to notice here is that, as the diagram alludes, the integrals are obviously separable.
On the other hand, we may express the trace from \cref{eq:trace-symbols} as
\begin{align}
	\Tr{\rho_A^2}
	&\propto \int\! \dif \vec{q} \, \symbdist{12/path_permuted},
		\label{eq:trace-unnormalized}
\end{align}
and it is clear that we cannot view this as a product of integrals.
We have been careful to claim that these quantities are not equal, but only proportional.
This is because when we substitute $\ketbeta{0}$ for $\ket{0}$ we must take into account the normalization, which amounts to $1/Z^2$, leading to the ratio
\begin{align}
	\Tr{\rho_A^2}
	&= \hugefrac{
			\int\! \dif \vec{q} \, \symbdist{12/path_permuted}
		}{
			\int\! \dif \vec{q} \, \symbdist{12/path_unpermuted}
		}.
\end{align}
As the astute reader has noticed, this is equivalent in form to \vref{eq:lepigs-integral-function}, with the distribution composed of two replicas of the system, and the estimator function
\begin{align}
	\mathcal{N}_\mathrm{P}
	&= \hugefrac{
			\symb{12/link_permuted}
		}{
			\symb{12/link_unpermuted}
		}
	= \hugefrac{
			\symb{12/link_permuted_simplified}
		}{
			\symb{12/link_unpermuted_simplified}
		}.
			\label{eq:renyi-estimator-primitive}
\end{align}
We will refer to $\mathcal{N}_\mathrm{P}$ as the \emph{primitive estimator of the trace}.
Now that the brunt of the work has been done, we may write the result down using conventional notation:
\begin{align}
	\mathcal{N}_\mathrm{P}
	&=
	\expb{\red{-}\sum_{i \in A} \frac{m_i}{2 \hbar^2 \tau} \left(
			\abs{\vec{q}_i\beadsys{M-1}{\lambda} - \vec{q}_i\beadsys{M}{\red{\mu}}}^2
			+ \abs{\vec{q}_i\beadsys{M-1}{\mu} - \vec{q}_i\beadsys{M}{\red{\lambda}}}^2
		\right)} \notag \\
	&\times
	\expb{\red{+} \sum_{i \in A} \frac{m_i}{2 \hbar^2 \tau} \left(
			\abs{\vec{q}_i\beadsys{M-1}{\lambda} - \vec{q}_i\beadsys{M}{\red{\lambda}}}^2
			+ \abs{\vec{q}_i\beadsys{M-1}{\mu} - \vec{q}_i\beadsys{M}{\red{\mu}}}^2
		\right)} \notag \\
	&\times
	\expb{\red{-}\frac{\tau}{2} \left(
			V_{AB}(\vec{q}_A\beadsys{M}{\lambda}, \vec{q}_B\beadsys{M}{\red{\mu}})
			+ V_{AB}(\vec{q}_A\beadsys{M}{\mu}, \vec{q}_B\beadsys{M}{\red{\lambda}})
		\right)} \notag \\
	&\times
	\expb{\red{+} \frac{\tau}{2} \left(
			V_{AB}(\vec{q}_A\beadsys{M}{\lambda}, \vec{q}_B\beadsys{M}{\red{\lambda}})
			+ V_{AB}(\vec{q}_A\beadsys{M}{\mu}, \vec{q}_B\beadsys{M}{\red{\mu}})
		\right)},
\end{align}
where $V_{AB}(\vec{q}_A, \vec{q}_B)$ is the sum of all potentials that act on particles in \emph{both} partitions.\footnote{
	When the system has only pairwise interactions, this is straightforward: if an interaction affects one particle in $A$ and one in $B$, it should be included.
	In the general case, where interactions may include more than two particles, the requirement may become more clear if inverted: $V_{AB}(\vec{q}_A, \vec{q}_B)$ is the sum of all potentials that are \emph{not} restricted to particles in only one partition.
}
Thus, if we can sample path configurations for the two-replica system, all we need to do is average the estimator over those configurations to obtain an estimate of the trace, from which we may obtain $S_2$.

We can generalize this approach by making the observation that
\begin{align}
	\mean{\mathcal{N}}_{\pi}
	&= \frac{\int\! \dif \vec{q} \, \pi(\vec{q}) \mathcal{N}(\vec{q})}{\int\! \dif \vec{q} \, \pi(\vec{q})}
	= \frac{\int\! \dif \vec{q} \, \pi'(\vec{q}) \mathcal{N}'(\vec{q})}{\int\! \dif \vec{q} \, \pi'(\vec{q}) \mathcal{D}'(\vec{q})}
	= \frac{\int\! \dif \vec{q} \, \pi'(\vec{q}) \mathcal{N}'(\vec{q})}{\int\! \dif \vec{q} \, \pi'(\vec{q})} \frac{\int\! \dif \vec{q} \, \pi'(\vec{q})}{\int\! \dif \vec{q} \, \pi'(\vec{q}) \mathcal{D}'(\vec{q})}
	= \frac{\mean{\mathcal{N}'}_{\pi'}}{\mean{\mathcal{D}'}_{\pi'}}
		\label{eq:dist-change}
\end{align}
for a suitably chosen distribution $\pi'(\vec{q})$ and estimators $\mathcal{N}'(\vec{q})$ and $\mathcal{D}'(\vec{q})$.
Obviously, we may choose anything whatsoever for $\pi'(\vec{q})$ and offload the details onto the estimators by dividing out the unnecessary bits and multiplying in the required ones, but most arbitrary choices of distribution would not be very sensible.
One natural approach is to choose a distribution which requires nothing to be divided out, and only the minimal amount to be multiplied in.
In our case, this results in the distribution
\begin{align}
	\symbdist{12/dist_minimal}
\end{align}
and the estimators
\begin{align}
	\mathcal{N}_\mathrm{M}
	&= \symb{12/link_permuted_simplified}
	&
	\mathcal{D}_\mathrm{M}
	&= \symb{12/link_unpermuted_simplified}.
			\label{eq:renyi-estimator-minimal}
\end{align}
We will refer to the ratio estimator $\mean{\mathcal{N}_\mathrm{M}} / \mean{\mathcal{D}_\mathrm{M}}$ as the \emph{minimal estimator of the trace}.
As we might expect from \cref{eq:dist-change},
\begin{align}
	\mathcal{N}_\mathrm{P}
	&= \frac{\mathcal{N}_\mathrm{M}}{\mathcal{D}_\mathrm{M}}.
\end{align}

It seems that there is inherent asymmetry in reconnecting the paths on the ``left'' side (\ie{} between beads $M - 1$ and $M$), and that it would be better to use an even number of beads so that there is a middle link.
However, if we allow ourselves the use of perspective in our diagrams, we can show that this is not necessary.
In \cref{fig:permuted-twisted}, we see that the paths corresponding to \cref{eq:trace-unnormalized} cross on one side when drawn in the usual way.
However, in \cref{fig:permuted-untwisted}, we see the same paths without any asymmetry between the left and right sides.
When viewed in this way, it instead seems that the middle bead is special, but none of the links are privileged.

\begin{figure}
	\setlength{\figspacing}{5 mm}
	\centering
	\begin{subfigure}[b]{\textwidth}
		\centering
		\includegraphics[width=0.95\textwidth]{12/permuted_twisted}
		\caption{Twisted permuted paths.}
		\label{fig:permuted-twisted}
		\vspace{\figspacing}
	\end{subfigure}
	\begin{subfigure}[b]{\textwidth}
		\centering
		\includegraphics[width=\textwidth]{12/permuted_untwisted}
		\caption{Untwisted permuted paths.}
		\label{fig:permuted-untwisted}
	\end{subfigure}
	\caption[
		Untwisting of permuted paths
	]{
		Untwisting of permuted paths for the trace.
		The symmetry that is plainly visible in \subref{fig:permuted-untwisted} is not evident in \subref{fig:permuted-twisted}.
	}
\end{figure}

\section{Implementation details}

Our tool of choice in the present work is MMTK, which contains an implementation of LePIGS.\footnote{
	As of this writing, the implementation in question has not been pushed upstream to \url{https://bitbucket.org/khinsen/mmtk}.
	There are plans to rectify this in the near future.
}
Prior to this work, the implementation was implicitly targeted at $Z$-sector simulations.
However, in order to use the minimal estimator of the trace, we are required to sample from a different sector.
This required ensuring that the LePIGS algorithm works outside the $Z$-sector and making the appropriate changes to MMTK.

The three distinct parts of the LePIGS procedure are:
\begin{itemize}
	\item propagate free particles in normal modes;
	\item apply force fields to Cartesian momenta; and
	\item thermostat normal modes.
\end{itemize}
Implicit in these is the conversion between Cartesian coordinates and normal modes.
The conversion itself works for paths of any length and is independent of the force fields, so it is not problematic.

The same is true for the free particle propagation: once we have converted the coordinates to normal modes, the equations of motion are the same for all paths.
The only subtleties involved have to do with masses and frequencies, but these are simple to deal with.
The effective masses of the normal modes are the same as those of the beads ($\fict{m}_n$), so we do not need to do anything to them even if we change sectors.
The frequencies $\omega_k$ depend on $\tau$ (the ``length'' of each link) and $P$ (the number of beads in the path).
The former is unchanged between the sectors we are considering, but the latter is a property of the paths that is now allowed to change.

Application of the force fields is similarly straightforward: some of the forces may be scaled or missing entirely, so the changes to the momenta will be modified accordingly.

Thermostatting of the normal modes is potentially tricky, due to the appearance of $\beta$, which in the derivation starts out as the imaginary-time propagation length.
Since some of the paths may change length, it may be tempting to adjust the value of $\beta$ for each path to account for the varying length.
However, that would be incorrect, because the $\beta$ which appears in the thermostat is related to the temperature of the simulation and is a global property.
Indeed, it is the same $\beta$ which appears in the exponent of \vref{eq:classical-Z}, and it is not affected by anything we might do to the connectivity of paths or scaling of potentials inside $\Hcl$.

Thus, to implement transitions to the sectors of interest, there were three requirements which had to be satisfied by MMTK:
\begin{itemize}
	\item it must be possible to scale the force fields arbitrarily;
	\item the integrator must deal with paths, rather than particles; and
	\item there must be a convenient mechanism for changing the force field scaling and path connectivity.
\end{itemize}

In order for the force fields to be scalable, it was necessary to add a function pointer field called \texttt{scale\_func} to the \texttt{PyFFEnergyTermObject} struct.
This function takes the new scaling value and is responsible for updating the force field to effect the change.
To declare that it has support for this mechanism, a force field must override \texttt{supportsDynamicScaling} to return \texttt{True}.
Pure Python force fields, which do not have direct access to the fields of \texttt{PyFFEnergyTermObject}, should implement a \texttt{setScaling} method; it is assigned automatically in the \texttt{PyEnergyTerm} Cython class from which all pure Python force field terms inherit.
The built-in restraint force fields, such as \texttt{HarmonicDistanceRestraint}, were modified to support this scaling; these modifications involve scaling the energy, gradients, and force constants by the required amount.

In order for the integrator to deal with paths explicitly (rather than with particles, which only map to paths exactly in the $Z$-sector), it was necessary to perform some subtle changes.
When calculating $\omega_\links$ (see \vref{eq:omega-links}) in \texttt{propagateOscillators}, \texttt{springEnergyNormalModes}, and \texttt{applyThermostat}, it would no longer be correct to determine $\tau$ as $\beta / \links$ (since the effective $\beta$ of each path would change with $\links$, while $\tau$ remains constant), so a particular value of $\tau$ was assigned to each path, and this value does not change even as the paths are broken and recombined.
On the other hand, it was important \emph{not} to change the $\beta$ which appears in the random force component of \texttt{applyThermostat}, as this has to do with the ``physical'' temperature of the simulation, not the length of the path.

To keep track of the connectivity of the paths, the global array \texttt{connectivity} was introduced, alongside the existing \texttt{bead\_data}.
This array consists of three columns and as many rows as there are beads in the simulation; the beads of a single path are stored in contiguous rows of the array.
Each of the three columns stores a different kind of information about the beads:
\begin{enumerate}[start=0]
	\item the index of the bead in the universe, which acts as a pointer into the existing data structures;
	\item the positive length of the path (used as a sentinel to signal the first bead of a path) \emph{or} the negative offset to the beginning of the path; and
	\item a flag indicating whether the path is closed (as in finite temperature simulations) or open (as in ground state simulations).
\end{enumerate}
Two additional consequences of this new array were the ability to combine the finite temperature and ground state integrators, reducing code duplication, and the ability to open and close paths during a simulation.
This connectivity data may be written to the output trajectory file and analyzed later.
The ability to view the connectivity as it changed during a simulation was added to the \texttt{tviewer} application which is included with MMTK.

In order to abstract away the above details for the end user, the Cython class \texttt{PIReconnector} was provided.
If a class inheriting from \texttt{PIReconnector} is given to the path integral integrator, its \texttt{step} method is called at the end of each integrator step, allowing the user to manipulate the simulation.
For convenience, the following methods are provided in \texttt{PIReconnector}:
\begin{itemize}
	\item \texttt{getScaling(self, Py\_ssize\_t t)} to get the scaling of a force field term;
	\item \texttt{setScaling(self, Py\_ssize\_t t, double x)} to set the scaling of a force field term;
	\item \texttt{openPath(self, Py\_ssize\_t p)} to open a closed path (\textit{i.e.} cyclic to linear);
	\item \texttt{closePath(self, Py\_ssize\_t p)} to close an open path (\textit{i.e.} linear to cyclic);
	\item \texttt{breakPath(self, Py\_ssize\_t p)} to break a single path into two paths; and
	\item \texttt{joinPaths(self, Py\_ssize\_t p1, Py\_ssize\_t p2)} to join two paths into a single path.
\end{itemize}
The four methods for operating on paths allow the user to achieve any connectivity with the restriction that a single bead may be connected via no more than two links.
Additionally, because one can manipulate the connectivity during the simulation, it is possible to perform the sort of updates discussed in ref.~\cite{herdman2014path} which are necessary to preserve permutation symmetry with broken paths.
Although the idea is not pursued in the present work, the above methods in principle also allow for sampling of the grand canonical ensemble via the worm algorithm: previously ``hidden'' beads may be added to the simulation from a reservoir by joining them to a path and turning on their force field terms.


\subsection{Simulated link}

\label{subsec:simulated-link}

One simple test for the implementation of broken paths is to break an open PIGS path into two parts and add a force field between the new free ends that simulates the removed link.
This should allow us to sample from an identical ensemble, so the distribution of all bead positions (including the middle bead) should be unchanged.
Since we know what the middle bead distribution should be for a harmonic oscillator, we perform this test for a single particle in a harmonic trap and compare to the exact distribution.

The Hamiltonian of the one-dimensional harmonic oscillator with mass $m$ and angular frequency $\omega$ is
\begin{align}
	\hat{H}
	&= \frac{\hat{p}^2}{2 m} + \frac{1}{2} m \omega^2 \hat{q}^2,
		\label{eq:harmonic-oscillator-hamiltonian}
\end{align}
so its ground state wavefunction is~\cite[440-441,492]{messiah1999quantum}
\begin{align}
	\psi_0(q)
	&= \left( \frac{m \omega}{\pi \hbar} \right)^\frac{1}{4} e^{-\frac{m \omega}{2 \hbar} q^2}
		\label{eq:ho-position-wf}
\end{align}
and the corresponding diagonal density is
\begin{align}
	\rho(q)
	&= \abs{\psi_0(q)}^2
	= \left( \frac{m \omega}{\pi \hbar} \right)^\frac{1}{2} e^{-\frac{m \omega}{\hbar} q^2}.
		\label{eq:ho-position-distribution}
\end{align}
This is what we expect to see for the middle bead distribution when we have converged in both the $\beta \to \infty$ and $\tau \to 0$ limits.
We will remove the link from bead $M - 1$ to bead $M$ (where $M$, as usual, is the index of the middle bead), which corresponds to dividing
\begin{align}
	\expb{-\frac{m}{2 \hbar^2 \tau} \left( q\bead{M-1} - q\bead{M} \right)^2}
\end{align}
out of the distribution.
In order to reinsert this in the form of a force field, we use the same approach as for trial functions in ref.~\cite{schmidt2014inclusion}.
The potential $V(\vec{q})$ for a force field enters the distribution as
\begin{align}
	e^{-\tau V(\vec{q})},
\end{align}
so the potential we require is the harmonic distance restraint
\begin{align}
	V(q\bead{M-1}, q\bead{M})
	&= \frac{m}{2 \hbar^2 \tau^2} \left( q\bead{M-1} - q\bead{M} \right)^2.
\end{align}

We choose the mass $m$ to be that of a single electron and the angular frequency to be $\omega = \SI{1}{\kelvin}$.
With the parameters $\beta = \SI{8}{\per\kelvin}$, $\links = 256$, $\tau = \SI{3.125e-2}{\per\kelvin}$, $\dt = \SI{0.1}{\pico\second}$, $\gamma\bead{0} = \SI{0.1}{\per\pico\second}$, and $\num{1e6}$ steps, we get the distribution in \cref{fig:simulated-link-regular} when we sample regularly.
If we remove the link, we instead get the distribution in \cref{fig:simulated-link-broken}, which, as expected, does not match the ``exact'' value.
The actual test is, of course, to simulate the link using an explicit harmonic distance restraint, as in \cref{fig:simulated-link-fixed}, where we see that the distribution is restored.

\begin{figure}
	\centering
	\begin{subfigure}[b]{\textwidth}
		\includegraphics[width=\textwidth]{12/simulated_link_regular}
		\caption{
			Sampling a regular PIGS path in a harmonic trapping potential.
		}
		\label{fig:simulated-link-regular}
	\end{subfigure}
	\begin{subfigure}[b]{\textwidth}
		\includegraphics[width=\textwidth]{12/simulated_link_broken}
		\caption{
			Sampling a broken PIGS path in a harmonic trapping potential.
			The break is between the middle bead (whose distribution is displayed) and an adjacent bead.
		}
		\label{fig:simulated-link-broken}
	\end{subfigure}
	\begin{subfigure}[b]{\textwidth}
		\includegraphics[width=\textwidth]{12/simulated_link_fixed}
		\caption{
			Sampling a fixed PIGS path in a harmonic trapping potential.
			The break in the path is filled with an explicit harmonic distance restraint.
		}
		\label{fig:simulated-link-fixed}
	\end{subfigure}
	\caption[
		Broken path middle bead distribution
	]{
		Middle bead distribution of a harmonic oscillator in various path configurations.
		Dashed curves are the exact harmonic oscillator ground state density.
	}
\end{figure}


\subsection{Momentum distribution}

Another way to check that the implementation works correctly is to look at the harmonic oscillator momentum distribution.
We will consider the same system as in \cref{subsec:simulated-link}, but we will actually use the broken path to our advantage.
The momentum distribution, like the position distribution in \cref{eq:ho-position-distribution}, is the square of the magnitude of the wavefunction, but in the momentum representation.
The momentum representation of a wavefunction may be obtained by the Fourier transform of the position representation~\cite{ceperley1995path}, so we obtain (by \vref{eq:gaussian-integral-amu})
\begin{subequations}
\begin{align}
	\psi_0(p)
	&= \frac{1}{\sqrt{2 \pi \hbar}} \int\! \dif q \, e^{-\frac{i}{\hbar} p q} \psi_0(q)
	= \left( \frac{m \omega}{4 \pi^3 \hbar^3} \right)^\frac{1}{4} \int\! \dif q \, e^{-\frac{m \omega}{2 \hbar} q^2 - \frac{i}{\hbar} p q}
	= \left( \frac{1}{\pi \hbar m \omega} \right)^\frac{1}{4} e^{-\frac{1}{2 \hbar m \omega} p^2} \\
	\rho(p)
	&= \abs{\psi_0(p)}^2
	= \left( \frac{1}{\pi \hbar m \omega} \right)^\frac{1}{2} e^{-\frac{1}{\hbar m \omega} p^2}.
\end{align}
\end{subequations}
Both of the densities so far described are \emph{diagonal} and may be expressed in terms of the more general \emph{off-diagonal} densities
\begin{subequations}
\begin{align}
	\rho(q)
	&= \rho(q ; q)
	= \braket{q | \hat{\rho} | q} \\
	\rho(p)
	&= \rho(p ; p)
	= \braket{p | \hat{\rho} | p}.
\end{align}
\end{subequations}

The translation operator~\cite[651]{messiah1999quantum}
\begin{align}
	\hat{T}(x)
	&= \expb{- \frac{i}{\hbar} x \hat{p}}
\end{align}
has the action~\cite[650]{messiah1999quantum}
\begin{align}
	\hat{T}(x) \ket{q}
	&= \ket{q + x},
\end{align}
where $\ket{q}$ is a position state ket and $\hat{p}$ is the momentum operator along the same direction as $q$.
Applying the Fourier transform to the momentum distribution, since the trace is independent of basis, we find
\begin{subequations}
\begin{align}
	\mathcal{F}[\rho(p)](\delta q)
	&= \frac{1}{\sqrt{2 \pi \hbar}} \int\! \dif p \, \expb{-\frac{i}{\hbar} p (\delta q)} \rho(p) \\
	&= \frac{1}{\sqrt{2 \pi \hbar}} \int\! \dif p \, \braket{p | \hat{\rho} \hat{T}(\delta q) | p} \\
	&= \frac{1}{\sqrt{2 \pi \hbar}} \Tr{\left[ \hat{\rho} \hat{T}(\delta q) \right]} \\
	&= \frac{1}{\sqrt{2 \pi \hbar}} \int\! \dif q \, \braket{q | \hat{\rho} \hat{T}(\delta q) | q} \\
	&= \frac{1}{\sqrt{2 \pi \hbar}} \int\! \dif q \, \rho(q, q + \delta q).
\end{align}
\end{subequations}
We define the quantity
\begin{align}
	\rho(\delta q)
	&= \int\! \dif q \, \rho(q, q + \delta q),
		\label{eq:rho-delta-q}
\end{align}
which is dimensionless, unlike the densities we have seen so far, and normalized so that $\rho(\delta q = 0) = 1$.
For a harmonic oscillator, it is given by
\begin{align}
	\rho(\delta q)
	&= \left( \frac{1}{\pi \hbar m \omega} \right)^\frac{1}{2} \int\! \dif p \, e^{-\frac{1}{\hbar m \omega} p^2 - \frac{i}{\hbar} p (\delta q)}
	= e^{-\frac{m \omega}{4} (\delta q)^2}.
\end{align}
It is clear that we may recover the momentum distribution from this quantity via the inverse Fourier transform:
\begin{align}
	\rho(p)
	&= \frac{1}{\sqrt{2 \pi \hbar}} \mathcal{F}\inv[\rho(\delta q)](p).
\end{align}

In order to figure out how to estimate $\rho(\delta q)$, we write it in the more suggestive form
\begin{align}
	\rho(\delta q)
	&= \int\! \dif q \, \braket{0 | q + \delta q} \braket{q | 0}.
\end{align}
If we were to expand each ground state $\ket{0}$ into half of a PIGS path, we would see that the halves only meet in the $\delta q = 0$ case.
Indeed, we may write this as
\begin{align}
	\rho(\delta q)
	&= \hugefrac{
			\int\! \dif q \, \symbdist{12/dist_one_particle_offset}
		}{
			\int\! \dif q \, \symbdist{12/dist_one_particle}
		}.
\end{align}
We have been careful to include the central potential in the diagrams, as it is important for this problem.
The separation between the ``real'' and ``virtual'' beads at $M$ is, of course, $\delta q$:
\begin{align}
	\symbdistdq{12/dist_one_particle_offset_labelled}.
\end{align}
We may therefore estimate $\rho(\delta q)$ by sampling from the broken distribution
\begin{align}
	\symbdist{12/dist_one_particle_minimal}
\end{align}
and using the ratio estimator
\begin{align}
	\hugefrac{
		\bigmean{\symb{12/link_one_particle_offset}}
	}{
		\bigmean{\symb{12/link_one_particle}}
	},
\end{align}
thereby testing the broken path implementation in MMTK.
Note that unlike the estimators we've encountered so far, which have not depended on any parameters, this estimator is a function of $\delta q$.

For the simulation, we used the same parameters as in \cref{subsec:simulated-link}, with the exception that configurations are sampled every $\SI{1}{\pico\second}$ rather than every $\SI{0.1}{\pico\second}$.
The results for $\rho(\delta q)$ and $\rho(p)$ for the harmonic oscillator system are shown in \cref{fig:momentum-dist}.
While the raw estimator output differs slightly from the expected curve, the Fourier transform provides a sufficient amount of smoothing that the obtained momentum distribution is indistinguishable from the expected one.

\begin{figure}
	\setlength{\figspacing}{10 mm}
	\centering
	\begin{subfigure}[b]{\textwidth}
		\includegraphics[width=\textwidth]{12/momentum_dist_untransformed}
		\caption{
			The quantity from \cref{eq:rho-delta-q} for a harmonic oscillator.
		}
		\label{fig:momentum-dist-untransformed}
		\vspace{\figspacing}
	\end{subfigure}
	\begin{subfigure}[b]{\textwidth}
		\includegraphics[width=\textwidth]{12/momentum_dist}
		\caption{
			Momentum distribution of a harmonic oscillator, obtained as the Fourier transform of the function in \subref{fig:momentum-dist-untransformed}.
		}
	\end{subfigure}
	\caption[
		Broken path momentum distribution
	]{
		Momentum distribution of a harmonic oscillator (including the unprocessed estimator results).
		Dashed curves are exact harmonic oscillator results.
	}
	\label{fig:momentum-dist}
\end{figure}

\section{Model system}

As our benchmark system for calculating the Rényi entropy, we use what is arguably the simplest non-trivial system for which particle entanglement can be defined: two harmonically-coupled harmonic oscillators.
This system is described in detail (including analytic solutions) in \vref{chap:oscillators}.
We consider the one-dimensional variant of the problem, with the Hamiltonian
\begin{align}
	\hat{H}
	&= \frac{\hat{p}_A^2}{2 m} + \frac{\hat{p}_B^2}{2 m}
		+ \frac{1}{2} m \omega_0^2 (\hat{q}_A^2 + \hat{q}_B^2)
		+ \frac{1}{2} m \omegaint^2 (\hat{q}_A - \hat{q}_B)^2.
\end{align}
We choose the parameters $m = m_\mathrm{e}$ (mass of an electron) and $\omega_0 = \SI{1}{\kelvin}$; we vary the coupling strength $\omegaint$ in order to vary the entanglement of the system.
These parameters are chosen specifically so that we can replicate the results of ref.~\cite{herdman2014path} using a molecular dynamics approach (instead of using Monte Carlo).
This system lends itself to a natural particle partitioning: one partition containing solely particle $A$ and the other $B$.

We must optimize the usual parameters for this system: $\beta$, $\tau$, $\dt$, and $\gamma\bead{0}$.
In order to be able to use a smaller value for $\beta$ (and therefore fewer beads, leading to shorter simulation times), we use a trial function that is very similar to the exact ground state.
Specifically, we use \vref{eq:oscillators-wf-1d}, but we scale the mass down to $m / 2$, making it closer to a uniform trial function.
This trial function is also used for the energy convergence study in \vref{sec:oscillators-energy-convergence}.
Unless otherwise specified, the parameter values from \cref{tab:model-parameters} are used for this model system.

\begin{table}
	\rowcolors{2}{}{_row}
	\begin{center}
	\begin{tabular}{ c | S S[table-format=1.3] S[table-format=3] S[table-format=1.2] c }
		\toprule
		$\omegaint / \omega_0$ & {$\beta / \si{\per\kelvin}$} & {$\tau / \si{\per\kelvin}$} & {$P$} & {$\dt / \si{\pico\second}$} & $\gamma\bead{0} / \si{\per\pico\second}$ \\
		\midrule
		0 & 3.0 & 0.25 & 13 & 0.5 & 0.1 \\
		1 & 3.5 & 0.125 & 29 & 0.5 & 0.1 \\
		2 & 4.0 & 0.1 & 41 & 0.25 & 0.1 \\
		4 & 4.5 & 0.05 & 91 & 0.2 & 0.1 \\
		8 & 5.0 & 0.025 & 201 & 0.1 & 0.1 \\
		\bottomrule
	\end{tabular}
	\end{center}
	\caption[
		Selected parameters for coupled harmonic oscillators
	]{
		Selected parameters for the model system of coupled harmonic oscillators.
	}
	\label{tab:model-parameters}
\end{table}


\subsection{Primitive estimator}

Because we expect the two estimators introduced in \cref{sec:estimators} to behave differently, we perform separate convergence studies.
We first look at the primitive estimator, defined in \vref{eq:renyi-estimator-primitive}, which requires us to use the regular sampling distribution.
This estimator is a ratio of two quantities, both of which become exponentially small with decreasing $\tau$.
This makes it challenging to sample well and is at odds with the need to decrease $\tau$ to remove the error due to the Trotter factorization.
Consequently, the results are not good, but are provided for comparison with the minimal estimator.

We start our optimization with the friction.
Since the goal when optimizing friction is to increase the efficiency of sampling, this should be reflected by a smaller standard error of the mean.
To estimate the error, we use the binning analysis described in ref.~\cite{ambegaokar2010estimating} for frictions spanning several orders of magnitude.
To get a feel for the effects of friction on the error, we look at \cref{fig:primitive-frictions-binning}.
All the curves in both plots plateau, which gives the illusion that the simulations are long enough for the error to converge.
However, even though the two plots were generated using simulations with the same parameters, they are drastically different.

\begin{figure}
	\setlength{\figspacing}{5 mm}
	\centering
	\begin{subfigure}[b]{\textwidth}
		\includegraphics[width=\textwidth]{12/primitive_frictions_binning_a}
		\caption{}
		\vspace{\figspacing}
	\end{subfigure}
	\begin{subfigure}[b]{\textwidth}
		\includegraphics[width=\textwidth]{12/primitive_frictions_binning_b}
		\caption{}
	\end{subfigure}
	\caption[
		Error convergence for primitive estimator
	]{
		Convergence of error with binning level for different frictions using the primitive estimator.
		$\omegaint = \SI{4}{\kelvin}$, $\num{1e6}$ steps.
		The two plots show results from different simulations using identical parameters.
	}
	\label{fig:primitive-frictions-binning}
\end{figure}

\begin{figure}
	\setlength{\figspacing}{5 mm}
	\centering
	\begin{subfigure}[b]{\textwidth}
		\includegraphics[width=\textwidth]{12/primitive_frictions_beta_tau}
		\caption{
			Minimization of error with friction for different values of $\beta$ and $\links$.
			$\omegaint = \SI{4}{\kelvin}$.
		}
		\label{fig:primitive-frictions-beta-tau}
		\vspace{\figspacing}
	\end{subfigure}
	\begin{subfigure}[b]{\textwidth}
		\includegraphics[width=\textwidth]{12/primitive_frictions_omega_int}
		\caption{
			Minimization of error with friction for different values of $\omegaint$.
		}
		\label{fig:primitive-frictions-omega-int}
	\end{subfigure}
	\caption[
		Friction optimization for primitive estimator
	]{
		Friction optimization for the primitive estimator with different parameter sets.
		$\num{1e6}$ steps.
		Some of the large magnitude values have been cut off to emphasize the smaller values.
	}
	\label{fig:primitive-frictions-parameters}
\end{figure}

\begin{figure}
	\setlength{\figspacing}{5 mm}
	\centering
	\begin{subfigure}[b]{\textwidth}
		\includegraphics[width=\textwidth]{12/primitive_histogram}
		\caption{
			Distribution of all traces in a $\num{1e6}$ step simulation.
			Note the outlier to the extreme right, causing the loss of detail on the left.
		}
		\vspace{\figspacing}
	\end{subfigure}
	\begin{subfigure}[b]{\textwidth}
		\includegraphics[width=\textwidth]{12/primitive_histogram_pruned}
		\caption{
			Distribution of only those traces in a $\num{1e6}$ step simulation that fell below a cutoff value.
		}
	\end{subfigure}
	\caption[
		Distribution of traces for primitive estimator
	]{
		Distribution of traces for the primitive estimator.
		The logarithmic scale on the $y$-axis is necessary to make anything other than the first few bins visible.
	}
	\label{fig:primitive-histogram}
\end{figure}

\begin{figure}
	\setlength{\figspacing}{5 mm}
	\centering
	\begin{subfigure}[b]{\textwidth}
		\includegraphics[width=\textwidth]{12/primitive_entropy_beta}
		\caption{
			Convergence of the entropy with $\beta$.
		}
		\label{fig:primitive-entropy-beta}
		\vspace{\figspacing}
	\end{subfigure}
	\begin{subfigure}[b]{\textwidth}
		\includegraphics[width=\textwidth]{12/primitive_entropy_tau}
		\caption{
			Convergence of the entropy with $\tau$.
		}
		\label{fig:primitive-entropy-tau}
		\vspace{\figspacing}
	\end{subfigure}
	\begin{subfigure}[b]{\textwidth}
		\includegraphics[width=\textwidth]{12/primitive_entropy_dt}
		\caption{
			Convergence of the entropy with $\dt$.
		}
		\label{fig:primitive-entropy-dt}
	\end{subfigure}
	\caption[
		Convergence of entropy with primitive estimator
	]{
		Unsuccessful convergence of the second Rényi entropy of the coupled oscillators with $\beta$, $\tau$, and $\dt$ using the primitive estimator.
		$\num{1e6}$ steps.
		\explainplotentropy{}
	}
\end{figure}

We might hope for a more useful landscape if we look from a different perspective, as in \cref{fig:primitive-frictions-parameters}, but almost no insights are to be had from these plots either.
We may make one minor observation: in \cref{fig:primitive-frictions-beta-tau}, we see that the curves for large $\tau$ (few beads, not yet converged) are better behaved than the others.
Those curves are not useful, but they do foreshadow the poor behaviour we expect to see when we try to converge the entropy by decreasing $\tau$.
This is confirmed in \cref{fig:primitive-frictions-omega-int}, where the curve corresponding to the system with no coupling is the only one that looks reasonable, and it happens to be the one with the largest $\tau$ value.

Undeterred, we choose an arbitrary friction of $\SI{0.1}{\per\pico\second}$ and press on.
At this point, it would be nice to see an example of how the values we sample are distributed.
This is shown in \cref{fig:primitive-histogram}, and the distribution doesn't look good: there are sporadic large values, which are difficult to sample.

The curious reader may very well want to know what happens if we use this estimator to estimate the second Rényi entropy.
As expected, the results are not particularly impressive.
There is not much to be gained by examining \cref{fig:primitive-entropy-beta,fig:primitive-entropy-dt}, but \cref{fig:primitive-entropy-tau} holds some explanations for us.
We can see that there is no hope of converging with $\beta$ or $\dt$, because we cannot choose a $\tau$ that is sufficiently small.
When we try to do so, we tend to underestimate the trace, leading us to overestimate the entropy.
Having seen the distribution, we should not be surprised by the small error bars on points that are too high: in those cases, we underestimate the trace by not sampling enough of the large outliers, so we don't even realize that we've underestimated it.

The reader should not be discouraged at this point, as the primitive estimator at small $\tau$ is expected to behave poorly.
It is difficult to expect anything reasonable from a distribution such as that in \cref{fig:primitive-histogram}, where the spread of values spans many orders of magnitude.


\subsection{Minimal estimator}

For the minimal estimator, we perform the same analyses as above.
We again start by optimizing the friction.
Judging from \cref{fig:minimal-frictions-binning}, we have enough steps in the simulation for the error to converge.
Looking at \cref{fig:minimal-frictions-parameters}, we see reasonable behaviour whether we have large or small $\tau$.
More importantly, there is a clear minimum in the error at $\gamma\bead{0} = \SI{0.1}{\per\pico\second}$ for all coupling strengths, which is rather convenient.

Having settled on a centroid friction, we look at the distribution of values for the numerator and denominator components in \cref{fig:minimal-histogram} and we see pleasant shapes that only extend as far as unity.
This means that we avoid the issue of arbitrarily large outliers that we saw with the primitive estimator.
Thus, we are able to make the well-behaved convergence plots in \cref{fig:minimal-entropy}.

As shown in \cref{fig:minimal-entropy-zoomed}, by increasing the number of steps (and therefore the time required to run the simulation), we are able to reduce the error bars to the same magnitude as the residual systematic error.
The largest relative error is about $\SI{0.3}{\percent}$.

Finally, now that we know that the minimal estimator works well for the second Rényi entropy, we apply it to a range of coupling strengths.
The results are shown in \cref{fig:minimal-entropy-all}.

\begin{figure}
	\setlength{\figspacing}{5 mm}
	\centering
	\begin{subfigure}[b]{\textwidth}
		\includegraphics[width=\textwidth]{12/minimal_frictions_binning_num}
		\caption{
			Error in the numerator component.
		}
		\vspace{\figspacing}
	\end{subfigure}
	\begin{subfigure}[b]{\textwidth}
		\includegraphics[width=\textwidth]{12/minimal_frictions_binning_denom}
		\caption{
			Error in the denominator component.
		}
		\vspace{\figspacing}
	\end{subfigure}
	\begin{subfigure}[b]{\textwidth}
		\includegraphics[width=\textwidth]{12/minimal_frictions_binning_quot}
		\caption{
			Error in the quotient.
		}
	\end{subfigure}
	\caption[
		Error convergence for minimal estimator
	]{
		Convergence of error with binning level for different frictions using the minimal estimator.
		$\omegaint = \SI{4}{\kelvin}$, $\num{1e6}$ steps.
	}
	\label{fig:minimal-frictions-binning}
\end{figure}

\begin{figure}
	\setlength{\figspacing}{5 mm}
	\centering
	\begin{subfigure}[b]{\textwidth}
		\includegraphics[width=\textwidth]{12/minimal_frictions_beta_tau}
		\caption{
			Minimization of error with friction for different values of $\beta$ and $\links$.
			$\omegaint = \SI{4}{\kelvin}$.
		}
		\vspace{\figspacing}
	\end{subfigure}
	\begin{subfigure}[b]{\textwidth}
		\includegraphics[width=\textwidth]{12/minimal_frictions_omega_int}
		\caption{
			Minimization of error with friction for different values of $\omegaint$.
		}
	\end{subfigure}
	\caption[
		Friction optimization for minimal estimator
	]{
		Friction optimization for the minimal estimator with different parameter sets.
		$\num{1e6}$ steps.
	}
	\label{fig:minimal-frictions-parameters}
\end{figure}

\begin{figure}
	\setlength{\figspacing}{5 mm}
	\centering
	\begin{subfigure}[b]{\textwidth}
		\includegraphics[width=\textwidth]{12/minimal_histogram_num}
		\caption{
			Distribution of the numerator estimator in a $\num{1e6}$ step simulation.
		}
		\vspace{\figspacing}
	\end{subfigure}
	\begin{subfigure}[b]{\textwidth}
		\includegraphics[width=\textwidth]{12/minimal_histogram_denom}
		\caption{
			Distribution of the denominator estimator in a $\num{1e6}$ step simulation.
		}
	\end{subfigure}
	\caption[
		Distribution of components for minimal estimator
	]{
		Distribution of numerator and denominator components for the minimal estimator.
		The logarithmic scale on the $y$-axis is necessary to make anything other than the first few bins visible.
	}
	\label{fig:minimal-histogram}
\end{figure}

\begin{figure}
	\setlength{\figspacing}{5 mm}
	\centering
	\begin{subfigure}[b]{\textwidth}
		\includegraphics[width=\textwidth]{12/minimal_entropy_beta}
		\caption{
			Convergence of the entropy with $\beta$.
		}
		\vspace{\figspacing}
	\end{subfigure}
	\begin{subfigure}[b]{\textwidth}
		\includegraphics[width=\textwidth]{12/minimal_entropy_tau}
		\caption{
			Convergence of the entropy with $\tau$.
		}
		\vspace{\figspacing}
	\end{subfigure}
	\begin{subfigure}[b]{\textwidth}
		\includegraphics[width=\textwidth]{12/minimal_entropy_dt}
		\caption{
			Convergence of the entropy with $\dt$.
		}
	\end{subfigure}
	\caption[
		Convergence of entropy with minimal estimator
	]{
		Successful convergence of the second Rényi entropy of the coupled oscillators with $\beta$, $\tau$, and $\dt$ using the minimal estimator.
		$\num{1e6}$ steps.
		\explainplotentropy{}
	}
		\label{fig:minimal-entropy}
\end{figure}

\begin{figure}
	\setlength{\figspacing}{5 mm}
	\centering
	\begin{subfigure}[b]{\textwidth}
		\includegraphics[width=\textwidth]{12/minimal_entropy_zoomed}
		\caption{
			Reasonably converged entanglement entropy.
			$\omegaint = \SI{4}{\kelvin}$.
		}
		\label{fig:minimal-entropy-zoomed}
		\vspace{\figspacing}
	\end{subfigure}
	\begin{subfigure}[b]{\textwidth}
		\includegraphics[width=\textwidth]{12/minimal_entropy_all}
		\caption{
			Entanglement entropy for various coupling strengths.
			The error bars look peculiar because they are smaller than the symbols.
		}
		\label{fig:minimal-entropy-all}
	\end{subfigure}
	\caption[
		Results for minimal estimator
	]{
		Detailed results for the minimal estimator.
		$\num{1e7}$ steps.
		\explainplotentropy{}
	}
\end{figure}

