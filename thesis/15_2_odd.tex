\section{PILE with odd $P$}

\label{sec:pile-odd}

When the PILE was introduced in ref.~\cite{ceriotti2010efficient}, the transformation to normal modes was only given for even $P$ ($n$ in the notation of the paper).
It is useful in practice to be able to apply the PILE in situations where $P$ is odd.
Since eq.~(18) of ref.~\cite{ceriotti2010efficient} simply states the result, we perform the derivation from the beginning.

The relevant difference between the PILE and LePIGS is that the former has closed paths.
This means that we may start the derivation the same way, but instead of \vref{eq:nm-matrix-full} we use the very similar matrix
\begin{align}
	\mat{A}
	&= \begin{pmatrix}
			\phantom{-}2 & -1 & \phantom{-}0 & \cdots & \phantom{-}0 & -1 \\
			-1 & \phantom{-}2 & -1 & \cdots & \phantom{-}0 & \phantom{-}0 \\
			\phantom{-}0 & -1 & \phantom{-}2 & \cdots & \phantom{-}0 & \phantom{-}0 \\
			\phantom{-}\vdots & \phantom{-}\vdots & \phantom{-}\vdots & \ddots & \phantom{-}\vdots & \phantom{-}\vdots \\
			\phantom{-}0 & \phantom{-}0 & \phantom{-}0 & \cdots & \phantom{-}2 & -1 \\
			-1 & \phantom{-}0 & \phantom{-}0 & \cdots & -1 & \phantom{-}2 \\
		\end{pmatrix}.
\end{align}
This is a circulant matrix, so we may directly write the eigenvalues and eigenvectors:~\cite{karner2003spectral}
\begin{subequations}
\begin{align}
	\lambda_k
	&= 2 - \left( e^{\frac{2 \pi i k}{P}} + e^{-\frac{2 \pi i k}{P}} \right)
	= 2 - 2 \cos{\left[ \frac{2 \pi k}{P} \right]}
	= 4 \sin^2{\left[ \frac{\pi k}{P} \right]} \\
	s_{k,n}\conj
	&= \sqrt{\frac{1}{P}} e^{\frac{2 \pi i n k}{P}}.
\end{align}
\end{subequations}
Thus, the transformation matrix $\mat{S}$ is the $(P \times P)$ \nomencl{DFT}{Discrete Fourier Transform} matrix, which makes it easy to obtain the normal mode coordinates.

This transformation matrix in general results in complex values, but we would prefer to keep things real.
Unlike in the open path case, not all the eigenvalues are distinct.
When $k \ne 0$ and $k \ne P/2$,
\begin{align}
	\lambda_k
	&= 4 \sin^2{\left[ \frac{\pi k}{P} \right]}
	= 4 \sin^2{\left[ \frac{\pi (P - k)}{P} \right]}
	= \lambda_{P-k}.
\end{align}
We can use any linear combination of eigenvectors with the same eigenvalue, so we use (for $0 < k < P/2$):
\begin{subequations}
\begin{align}
	s_{k,n}'
	&= \sqrt{\frac{1}{2 P}} \left( e^{\frac{2 \pi i n k}{P}} + e^{\frac{2 \pi i n (P - k)}{P}} \right)
	= \sqrt{\frac{1}{2 P}} \left( e^{\frac{2 \pi i n k}{P}} + e^{-\frac{2 \pi i n k}{P}} \right)
	= \sqrt{\frac{2}{P}} \cos{\left[ \frac{2 \pi n k}{P} \right]} \\
	s_{P-k,n}'
	&= i \sqrt{\frac{1}{2 P}} \left( e^{\frac{2 \pi i n k}{P}} - e^{\frac{2 \pi i n (P - k)}{P}} \right)
	= i \sqrt{\frac{1}{2 P}} \left( e^{\frac{2 \pi i n k}{P}} - e^{-\frac{2 \pi i n k}{P}} \right)
	= -\sqrt{\frac{2}{P}} \sin{\left[ \frac{2 \pi n k}{P} \right]}.
			\label{eq:pile-odd-sin-flip}
\end{align}
\end{subequations}
Thus, for even $P$, the matrix elements are given by
\begin{align}
	s_{k,n}'
	&= \begin{cases}
			\sqrt{\frac{1}{P}} & \text{if } k = 0 \\
			\sqrt{\frac{2}{P}} \cos{\left[ \frac{2 \pi n k}{P} \right]} & \text{if } 0 < k < \frac{P}{2} \\
			\sqrt{\frac{1}{P}} (-1)^n & \text{if } k = \frac{P}{2} \\
			\sqrt{\frac{2}{P}} \sin{\left[ \frac{2 \pi n k}{P} \right]} & \text{if } \frac{P}{2} < k < P
		\end{cases},
\end{align}
which is exactly eq.~(18) of ref.~\cite{ceriotti2010efficient}.\footnote{
	Except our matrix $\mat{S}'$ happens to be the transpose of their matrix $\mathbf{C}$.
}
The phase of the $\sin$ elements appears to be flipped relative to \cref{eq:pile-odd-sin-flip} because
\begin{align}
	\sin{\left[ \frac{2 \pi n (P - k)}{P} \right]}
	&= \sin{\left[ 2 \pi n - \frac{2 \pi n k}{P} \right]}
	= -\sin{\left[ \frac{2 \pi n k}{P} \right]}.
\end{align}
Finally, we recognize that when $P$ is odd, nothing in the derivation changes, but we no longer have to account for the $k = P/2$ case, so we may simply ignore it.
This results in the matrix elements
\begin{align}
	s_{k,n}'
	&= \begin{cases}
			\sqrt{\frac{1}{P}} & \text{if } k = 0 \\
			\sqrt{\frac{2}{P}} \cos{\left[ \frac{2 \pi n k}{P} \right]} & \text{if } 0 < k < \frac{P}{2} \\
			\sqrt{\frac{2}{P}} \sin{\left[ \frac{2 \pi n k}{P} \right]} & \text{if } \frac{P}{2} < k < P
		\end{cases}.
\end{align}
