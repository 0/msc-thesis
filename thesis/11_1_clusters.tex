\section{Molecular clusters}

It has now been nearly a century since the discovery of superfluidity in liquid helium, and over that time the phenomenon has been extensively studied both experimentally and theoretically~\cite{balibar2007discovery}.
However, it is only much more recently that studies of low-temperature helium clusters, rather than of bulk helium, have been performed.
While in bulk liquids one can directly observe manifestations of superfluidity (one only needs to search YouTube for ``superfluid helium''), the microscopic nature of molecular clusters makes it more difficult to detect their superfluidity.
After all, even asking whether a cluster is solid or liquid is itself not a well-defined question.
These are states of bulk materials, which do not suffer from finite-size effects.
Molecular clusters are instead described by the nebulous terms ``solid-like'' and ``liquid-like''~\cite{cuervo2008solid}.

The Andronikashvili experiment from 1946 used an ingenious method to study the behaviour of superfluid helium~\cite{grebenev1998superfluidity}.
It involved placing a probe of rotating disks in a container of helium and observing its effective moment of inertia as the helium was cooled below its superfluid transition temperature.
One would typically expect the moment of inertia to increase as a liquid is cooled, since greater viscosity should lead to greater drag.
For superfluid helium, the opposite was the case: the moment of inertia decreased, meaning that there was less drag on the probe, an indication of superfluidity.

More recently, a microscopic version of the Andronikashvili experiment was performed, using small helium droplets with \ce{OCS} acting as a molecular rotational probe~\cite{grebenev1998superfluidity}.
It was observed that the \ce{OCS} IR spectrum had broad peaks when placed in pure \ce{^3He} droplets, but that the peaks became sharper with the addition of \ce{^4He} to the droplets, revealing rotational structure.
These sharp peaks are an indication that the \ce{OCS} probe is rotating nearly freely in the droplets containing sufficient amounts of \ce{^4He}.
Thus, it turns out that finite-sized clusters may also exhibit superfluid effects and there is a way of quantifying these effects experimentally.

Although no bulk materials other than helium are currently known to act as superfluids, there is evidence that \paraH{} clusters can also display superfluidity~\cite{grebenev2000evidence}.
These findings are confirmed by several recent theoretical studies~\cite{li2010molecular,raston2012persistent,zeng2012simulating,zeng2013probing}.
Additionally, it has been suggested that some \paraH{} clusters have structure in their radial density and may be referred to as ``supersolid''~\cite{sindzingre1991superfluidity}.

Thus, \ce{^4He} and \paraH{} clusters are a reasonably good test subject for probing superfluidity and related properties.
Although we do not simulate such clusters in the present work, they are the main motivation behind the methods which are implemented and tested.
We hope to apply the presented theory and software to molecular clusters in the near future.
